\section{Further Work}
The main shortcoming of the testing I have carried out is the random instance generation described in \cref{randomGen}.
In all problems, the results such as \cref{arrivalWalkPlot} give strong evidence
that for the most basic algorithms like simulating the walk and value iteration,
my scheme for random problem generation does not generate 'hard' problems so investigation
into this could be carried out. Another extension
could be to implement strategy improvement and quadratic programming
to solve simple stochastic games and further compare.\\
In \citep{valueIterationTest} both of these issues are somewhat addressed when testing value iteration,
strategy improvement, and quadratic programming on simple stochatic games. The authors
build a library of extremal problems that are difficult for particular algorithms, as well as a more
sophisticated method of generating random problems than I have done here. Perhaps I can make a transitive
comparison of the binary search style algorithms to strategy improvement and quadratic programming given
we have both tested value iteration, although the difference in how we generate problems makes this tenuous. \\
Time could be spent on on optimizing the monotone function for shapley's stochastic games. 
The amount of time it takes to compute the monotone function has lead to only being able to test
shapley's stochastic games with small sizes, so an increase in performance here could provide
more useful comparison.
