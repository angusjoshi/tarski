\section{Discussion}
Inspecting the plots from \cref{resultsSec} clearly shows that
the simplest algorithm KT outperforms all other
algorithms in almost all cases. Interestingly, of the binary-search based
algorithms, FPS 
performs the best in all cases,
while CL and DQY alternate
which performs better in different cases. I now provide more detailed
analysis for each individual case.
\subsection{\arr Tests}
It can clearly be seen in \cref{arrivalMainPlot} and \cref{arrivalWalkPlot}
that simulating the arrival walk vastly outperforms the binary-search based
algorithms on randomly generated instances of the arrival problem.
The distinction is less clear when testing with the worst case instance
for the walk as seen in \cref{arrivalLongPlot}; FPS outperforms KT
in terms of query count for more than 15 vertices, and in terms of time for more than
20. This is somewhat unexpected as the number of steps in the walk on
the worst case instance is $\Theta(2^n)$, while the upper bound we
get from FPS when applied to the walk is 
$O(\log^{\lceil (n + 2)/3 \rceil } (2^n)) = O(n^{\lceil (n + 2)/3 \rceil})$,
which is clearly a weaker bound than $O(2^n)$. I believe the cause of this
is that the recursive binary search algorithms work particularly well
on this specific long walk instance; the fixpoint for this specific instance
will be something like $\vec{x} = (2^{n}, 2^{n - 1}, ..., 1)$, and the
binary search algorithms always start by querying midpoints which happen
to be powers of two so coincide exactly with the actual fixpoint.
This perhaps motivates further testing- are there other instances
for which FPS outperforms KT?  \\
In comparing the binary search style algorithms, it can be seen that
FPS performs best in all cases, and in particular performs better
than the asymptotically superior CL algorithm. The difference between
CL and DQY is less clear however; for random instances CL performs
better as seen in \cref{arrivalMainPlot},
but DQY performs better on the long walk instance as seen in \cref{arrivalLongPlot}.

\subsection{Simple Stochastic Games Tests}
Similarly to $\arr$, KT is the most performant algorithm for solving
random simple stochastic games - it even seems to be the case
that the number of queries goes \emph{down} as the number of vertices
goes up for the walk as seen in \cref{simpleWalkPlot}. 
I believe that this is a limitation of the method
of random instance generation which perhaps motivates further investigation
into methods for generating 'hard' simple stochastic games. It could also
be the case that testing with a fixed approximiation factor $\varepsilon$
and stopping probability $\beta$ causes this behaviour. \\
In comparing the binary search based algorithms, it can be seen in \cref{ssgMainTest} 
that
FPS is again the most performant, and DQY is the least performant. The difference
between FPS and CL is small in this case however. \\
In the test with varying approximation factor as seen in \cref{ssgApproxTest}, KT 
is again the most performant, with FPS the best of the binary search algorithms.
DQY is again the slowest, with CL in the middle. Varying the approximation factor
for the simple stochastic game problem has the effect of changing the height of the lattice
that is searched; if $\beta$ is the stopping probability, and $\varepsilon$ is the approximation
factor, the associated discretized function is 
defined on $\left[\left \lfloor \frac{1}{\beta \varepsilon}\right \rfloor\right]^d$.
Since KT runs in worst case complexity $O(Nd)$ where $N$ is the height of the lattice,
and FPS in $O(\log^{\lceil \frac{n + 2}{3} \rceil} N)$, one might expect that for
very small approximation factors that FPS is more performant. This is not shown in
the results however, so perhaps more investigation should be done into finding
instances of simple stochastic games which are 'hard' for KT.

\subsection{Shapley Stochastic Games Tests}
Testing on shapley's stochastic games was much more limited than the other two problems
as the associated monotone function took a lot longer to compute. It could be the case
the the LP solver that I used (\lstinline{soplex}) is not optimized for solving
many thousands of small LPs, or that solving LPs in this fashion will necessarily
take significantly more time than the functions for \arr and simple stochastic games. \\
In comparing the algorithms running on random instances, it can be seen in \cref{shapleyMainPlot}
and \cref{shapleyWalkPlot} that 
again KT is the most performant. DQY is the least performant on random instances, and the difference
between CL and FPS is small. Interestingly, there seems to be some association between
the parity of the dimension and the performance of CL and FPS. I believe this is because of the 
$\lceil \cdot \rceil$ in the exponent of their respective complexities, and the only reason
we don't see this in other plots is because we are measuring with less granularity on size. \\
The test with varying approximation factor as seen in \cref{shapleyApproxPlot} shows
again that KT is the most performant, and FPS is the most performant of the binary search algorithms.
There is not much difference between CL and DQY in this case. Similarly to simple stochastic games,
one would expect that for very small approximation factors that the binary search algorithms
perform better than KT - but again this is not seen in the results for these tests. This is again
perhaps motivation for more investigation in finding 'hard' stochastic games for the KT algorithm.
