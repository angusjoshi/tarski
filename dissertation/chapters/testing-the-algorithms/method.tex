\section{Method}
\subsection{Algorithm Implementation Detail}
\subsubsection{Implementation}
I implemented these algorithms in the progamming language C++. 
Complete source code can be found \href{https://github.com/angusjoshi/tarski}{here}
including all algorithms in \cref{stateAlgsChap},
and solvers for all problems in \cref{relatedProblemsChapter} using $\trsk$ algorithms.
Compilation and linking was done with
clang version 14.0.3 with the C++20 standard and \lstinline{-03} optimization settings.
Soplex\citep{soplex} was used as a dependency to solve linear programs
as part of the shapley's stochastic games solver.  \\
\subsubsection{Performance Improvements}
There are some performance optimizations that could be made. For simplicity of implementation\footnote{particularly
in implementing slices of functions as described in \cref{sliceDef}},
\lstinline{std::vector}s are shuffled and appended to unnecessarily, and I believe
performance could be gained by changing this. \lstinline{std::function} is the main abstraction for passing
the monotone functions around the system and are shown to be particularly innefficient in \citep{stdFunctionBad},
so I believe performance can be gained by changing this to something like the \lstinline{function_view}
described in \citep{stdFunctionBad}.
In performance profiling, I found \lstinline{soplex} was a bottleneck in solving shapley's stochastic games.
Perhaps \lstinline{soplex} is not optimized for solving a large number of very small LPs and a better alternative could be found.
There is also perhaps scope for using sensitivity
analysis as described in \citep{sensAnalysis} to reuse values from previous function queries to improve
solver performance; although this could be incredibly complex and not worth the effort.

\subsection{Random Problem Generation} \label{randomGen}
Instances of all three problems were generated randomly to facilitate testing. The method of randomization used
for each instance is detailed in this subsection. Throughout random numbers were generated using tools
from the \lstinline{<random>} header in the C++ standard library.
\subsubsection{\arr} \label{arrRandom}
Recall from \cref{arr} that an instance of the arrival problem consists of a directed graph with
a designated target vertex such that every vertex has exactly two labelled outgoing edges.
This leads to a natural notion of a random arrival instance on $n$ vertices $v_1, ..., v_n$.
Simply choose for each vertex $v_i$ the successors $s_0(v_i)$ and $s_1(v_i)$ uniformly at random
from the set of vertices, and note that it is without loss of generality to fix the target to be $v_n$.
Random instances for various fixed sizes of the $\arr$ problem were generated thusly for testing.

\subsubsection{Simple Stochastic Games} \label{ssgRandom}
Simple stochastic games do not have as natural a notion of random problem instances as $\arr$ for the following reasons,
\begin{itemize}
  \item vertices can be one of three types, 
  \item vertices can have different numbers of successors,
  \item chance vertices can have arbitrary probability distributions on their successors.
\end{itemize}
For simplicity, I generate a random simple stochastic game on $n$ vertices $v_1, ..., v_n$ as follows,
\begin{itemize}
  \item the type of each vertex is chosen uniformly at random from the three possibilities,
  \item all vertices have exactly two successors,
  \item the probability distribution on the two successors of a chance node is chosen by
    partitioning the interval $[0, 1]$ with a number chosen uniformly at random from the range $[0, 1]$.
  \item $v_n$ is fixed to be the target for the maximizer.
\end{itemize}

\subsubsection{Shapley's Stochastic Games} \label{shapleyRandom}
The degrees of freedom for defining an instance of shapley's stochastic games are as follows,
\begin{itemize}
  \item action sets can have arbitrary size at each state,
  \item for each joint action at each state, an arbitrary probability distribution on the all
    the states in the game can be chosen,
  \item payoffs for each joint action for each state can be chosen arbitrarily.
\end{itemize}
In order, these are addressed as follows,
\begin{itemize}
  \item both players have three actions at every state,
  \item payoff and successor matrices are all $3 \times 3$ (which follows from the above item),
  \item every entry in every successor matrix is a probability distribution on exactly two vertices.
    That is to say that at every state when a joint action is chosen the transition is chosen
    to be one of two states,
  \item every probability distribution in the successor matrix is chosen as a u.a.r. partition of $[0, 1]$
    as in the simple stochastic game case,
  \item all entries of the payoff matrices are chosen to be u.a.r. integers in the range $[-10, 10]$.
\end{itemize}
\subsubsection{Limitations}
Testing with random instances in this fashion is necessarily limited; the results
shown later give evidence that random generation in this fashion does not tend to generate 'hard' instances.
For instance, as will be shown in \cref{arrivalWalkPlot}, the length of the walk in a random instance
of the $\arr$ problem as described above seems to scale linearly with the size of the problem despite
the fact that in the worst case the walk can have an exponential length.
\subsection{Testing Protocol}
Separate tests were carried out for the three problems detailed in \cref{relatedProblemsChapter} as follows.
For all problems, all of the algorithms were tested with varying instance sizes. For simple stochastic games
and shapley's stochastic games all the algorithms were also tested with a fixed problem size and varying
approximation constant $\varepsilon$. In all tests, the number of queries to the monotone function is
measured, and the time to run the algorithm is measured. The measured time is precisely the time between
the function to run the algorithm is called, and the function returning with a fixpoint, so preprocessing
and other miscellaneous actions do not have an effect. All tests were repated 20 times with the
mean values recorded recorded. Different sizes were used for different algorithms in the same test
to ensure tests terminated in a reasonable amount of time. \\
The Kleene, Tarski \cref{kleeneTarski} was not tested directly on any of the monotone functions.
Instead, for shapley's and simple stochastic games, the continuous function is iterated on directly,
and for $\arr$ the walk is simulated directly. All of these are essentially the same as \cref{kleeneTarski}
but are slightly more performant due to skipping unnecessary scaling. \\
From here on I will denote the Fearnley, \pav, Savani algorithm described in \cref{fixDecompAlg}
as FPS, the Dang, Qi, and Ye algorithm descbribed in \cref{dQiYiAlg} as
DQY, Chen and Li described in \cref{monotoneDecompChap} as CL, and Kleene, Tarski 
described in \cref{kleeneTarski} as KT.
\subsubsection{\arr}
\begin{test}[Arrival Main] \label{arrMainTest}
  The three algorithms listed below were tested on random arrival instances as in \cref{arrRandom}
  with varying sizes between $3$ and $18$.
\end{test}
\begin{test}[Arrival Walk] \label{arrWalkTest}
  The arrival walk with preprocessing as in \cref{arrivalPreprocess} 
  was simulated to termination for random arrival instances with varying sizes
  between $10$ and $100000$.
\end{test}
\begin{test}[Long Arrival] \label{longArrivalTest}
  All algorithms were tested on the arrival instance with the
  longest possible walk, as seen in \cref{expLongArrival},
\end{test}
\subsubsection{Simple Stochastic Games}
\begin{test}[Simple Stochastic Game Main] \label{ssgMainTest}
  All algorithms were tested on random simple stochastic
  games as in \cref{ssgRandom} with varying sizes. The binary
  search algorithms were tested on sizes between $3$ and $13$.
  Value iteration was tested with sizes between $10$ and $100000$.
\end{test}
\begin{test}[Simple Stochastic Game Approximation] \label{ssgApproxTest}
  All four algorithms were tested on random simple stochastic
  games as in \cref{ssgRandom} with fixed size $N = 11$ and varying
  approximation constant $\varepsilon \in \{ 0.1, 0.01, 0.001, 0.0001, 0.00001, 0.000001 \}$.
\end{test}

\subsubsection{Shapley's Stochastic Games}
\begin{test}[Shapley Main] \label{shapleyMainTest}
  All algorithms were tested on random shapley's stochastic games
  as in \cref{shapleyRandom} with varying size. The binary search style algorithms
  were tested with sizes between $2$ and $6$. Value iteration was tested with
  sizes between $10$ and $40$.
\end{test}

\begin{test}[Shapley Approximation] \label{shapleyApproxTest}
  All algorithms were tested on random simple stochastic
  games as in \cref{shapleyRandom} with fixed size $N = 6$ and varying
  approximation constant $\varepsilon \in \{ 0.5, 0.1, 0.01, 0.001, 0.0001 \}$.
\end{test}
