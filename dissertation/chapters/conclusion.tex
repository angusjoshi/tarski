\chapter{Conclusions}
\section{Testing the Algorithms}
The main conclusions drawn from the practical section of this project
are the following,
\begin{itemize}
  \item the simplest iterative algorithms of value iteration and simulating the $\arr$ walk
    perform better in almost all cases on the model of random instances of $\arr$, simple stochastic games,
    and shapley's stochastic games described in \cref{randomGen},
  \item the asymptotically superior monotone decomposition \cref{monDecompAlg} tends
    to require more queries and time than the simpler fixpoint decomposition \cref{fixDecompAlg},
  \item the inner algorithm combined with the fixpoint decomposition \cref{fixDecompAlg} tends to be the most performant
    of all of the binary search algorithms.
  \item one case where the fixpoint decomposition \ref{fixDecompAlg} performs better than
    simulating the $\arr$ walk was found in \cref{arrivalLongPlot}, although I believe
    this to be anomalous due to the nature of the specific problem being particularly
    easy for the binary-search style algorithms.
\end{itemize}
The main limitation of testing was the methods of generating problems for testing;
randomly generated instances of the $\arr$ problem proved to be relatively easy
in comparison to the worst case, with similar findings for simple stochastic games
and shapley's stochastic games. The solver for shapley's stochastic games was
particularly slow so testing was restricted to lower dimensional versions of the problem.

\section{State of the Art Algorithms}
The key takeaway is the many theoretical questions relating to the $\trsk$ problem
remain open. These include,
\begin{itemize}
  \item \cref{tarskiFixedParameterTractable}. Is $\trsk$ fixed-parameter tractable? The inner algorithm
    from \citep{fasterTarski} described in \cref{innerAlgChap} makes this seem somewhat plausible. Is
    there notion of a 'witness' in higher than 3 dimensions?
  \item \cref{improvedLowerBound}. Is there an improved lower bound for the $\trsk$ problem in some dimension
    larger than 3? Perhaps the methods in proving the lower bound for the 2 dimensional variant in \citep{lowerBound}
    have an analogue in some dimension larger than 3.
  \item \cref{complete}. Is $\trsk$ complete for some natural complexity class? Although in \citep{lowerBound}
    inclusion of $\trsk$ in $\PPAD \cap \PLS$ is shown, there is no non-trivial complexity class for
    which $\trsk$ is known to be complete. Such a result could be useful on giving evidence on the hardness of $\trsk$.
\end{itemize}

\section{Future Work}
With regard to the practical testing of the $\trsk$ algorithms, some ideas for future improvements are as follows,
\begin{itemize}
  \item can problem generation for all three problems be improved?
    \begin{itemize}
      \item is there a method of generating random $\arr$ instances which have a comparable walk-length to the worst-case instance?
      \item are there 'long' $\arr$ instances other than the worst-case described in \cref{expLongArrival} for which the fixpoint decomposition still
        outperforms simulating the walk as in \cref{arrivalLongPlot}?
      \item does randomizing other parameters like number of successors, stopping probability,
        and payoff matrix entry size, of simple stochastic games and shapley's stochastic games have any effect on
        the relative performance of all of the algorithms?
    \end{itemize}
  \item can the performance of the implementation of solving shapley's stochastic games using $\trsk$ algorithms be improved?
    The slowness of solving many linear programs for the monotone function for shapley's stochastic games resulted
    in the experiment for this problem being limited in size.
\end{itemize}
For future work on the theoretical aspects of $\trsk$ and related problems, I propose the following problems as being worth some thought,
\begin{itemize}
  \item \cref{tightFourDimension}. Is there an algorithm for solving the four dimensional $\trsk(N, 4)$ problem
    in $O(\log^2 N)$ queries? Generalizing the inner algorithm to arbitrary dimensions is 
    potentially a very difficult problem, and
    working on the $\trsk(N, 4)$ problem would seem to be a reasonable intermediate step. A result in the positive
    could perhaps also be illuminating to higher dimensions.
  \item is there anything to be gleaned from studying the $\trsk$ problem in the context of monotone functions from instances of the $\arr$
    problem? $\arr$ problem instances have a very simple combinatorial structure and there are already interesting connections between
    the $\arr$ and $\trsk$ through the reduction given in \cref{arr}. Is there an interpretation of any of the features in the state
    of the art algorithms in the context of $\arr$? And do they provide any insight on how to make progress with the general problem?
\end{itemize}
%todo note on strategy improvement
