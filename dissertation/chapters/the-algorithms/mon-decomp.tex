%todo all of this.
\section{Monotone Decomposition} \label{monotoneDecompChap}
The algorithm described in \cref{fixDecompChapter} makes a decomposition
of $\trsk$ by decomposing into smaller fixpoint computation problems, with
the asymptotic improvement coming from the fact that there is an algorithm
to efficently solve the $\trsk(N, 3)$ problem. We can do better however;
Chen and Li describe in \citep{chenLi} a method of decomposing the $\trsks$
problem instead. To save space, this section will show the algorithms used
but for proofs of correcteness the reader is referred to \citep{chenLi}.
It will be simpler to give an overview after a definition of 
an auxiliary problem.
\newcommand{\trskst}{\textsc{Tarksi}^*}
\begin{definition}[$\trskst$, \citep{chenLi}] \label{trskst}
  The problem $\trskst(N, d)$ is given oracle access to a function $f : [N]^d \to \{-1, 0, 1\}^d$
  such that,
  \begin{enumerate}
    \item for each $x \in [N]^d$ and $i \in [d]$, $x_i + f(x)_i \in [N]^d$,
    \item for each $x, y \in [N]^d$ with $x \leq y$, $(x, 0) + f(x) \leq (y, 0) + f(y)$,
  \end{enumerate}
  find a point $x \in [N]^d$ such that $f(x) \geq 0$ or $f(x) \leq 0$.
\end{definition}
This function $f$ is designed to be an indicator of a monotone function $F : [N]^d \to [N]^d$;
that is if I define $f$ coordinatewise as 
$f(x)_i = \begin{cases} 1, & F(x)_i > x_i \\ 0, & F(x)_i = x_i \\ -1, & F(x)_i < x_i \end{cases}$
then $f$ satisfies the conditions in \cref{trskst}. The first condition is the requirement that
the codomain of $F$ is correct, and the second is monotonicity. It is also clear that $\trsks(N, d)$ 
trivially reduces to $\trskst(N, d)$. Chen and Li then wish to make decompositions from
$\trskst(N, a + b)$ into $\trskst(N, a)$ and $\trskst(N, a)$. The fixpoint decomposition
method \cref{fixDecompAlg} does not obviously apply here; it is not clear what to do with the extra
coordinate in the codomain, and further if I split into $f_l$ and $f_r$ similarly to \cref{fixDecompAlg},
if $(x_l, x_r) \in [N]^a \times [N]^b$, are found as monotone points of $f_l$ and $f_r$,
$x_l$ can be a monotone-down point, and $x_r$ a monotone up point so $(x_l, x_r)$ is not necessarily
a monotone point of $f$. Chen and Li's first step towards a solution is defining a refinement of the
$\trskst$ problem.
\begin{definition}[$\textsc{RefinedTarski}^*$, \citep{chenLi}]
  Given a function $f : [N]^d \to \{-1, 0, 1\}^{d + 1}$ as in \cref{trskst}, find
  a pair of points $\bot, \top \in [N]^d$ such that $\bot \leq \top$, for each $i \in [d]$,
  $f(\bot)_i \geq 0$, $f(\top)_i \leq 0$, and at least one of the following hold,
  \begin{enumerate}
    \item $f(\bot)_{d + 1} = 1$,
    \item $f(\top)_{d + 1} = -1$,
    \item $f(\bot) = f(\top) = 0$.
  \end{enumerate}
\end{definition}
Interestingly, Chen and Li show that this problem is no harder than $\trskst$ in the sense
that it can be solved in at most two queries to a $\trskst$ orcacle. Their algorithm for doing this is
shown in \cref{refinedTarskAlg}.
\begin{algorithm}[H]
  \caption{\citep{chenLi}} \label{refinedTarskAlg}
  \begin{algorithmic}[1]
    \Procedure{RefinedTarski$^*$}{\\
      \qquad $f : [N]^d \to \{-1, 0, 1\}^{d + 1}$ as in \cref{trskst},\\
      \qquad algorithm $A$ for solving $\trskst(N, d)$
    }
    \State $(\bot, \top) \gets (\vec{1}, \vec{N})$
    \State Let $g^+ : [N]^d \to [N]^{d+1}$ with $g^+(x)_i = 
      \begin{cases} 1, & i = d + 1 \text{ and } g(x)_{d + 1} \geq 0 \\ g(x)_i, & \text{otherwise}\end{cases}$
    \State $x \gets A(g^+)$ with bounds $\bot$, $\top$
    \State \If{$g^+(x) = -1$} $\top \gets x$ and \Return $(\bot, \top)$ \EndIf
    \State $\bot \gets x$
  \State Let $g^- : [N]^d \to [N]^{d+1}$ with $g^+(x)_i = 
    \begin{cases} -1, & i = d + 1 \text{ and } g(x)_{d + 1} \leq 0 \\ g(x)_i, & \text{otherwise}\end{cases}$
    \State $y \gets A(g^-)$ with bounds $\bot$, $\top$
    \If{$g^-(y)_{d + 1} = 1$} $\bot \gets y$ \algorithmicelse\ $\top \gets y$ \EndIf
    \State \Return $(\bot, \top)$
  \EndProcedure
  \end{algorithmic}
\end{algorithm}
Correctness is relatively simple to verify and is ommitted for brevity.
Now the main algorithm is described in \cref{monDecompAlg}.
\begin{algorithm}[h]
  \caption{\citep{chenLi}} \label{monDecompAlg}
  \begin{algorithmic}[1]
    \Procedure{MonotoneDecomposition}{\\
      \qquad $f : [N]^{a + b}  \to \{-1, 0, 1\}^{a + b + 1}$ as in \cref{trskst},\\
      \qquad algorithm $A$ for solving $\trsk(N, a)$, \\
      \qquad algorithm $B$ for solving $\trsk(N, b)$ 
    }
    \State $\mathrm{prev} \gets \emptyset$
    \Procedure{$f_r : [N]^b \to \{-1, 0, 1\}^{b + 1}$}{$x_r \in [N]^b$}
      \State $\bot_l \gets \bigvee\{p_l : (p_l, p_r) \in \mathrm{prev}, \; p_r \leq x_r \}$
      \State $\top_l \gets \bigwedge\{p_l : (p_l, p_r) \in \mathrm{prev}, \; p_r \geq x_r \}$
      \For{$j$ from $a + 1$ to $a + b + 1$ }
      \Procedure{$f_l^j : [N]^a \to \{-1, 0, 1\}^{a + 1}$}{$x_l \in [N]^a$}
        \State $y \gets f((x_l, x_r))$
        \State \Return $(y_1, ..., y_a, y_j)$.
      \EndProcedure
      \State $(\bot_l^j, \top_l^j) \gets \Call{RefinedTarski$^*$}{f_l^j, A}$ with bounds $\bot_l$, $\top_l$
      \State $(\bot_l, \top_l) \gets (\bot_l^j, \top_l^j)$
      \EndFor

      \State $\mathrm{prev} \gets \mathrm{prev} \cup \{(\bot_l, x_r)\}$
      \State $x \gets f((\bot_l, x_r))$
      \State \Return $f(x_{a + 1}, ..., x_{b + 1})$.
    \EndProcedure
    \State $x_r^* \gets B(f_r)$
    \State let $(\bot_l, \top_l)$ be the bounds found when evaluating $f_r(x_r^*)$
    \If{$f_r(x_r^*) \geq 0$} \Return $(\bot_l, x_r^*)$ \algorithmicelse\ \Return $(\top_l, x_r^*)$ \EndIf
  \EndProcedure
  \end{algorithmic}
\end{algorithm}
\begin{lemma}
  \cref{monDecompAlg} gives a correct solution to $\trskst$.
\end{lemma}
\begin{sproof}
The proof that all intermediate functions restrict to the transient bounds $(\bot_l, \top_l)$ is similar
to the proof of correctness of \cref{fixDecompAlg} with a few extra cases. 
  The crux of the proof for \cref{monDecompAlg} 
is \citep[Lemma 6]{chenLi}
where it is shown that at line 16, for each $i \in \{a + 1, ..., b + 1\}$, and every $p, p' \in [\bot_l, \top_l]$,
  $f((p, x_r^*)))_i = f((p', x_r^*))_i$. This is because inductively from the loop on line 6 I find
$\bot_l \geq \bot_l^j$ and $\top_l \leq \top_l^j$,
  so if $f((\bot_l, x_r^*))_i = 1$, then since $(\bot_l^j, \top_l^j)$ was a solution to
  $\textsc{RefinedTarski}^*(f_l^j, A)$ I find $f((p, x_r^*)) \geq f^j((\bot_l, x_r^*))_j \geq f_l^j(\bot_l^j)_j = 1$.
  The cases with $f((\bot_l, x_r^*))_i \in \{-1, 0 \}$ are similar. \\
  Then when the algorithm returns on line 17, $(\bot_l, \top_l)$ was a solution to
  $\textsc{RefinedTarski}^*$ in the slice at $x_r^*$, so for each $i \in [a]$, 
  $f((\bot_l, x_r^*))_i \geq 0$ and $f((\top_l, x_r^*))_i \leq 0$, so the condition on line 17
  ensures that the returned point is a solution to $\trskst$.
\end{sproof}
For complete proof of correctness of \cref{monDecompAlg} see \citep{chenLi}. It is clear that
\cref{monDecompAlg} gives the asymptotic bound in \cref{monDecomp}.
