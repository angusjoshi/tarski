\section{The Inner Algorithm}
I will not detail the inner algorithm in it's entirety, nor prove it's correctness;
the reader is instead directed to \citep{fasterTarski}. Instead I will introduce
the main invariant, showing how the algorithm can make progress in the important 
cases. I require a weakening of the invariant used in \cref{dQiYiAlg}; whereas
in \cref{dQiYiAlg} two opposing monotone points are maintained essentially
defining the top and bottom of a sublattice on which the monotone function
naturally restricts (and hence contains a fixpoint), 
the inner algorithm maintains an invariant so that only a monotone point is guaranteed
within the current search space.
\begin{definition}[Witness, \citep{fasterTarski}]
  Let $f : [N]^3 \to [N]^3$. A \emph{down set witness} is a pair of points
  $(d, b) \in ([N]^3)^2$ such that $d_3 = b_3$ and for some $i, j \in \{1, 2\}$
  with $i \neq j$,
  \begin{itemize}
    \item $d_i = b_i$, $b_j \leq d_j$, $f(b)_j \geq b_j$, and if $b \neq d$ then $f(d)_j \leq d_j$,
    \item $f(d)_3 \geq d_3$ and $f(b)_3 \geq b_3$.
  \end{itemize}
  An \emph{up set witness} is a pair of points
  $(a, u) \in ([N]^3)^2$ such that $a_3 = u_3$ and for some $i, j \in \{1, 2\}$
  with $i \neq j$,
  \begin{itemize}
    \item $a_i = u_i$, $a_j \leq u_j$, $f(a)_j \geq a_j$, and if $u \neq a$ then $f(u)_j \leq u_j$,
    \item $f(d)_3 \geq d_3$ and $f(b)_3 \geq b_3$.
  \end{itemize}
\end{definition}
I use a notational convenience from \citep{fasterTarski}.
\begin{notation}
  Let $f : L \to L$ be a monotone function. Then Up$(f) = \{x \in L : x \leq f(x)\}$
  and Down$(f) = \{x \in L : x \geq f(x)\}$. At times an abuse of notation is used,
  where elements of the lattice $x \in [N]^d$ are said to be in the up set or down
  set of the slice $f_s$ of a monotone function $f : [N]^d \to [N]^d$. The meaning
  is assumed to be clear from context.
\end{notation}
[add ref to figure with witness diagrams] shows the different possible configurations
of witnesses to aid the reader in digesting this definition.
Now for the main invariant of the inner algorithm.
\begin{definition}[Inner algorithm invariant]
  Let $f : [N]^3 \to [N]^3$ be a monotone function. The \emph{inner algorithm invariant}
  is an up set witness $(d, b)$ and a down set witness $(a, u)$ such that $u \leq d$.
\end{definition}
\begin{lemma}
  Let $f : [N]^3 \to [N]^3$ be a monotone function with $(a, u)$, $(d, b)$ up
  and down set witnesses respectively satisfying the inner algorithm invariant.
  Then there is a point $x \in [N]^3$ with $a \leq x \leq b$ and $x \in \Up(f)$ or $x \in \Down(f)$.
\end{lemma}
\begin{proof}
  I will prove the case that $d_1 = b_1$ and $a_1 = u_1$, and the others follow similarly.
  Let $f_s$ be the slice of $f$ at coordinate $3$ with value $b_3 = a_3 = d_3 = u_3$ (where all of
  these equalities follow from the witness definitions). If $a = u$ and $b = d$ then $a \in \Up(f_s)$ and
  $b \in \Down(f_s)$, so \cref{restricts} guarantees fixpoint of $f_s$ in the range $[a, b]$, which is then
  necessarily a monotone point. \\
  If $a \neq u$ then \cref{restricts} gives a point $c \in [N]^3$ with $a \leq c \leq u$ and $f(c)_2 = c_2$.
  By monotonicity and $c \leq u$ I have $f(c)_3 \leq c_3$. If $f(c)_1 \leq c_1$ then $c \in \Down(f)$ and I'm done.
  If $f(c)_1 \geq c_1$ then I consider $b$ and $d$. Using a similar argument there is a $v \in [N]^3$ with $d \leq v \leq b$
  such that $f(v)_3 \geq v_3$, and $f(v)_2 = v_2$. If $f(v)_1 \geq v_1$ then $v \in \Up(f)$, and if $f(v)_1 \leq v_1$
  then \cref{restricts} guarantees a fixpoint of $f_s$ in the range $[c, v]$, which is neccessarily monotone.
\end{proof}
The key is then to half the search instance in a constant amount of time by finding
a new set of points satisfying the inner algorithm invariant. There are many distinct cases
to be handled here and I will not cover them all. In particular, special cases are required for
instances which have are narrow (that is, for some $i \in \{1, 2, 3\}$ $b_i - a_i \leq 1$). The 
reader is directed to \citep{fasterTarski} for these. The first case is under the assumption that $u$ and $d$ are
not past the midpoint of the line they are on.
Throughout I set $\mi_i = \left \lfloor \frac{a_i + b_i}{2} \right\rfloor$ for each $i \in \{1, 2, 3\}$, and take
$f_s$ to be the slice of $f$ at the $3$rd coordinate with value $a_3 = b_3$.
\begin{lemma}[\citep{fasterTarski}]\label{innerMainCase}
  Suppose for each $i \in \{1, 2\}$ that $u_i \leq \left \lfloor \frac{a_i + b_i}{2} \right \rfloor$
  and $d_i \geq \left \lfloor \frac{a_i + b_i}{2} \right\rfloor$.
  Then a pair of witnesses $(\overline{a}, \overline{u})$, $(\overline{d}, \overline{b})$ satisfying the invariant can be found
  such that for some $j \in \{1, 2\}$ $\overline{b}_j - \overline{a}_j \leq \left\lceil \frac{a_j - b_j}{2}\right \rceil$ using a constant
  number of queries. 
\end{lemma}
\begin{sproof}
  Evaluate $f(\mi)$. If $\mi \in \Up(f)$ or
  $\mi \in \Down(f)$ then the inner algorithm can return immediately. If $\mi \in \Up(f_s)$ and $\mi \not\in \Up(f)$
  then $f(\mi)_3 \leq \mi$ and I can set
  $\overline{a} = \overline{u} = \mi$. By assumption $u \geq \mi$ so the pair $(\overline{a}, \overline{u})$, $(d, b)$ will do.
  The case $\mi \in \Down(f_s)$ is similar.
  Suppose for some $i, j \in \{1, 2\}$ with $i \neq j$ that $f(\mi)_i \leq \mi_i$ and $f(\mi)_j \geq \mi_j$.
  If $f(\mi)_3 \geq \mi_3$ then put $p_j = b_j$, $p_i = \mi_i$ and $p_3 = \mi_3$ and evaluate $f(p)$.  
  By monotonicity $f(p)_3 \geq p_3$, so if $f(p)_j \leq p_j$ then put $\overline{u} = \mi$ and $\overline{a} = p$ and I'm done.
  If $f(p)_j > p_j$ then by monotonicity and $b_j = p_j$ I have $f(b)_j > b_j$. It follows by definition of witnesses that if $b \neq d$ then 
  $b_i \neq d_i$ and $b_j = d_j$. Then again by monotonicity and $d \leq p$ I have $d_j \leq f(d)_j$, and by definition
  of a witness $d_i \leq f(d)_i$ and $d_3 \leq f(d)_3$, so $d \in \Up(f)$ and can be returned immediately. \\
  The case $f(\mi)_3 \leq \mi_3$ is similar and ommitted for brevity.
\end{sproof}
[insert ref to a figure here] should be examined to help understand what is happening in the proof sketch of \cref{innerMainCase}.
I must now justify the assumption in \cref{innerMainCase} that 
$u_i \leq \mi$ and $d_i \geq \mi$.
\begin{lemma}[\citep{fasterTarski}]\label{innerOtherCase}
  Suppose for some $i \in \{1, 2\}$ that $u_i \not\leq \mi_i$.
  Then a point $p \in [N]^3$ can be found such that $p_i \leq \mi_i$ and $(a, p)$ is a valid down-set
  witness, or $(p, u)$ is a valid down-set witness. 
  If $d_i \not\geq \mi$ then a point $q \in [N]^3$ can be found such that $q_i \geq \mi$
  and $(q, b)$ is a valid up-set
  witness, or $(d, q)$ is a valid up-set witness. 
\end{lemma}
\begin{sproof}
  Suppose for some $i \in \{1, 2\}$ I have $u_i > \mi_i$. Then
  by definition of a witness, if $j \in \{1, 2\}$ and $i \neq j$ then $u_j = a_j$.
  Put $p_j = u_j$, $p_i = \mi_i$. Then since $u \geq p$ by monotonicity
  and definition of a witness I have $f(u)_3 \geq u_3$. If $f(p)_i \leq p_i$ then
  $(a, p)$ is a valid down set witness that satisfies $p_i \leq \mi_i$. If
  $f(p)_i \geq p_i$ then $(p, u)$ is a valid down-set witness.
  The case with $d_i < \mi_i$ is similar.
\end{sproof}
[ref to a figure] shows visually how the proof of \cref{innerOtherCase} works. Combining
the previous two lemmas, a 'normal' iteration of the inner algorithm is as follows.
Check if $u < \mi$ or $d > \mi$. If so, using \cref{innerOtherCase} either find a new
set of witnesses with a search space that is half the size of the previous, or a new of
witnesses such that $u \leq \mi$ and $d \geq \mi$. Then continue to the next iteration.
If $u \geq \mi$ and $d \leq \mi$ then using \cref{innerMainCase} find a new set of witnesses
such that the new search space is at most half the size of the previous.
With the additional fact from \citep{fasterTarski} that the search space can be halfed in a constant amount of
queries when the search space as a width of at most 1, this gives an $O(\log N)$ query algorithm
for the $\trsks(N, 3)$ problem.
