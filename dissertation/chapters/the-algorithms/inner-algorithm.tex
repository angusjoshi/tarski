\section{The Inner Algorithm} \label{innerAlgChap}
In the interest of saving space, I will not detail the inner algorithm in it's entirety.
Instead I will introduce
the main invariant, proving lemmas showing how the inner algorithm
can make progress in the main cases.\footnote{The key missing cases are when the current search space is 'narrow',
in that the width in some dimension is $\leq 1$.
The reader is directed to \citep{fasterTarski} for these.} In contrast to
\cref{dQiYiAlg} where two opposing monotone points are maintained essentially
defining the top and bottom of a sublattice on which the monotone function
naturally restricts (and hence contains a fixpoint), 
the inner algorithm maintains an invariant so that only a monotone point is guaranteed
within the current search space.
\begin{definition}[Witness, \citep{fasterTarski}] \label{witnessDef}
  Let $f : [N]^3 \to [N]^3$. A \emph{down set witness} is a pair of points
  $(d, b) \in ([N]^3)^2$ such that $d_3 = b_3$ and for some $i, j \in \{1, 2\}$
  with $i \neq j$,
  \begin{itemize}
    \item $d_i = b_i$, $b_j \leq d_j$, $f(b)_j \geq b_j$, and if $b \neq d$ then $f(d)_j \leq d_j$,
    \item $f(d)_3 \geq d_3$ and $f(b)_3 \geq b_3$.
  \end{itemize}
  An \emph{up set witness} is a pair of points
  $(a, u) \in ([N]^3)^2$ such that $a_3 = u_3$ and for some $i, j \in \{1, 2\}$
  with $i \neq j$,
  \begin{itemize}
    \item $a_i = u_i$, $a_j \leq u_j$, $f(a)_j \geq a_j$, and if $u \neq a$ then $f(u)_j \leq u_j$,
    \item $f(d)_3 \geq d_3$ and $f(b)_3 \geq b_3$.
  \end{itemize}
\end{definition}

I use a notational convenience from \citep{fasterTarski}.
\begin{definition}[Up/Down Set]
  Let $f : L \to L$ be a monotone function. Then the \emph{up-set} of $f$ is defined $\Up(f) = \{x \in L : x \leq f(x)\}$
  and the \emph{down-set} of $f$ is defined $\Down(f) = \{x \in L : x \geq f(x)\}$. At times an abuse of notation is used,
  where elements of the lattice $x \in [N]^d$ are said to be in the up set or down
  set of the slice $f_s$ of a monotone function $f : [N]^d \to [N]^d$. The meaning
  is assumed to be clear from context.
\end{definition}
Now for the main invariant of the inner algorithm.
\begin{definition}[Inner algorithm invariant]
  Let $f : [N]^3 \to [N]^3$ be a monotone function. The \emph{inner algorithm invariant}
  is an up set witness $(d, b)$ and a down set witness $(a, u)$ such that $u \leq d$.
\end{definition}
The reader is directed to \cref{witnessFig} to aid in digesting the witness and invariant
definitions. Throughout whenever the inner algorithm invariant is satisfied,  
I fix $f_s$ be the slice of $f$ at coordinate $3$ with value $b_3 = a_3 = d_3 = u_3$ (where all of
these equalities follow from the witness definitions).
\vspace{-10pt}
\begin{figure}[ht] 
  \centering
  \scalebox{0.4}{%%%%%%%%%%%%%%%%%%%%%%%%%%%%%%%%%%%%%%%%%%%%%%%%%%%%%%%%%%%
% WITNESSES (4 examples out of 9)
%%%%%%%%%%%%%%%%%%%%%%%%%%%%%%%%%%%%%%%%%%%%%%%%%%%%%%%%%%%

%%%%%%%%%%%%%%%%%%%%%%%%%%%%%%%%%%%%%%%%%%%%%%%%%%%%%%%%%%%%%%%%%%%%%%%%%%%%%%%
% ONE -- 02_both_points.tex
%%%%%%%%%%%%%%%%%%%%%%%%%%%%%%%%%%%%%%%%%%%%%%%%%%%%%%%%%%%%%%%%%%%%%%%%%%%%%%%
\newsavebox\one
\begin{lrbox}{\one}
\begin{tikzpicture}[scale=\myscale]
\tikzset{every node}=[font=\Huge]
%%%%%%%%%%%%%%%%%%%%%%%%%%%%%%%%%%%%%%%%%%%%%%%%%%%%%%%%%%%%%%%%%%%%%%%%%%%%%%%
% MAIN SQUARE
\draw (0, 0) rectangle (8, 8);
%%%%%%%%%%%%%%%%%%%%%%%%%%%%%%%%%%%%%%%%%%%%%%%%%%%%%%%%%%%%%%%%%%%%%%%%%%%%%%%
%%%%%%%%%%%%%%%%%%%%%%%%%%%%%%%%%%%%%%%%%%%%%%%%%%%%%%%%%%%%%%%%%%%%%%%%%%%%%%%
\mydnarrow{8}{8}
\myleftarrow{8}{8}
%%%%%%%%%%%%%%%%%%%%%%%%%%%%%%%%%%%%%%%%%%%%%%%%%%%%%%%%%%%%%%%%%%%%%%%%%%%%%%%
\myuparrow{0}{0}
\myrightarrow{0}{0}
%%%%%%%%%%%%%%%%%%%%%%%%%%%%%%%%%%%%%%%%%%%%%%%%%%%%%%%%%%%%%%%%%%%%%%%%%%%%%%%
  \node[query, label=left:{$a = u$}] at (0,0) {};
  \node[query, label=right:{$d = b$}] at (8,8) {};
\end{tikzpicture}
\end{lrbox}

%%%%%%%%%%%%%%%%%%%%%%%%%%%%%%%%%%%%%%%%%%%%%%%%%%%%%%%%%%%%%%%%%%%%%%%%%%%%%%%
% TWO -- 02_opposite_ranges.tex
%%%%%%%%%%%%%%%%%%%%%%%%%%%%%%%%%%%%%%%%%%%%%%%%%%%%%%%%%%%%%%%%%%%%%%%%%%%%%%%
\newsavebox\two
\begin{lrbox}{\two}
\begin{tikzpicture}[scale=\myscale]
\tikzset{every node}=[font=\Huge]
%%%%%%%%%%%%%%%%%%%%%%%%%%%%%%%%%%%%%%%%%%%%%%%%%%%%%%%%%%%%%%%%%%%%%%%%%%%%%%%
% MAIN SQUARE
\draw (0, 0) rectangle (8, 8);
%%%%%%%%%%%%%%%%%%%%%%%%%%%%%%%%%%%%%%%%%%%%%%%%%%%%%%%%%%%%%%%%%%%%%%%%%%%%%%%
%%%%%%%%%%%%%%%%%%%%%%%%%%%%%%%%%%%%%%%%%%%%%%%%%%%%%%%%%%%%%%%%%%%%%%%%%%%%%%%
% BOTTOM
%%%%%%%%%%%%%%%%%%%%%%%%%%%%%%%%%%%%%%%%%%%%%%%%%%%%%%%%%%%%%%%%%%%%%%%%%%%%%%%
\forwardarrow{0}{0}{darkpastelgreen}
\forwardarrow{5}{0}{darkpastelgreen}
%%%%%%%%%%%%%%%%%%%%%%%%%%%%%%%%%%%%%%%%%%%%%%%%%%%%%%%%%%%%%%%%%%%%%%%%%%%%%%%
\myrightarrow{0}{0}
\myleftarrow{5}{0}
%%%%%%%%%%%%%%%%%%%%%%%%%%%%%%%%%%%%%%%%%%%%%%%%%%%%%%%%%%%%%%%%%%%%%%%%%%%%%%%
\node[query,label={[yshift=2mm] left:$a$}] (a) at (0,0) {};
\node[query,label=above right:$u$] (u) at (5,0) {};
%%%%%%%%%%%%%%%%%%%%%%%%%%%%%%%%%%%%%%%%%%%%%%%%%%%%%%%%%%%%%%%%%%%%%%%%%%%%%%%
%%%%%%%%%%%%%%%%%%%%%%%%%%%%%%%%%%%%%%%%%%%%%%%%%%%%%%%%%%%%%%%%%%%%%%%%%%%%%%%
% TOP
%%%%%%%%%%%%%%%%%%%%%%%%%%%%%%%%%%%%%%%%%%%%%%%%%%%%%%%%%%%%%%%%%%%%%%%%%%%%%%%
\backarrow{8}{8}{darkpastelgreen}
\backarrow{5}{8}{darkpastelgreen}
%%%%%%%%%%%%%%%%%%%%%%%%%%%%%%%%%%%%%%%%%%%%%%%%%%%%%%%%%%%%%%%%%%%%%%%%%%%%%%%
\myleftarrow{8}{8}
\myrightarrow{5}{8}
%%%%%%%%%%%%%%%%%%%%%%%%%%%%%%%%%%%%%%%%%%%%%%%%%%%%%%%%%%%%%%%%%%%%%%%%%%%%%%%
\node[query,label={[yshift=-2mm] right:$b$}] (c) at (8,8) {};
\node[query,label=below right:$d$] (d) at (5,8) {};
  \draw [dashed](u) -- (d);
%%%%%%%%%%%%%%%%%%%%%%%%%%%%%%%%%%%%%%%%%%%%%%%%%%%%%%%%%%%%%%%%%%%%%%%%%%%%%%%
\end{tikzpicture}
\end{lrbox}

%%%%%%%%%%%%%%%%%%%%%%%%%%%%%%%%%%%%%%%%%%%%%%%%%%%%%%%%%%%%%%%%%%%%%%%%%%%%%%%
% THREE -- 02_mixed_ranges.tex
%%%%%%%%%%%%%%%%%%%%%%%%%%%%%%%%%%%%%%%%%%%%%%%%%%%%%%%%%%%%%%%%%%%%%%%%%%%%%%%
\newsavebox\three
\begin{lrbox}{\three}
\begin{tikzpicture}[scale=\myscale]
\tikzset{every node}=[font=\Huge]
%%%%%%%%%%%%%%%%%%%%%%%%%%%%%%%%%%%%%%%%%%%%%%%%%%%%%%%%%%%%%%%%%%%%%%%%%%%%%%%
% MAIN SQUARE
\draw (0, 0) rectangle (8, 8);
%%%%%%%%%%%%%%%%%%%%%%%%%%%%%%%%%%%%%%%%%%%%%%%%%%%%%%%%%%%%%%%%%%%%%%%%%%%%%%%
%%%%%%%%%%%%%%%%%%%%%%%%%%%%%%%%%%%%%%%%%%%%%%%%%%%%%%%%%%%%%%%%%%%%%%%%%%%%%%%
% BOTTOM
%%%%%%%%%%%%%%%%%%%%%%%%%%%%%%%%%%%%%%%%%%%%%%%%%%%%%%%%%%%%%%%%%%%%%%%%%%%%%%%
\forwardarrow{0}{0}{darkpastelgreen}
\forwardarrow{0}{5}{darkpastelgreen}
%%%%%%%%%%%%%%%%%%%%%%%%%%%%%%%%%%%%%%%%%%%%%%%%%%%%%%%%%%%%%%%%%%%%%%%%%%%%%%%
\myuparrow{0}{0}
\mydnarrow{0}{5}
%%%%%%%%%%%%%%%%%%%%%%%%%%%%%%%%%%%%%%%%%%%%%%%%%%%%%%%%%%%%%%%%%%%%%%%%%%%%%%%
\node[query,label={[yshift=2mm] left:$a$}] (a) at (0,0) {};
\node[query,label=right:$u$] (a) at (0,5) {};
%%%%%%%%%%%%%%%%%%%%%%%%%%%%%%%%%%%%%%%%%%%%%%%%%%%%%%%%%%%%%%%%%%%%%%%%%%%%%%%
%%%%%%%%%%%%%%%%%%%%%%%%%%%%%%%%%%%%%%%%%%%%%%%%%%%%%%%%%%%%%%%%%%%%%%%%%%%%%%%
% TOP
%%%%%%%%%%%%%%%%%%%%%%%%%%%%%%%%%%%%%%%%%%%%%%%%%%%%%%%%%%%%%%%%%%%%%%%%%%%%%%%
\backarrow{8}{8}{darkpastelgreen}
\backarrow{5}{8}{darkpastelgreen}
%%%%%%%%%%%%%%%%%%%%%%%%%%%%%%%%%%%%%%%%%%%%%%%%%%%%%%%%%%%%%%%%%%%%%%%%%%%%%%%
\myleftarrow{8}{8}
\myrightarrow{5}{8}
%%%%%%%%%%%%%%%%%%%%%%%%%%%%%%%%%%%%%%%%%%%%%%%%%%%%%%%%%%%%%%%%%%%%%%%%%%%%%%%
\node[query,label={[yshift=-2mm] right:$b$}] (c) at (8,8) {};
\node[query,label=below:$d$] (d) at (5,8) {};
%%%%%%%%%%%%%%%%%%%%%%%%%%%%%%%%%%%%%%%%%%%%%%%%%%%%%%%%%%%%%%%%%%%%%%%%%%%%%%%
\end{tikzpicture}
\end{lrbox}

%%%%%%%%%%%%%%%%%%%%%%%%%%%%%%%%%%%%%%%%%%%%%%%%%%%%%%%%%%%%%%%%%%%%%%%%%%%%%%%
% FOUR -- 02_point_and_range.tex
%%%%%%%%%%%%%%%%%%%%%%%%%%%%%%%%%%%%%%%%%%%%%%%%%%%%%%%%%%%%%%%%%%%%%%%%%%%%%%%
\newsavebox\four
\begin{lrbox}{\four}
\begin{tikzpicture}[scale=\myscale]
\tikzset{every node}=[font=\Huge]
%%%%%%%%%%%%%%%%%%%%%%%%%%%%%%%%%%%%%%%%%%%%%%%%%%%%%%%%%%%%%%%%%%%%%%%%%%%%%%%
% MAIN SQUARE
\draw (0, 0) rectangle (8, 8);
%%%%%%%%%%%%%%%%%%%%%%%%%%%%%%%%%%%%%%%%%%%%%%%%%%%%%%%%%%%%%%%%%%%%%%%%%%%%%%%
\myuparrow{0}{0}
\myrightarrow{0}{0}
  \node[query, label=left:{$a = u$}] at (0,0) {};
%%%%%%%%%%%%%%%%%%%%%%%%%%%%%%%%%%%%%%%%%%%%%%%%%%%%%%%%%%%%%%%%%%%%%%%%%%%%%%%
% TOP
%%%%%%%%%%%%%%%%%%%%%%%%%%%%%%%%%%%%%%%%%%%%%%%%%%%%%%%%%%%%%%%%%%%%%%%%%%%%%%%
\backarrow{8}{8}{darkpastelgreen}
\backarrow{8}{5}{darkpastelgreen}
%%%%%%%%%%%%%%%%%%%%%%%%%%%%%%%%%%%%%%%%%%%%%%%%%%%%%%%%%%%%%%%%%%%%%%%%%%%%%%%
\mydnarrow{8}{8}
\myuparrow{8}{5}
%%%%%%%%%%%%%%%%%%%%%%%%%%%%%%%%%%%%%%%%%%%%%%%%%%%%%%%%%%%%%%%%%%%%%%%%%%%%%%%
\node[query,label={[yshift=-2mm] right:$b$}] (c) at (8,8) {};
\node[query,label=left:$d$] (d) at (8,5) {};
%%%%%%%%%%%%%%%%%%%%%%%%%%%%%%%%%%%%%%%%%%%%%%%%%%%%%%%%%%%%%%%%%%%%%%%%%%%%%%%
\end{tikzpicture}
\end{lrbox}

%%%%%%%%%%%%%%%%%%%%%%%%%%%%%%%%%%%%%%%%%%%%%%%%%%%%%%%%%%%%%%%%%%%%%%%%%%%%%%%
%%%%%%%%%%%%%%%%%%%%%%%%%%%%%%%%%%%%%%%%%%%%%%%%%%%%%%%%%%%%%%%%%%%%%%%%%%%%%%%
%%%%%%%%%%%%%%%%%%%%%%%%%%%%%%%%%%%%%%%%%%%%%%%%%%%%%%%%%%%%%%%%%%%%%%%%%%%%%%%
%%%%%%%%%%%%%%%%%%%%%%%%%%%%%%%%%%%%%%%%%%%%%%%%%%%%%%%%%%%%%%%%%%%%%%%%%%%%%%%
%%%%%%%%%%%%%%%%%%%%%%%%%%%%%%%%%%%%%%%%%%%%%%%%%%%%%%%%%%%%%%%%%%%%%%%%%%%%%%%

%%%%%%%%%%%%%%%%%%%%%%%%%%%%%%%%%%%%%%%%%%%%%%%%%%%%%%%%%%%%%%%%%%%%%%%%%%%%%%%
% THE OVERALL GRID
%%%%%%%%%%%%%%%%%%%%%%%%%%%%%%%%%%%%%%%%%%%%%%%%%%%%%%%%%%%%%%%%%%%%%%%%%%%%%%%
\begin{tikzpicture}

\subfig{0}{0}{}{\one}
\subfig{9}{0}{}{\two}
\subfig{18}{0}{}{\three}
\subfig{27}{0}{}{\four}

\end{tikzpicture}
}
  \caption{There are many possible configurations of witnesses that satisfy the invariant.
  All of the above are representations of witness pairs which satisfy the invariant using the useful diagramming
  notation from \citep{fasterTarski} that should be read as follows; the outer square
  represents some 'slice' of the three dimensional cube in the third coordinate. An arrow
  extending from a point $x \in [N]^3$ in direction $i \in \{1, 2, 3\}$ in the positive (negative) direction
  should be read as $f(x)_i \geq (\leq) x_i$. For example in the rightmost square if 1 is the horizontal
  direction and 2 is the vertical direction, and 3 is forwards dimension then $f(b)_2 \leq b_2$ and $f(b)_3 \geq b_3$. 
  Diagram source: \citep{fasterTarski}} \label{witnessFig}
\end{figure}
\vspace{-10pt}
\begin{lemma}
  Let $f : [N]^3 \to [N]^3$ be a monotone function with $(a, u)$, $(d, b)$ up
  and down set witnesses respectively satisfying the inner algorithm invariant.
  Then there is a point $x \in [N]^3$ with $a \leq x \leq b$ and $x \in \Up(f)$ or $x \in \Down(f)$.
\end{lemma}
\begin{proof}
   If $a = u$ and $b = d$ then by the witness definition $a \in \Up(f_s)$ and
  $b \in \Down(f_s)$, so \cref{restricts} guarantees fixpoint of $f_s$ in the range $[a, b]$, which is then
  necessarily a monotone point of $f$. \\
  By the witnesses definition, there are $j, j' \in \{1, 2\}$ such that $d_j = b_j$ and $a_{j'} = u_{j'}$.
  Then let $i, i' \in \{1, 2\}$ with $i \neq j$ and $i' \neq j'$.
  If $a \neq u$ then \cref{restricts} gives a point $c \in [N]^3$ with $a \leq c \leq u$ and $f(c)_{i'} = c_{i'}$.
  By monotonicity and $c \leq u$ I have $f(c)_3 \leq c_3$. If $f(c)_1 \leq c_{j'}$ then $c \in \Down(f)$ and I'm done.
  If $f(c)_{j'} \geq c_{j'}$ then I consider $b$ and $d$. Using a similar argument there is a $v \in [N]^3$ with $d \leq v \leq b$
  such that $f(v)_3 \geq v_3$, and $f(v)_i = v_i$. If $f(v)_j \geq v_j$ then $v \in \Up(f)$, and if $f(v)_j \leq v_j$
  then by the invariant, $v \geq d \geq u \geq c$ gives $v \geq c$, 
  and \cref{restricts} guarantees a fixpoint of $f_s$ in the range $[c, v]$, which is neccessarily a monotone
  point of $f$.
\end{proof}
The key is then to half the current search space in a constant amount of time by finding
a new set of points satisfying the inner algorithm invariant. There are many distinct cases
to be handled here and I have not covered them all. In particular, special cases are required for
instances which are narrow (that is, for some $i \in \{1, 2 \}$ $b_i - a_i \leq 1$). The 
reader is directed to \citep{fasterTarski} for these. The first case is under the assumption that $u$ and $d$ are
not past the midpoint of the line they are on.
Throughout I set $\mi_i = \left \lfloor \frac{a_i + b_i}{2} \right\rfloor$ for each $i \in \{1, 2, 3\}$, and take
$f_s$ to be the slice of $f$ at the $3$rd coordinate with value $a_3 = b_3$.
\begin{lemma}[\citep{fasterTarski}]\label{innerMainCase}
  Let $f : [N]^3 \to [N]^3$ and
  suppose for each $i \in \{1, 2\}$ that $u_i \leq \left \lfloor \frac{a_i + b_i}{2} \right \rfloor$
  and $d_i \geq \left \lfloor \frac{a_i + b_i}{2} \right\rfloor$.
  Then either a pair of witnesses $(\overline{a}, \overline{u})$, $(\overline{d}, \overline{b})$ satisfying the invariant
  such that for some $j \in \{1, 2\}$ $\overline{b}_j - \overline{a}_j \leq \left\lceil \frac{a_j - b_j}{2}\right \rceil$, 
  or a point $x \in \Up(f) \cup \Down(f)$
  can be found using a constant
  number of queries. 
\end{lemma}
\begin{proof} 
  Evaluate $f(\mi)$. If $\mi \in \Up(f)$ or
  $\mi \in \Down(f)$ then the inner algorithm can return immediately. If $\mi \in \Up(f_s)$ and $\mi \not\in \Up(f)$
  then $f(\mi)_3 \leq \mi$ and I can set
  $\overline{a} = \overline{u} = \mi$. By assumption $u \geq \mi \geq d$ so the pair $(\overline{a}, \overline{u})$, $(d, b)$ will do.
  The case $\mi \in \Down(f_s)$ is similar.
  Suppose for some $i, j \in \{1, 2\}$ with $i \neq j$ that $f(\mi)_i \leq \mi_i$ and $f(\mi)_j \geq \mi_j$.
  If $f(\mi)_3 \geq \mi_3$ then put $p_j = b_j$, $p_i = \mi_i$ and $p_3 = \mi_3$ and evaluate $f(p)$.  
  By monotonicity $f(p)_3 \geq p_3$, so if $f(p)_j \leq p_j$ then put $\overline{u} = \mi$ and $\overline{a} = p$ and I'm done.
  If $f(p)_j > p_j$ then by monotonicity and $b_j = p_j$ I have $f(b)_j > b_j$. It follows by definition of witnesses that if $b \neq d$ then 
  $b_i \neq d_i$ and $b_j = d_j$. Then again by monotonicity and $d \leq p$ I have $d_j \leq f(d)_j$, and by definition
  of a witness $d_i \leq f(d)_i$ and $d_3 \leq f(d)_3$, so $d \in \Up(f)$ and can be returned immediately. \\
  The case $f(\mi)_3 \leq \mi_3$ is similar except I consider $p_i = a_i$, $p_j = \mi_j$, $p_3 = \mi_3$ and
  find either $(p, \mi)$ is an up set witness, or $u \in \Down(f)$.
\end{proof}
\vspace{-10pt}
\begin{figure}[ht]
  \centering
  \scalebox{0.4}{%%%%%%%%%%%%%%%%%%%%%%%%%%%%%%%%%%%%%%%%%%%%%%%%%%%%%%%%%%%%%%%%%%%%%%%%%%%%%%%
% ROW 1 COL 1/2: 04_up_set.tex
%%%%%%%%%%%%%%%%%%%%%%%%%%%%%%%%%%%%%%%%%%%%%%%%%%%%%%%%%%%%%%%%%%%%%%%%%%%%%%%
\newsavebox\upset
\begin{lrbox}{\upset}
\begin{tikzpicture}[scale=\myscale]
\tikzset{every node}=[font=\Huge]
%%%%%%%%%%%%%%%%%%%%%%%%%%%%%%%%%%%%%%%%%%%%%%%%%%%%%%%%%%%%%%%%%%%%%%%%%%%%%%%
% Phantom arrows for equal spacing of subfigs
\backarrow{10}{10}{white}
\forwardarrow{0}{0}{white}
%%%%%%%%%%%%%%%%%%%%%%%%%%%%%%%%%%%%%%%%%%%%%%%%%%%%%%%%%%%%%%%%%%%%%%%%%%%%%%%
% fill in lhs (early so in background)
\draw[draw=none,fill=lightgray] (0,0) rectangle (5,10);
\draw[draw=none,fill=lightgray] (0,0) rectangle (10,5);
%%%%%%%%%%%%%%%%%%%%%%%%%%%%%%%%%%%%%%%%%%%%%%%%%%%%%%%%%%%%%%%%%%%%%%%%%%%%%%%
% MAIN SQUARE
\draw (0, 0) rectangle (10, 10);
%%%%%%%%%%%%%%%%%%%%%%%%%%%%%%%%%%%%%%%%%%%%%%%%%%%%%%%%%%%%%%%%%%%%%%%%%%%%%%%
\draw[dashed] (5,5) -- (5,10);
\draw[dashed] (5,5) -- (10,5);
%%%%%%%%%%%%%%%%%%%%%%%%%%%%%%%%%%%%%%%%%%%%%%%%%%%%%%%%%%%%%%%%%%%%%%%%%%%%%%%
\myuparrow{5}{5}
\myrightarrow{5}{5}
%%%%%%%%%%%%%%%%%%%%%%%%%%%%%%%%%%%%%%%%%%%%%%%%%%%%%%%%%%%%%%%%%%%%%%%%%%%%%%%
\node[query, label=above right:\midp] at (5,5) {};
\node[query,label=above right:$a$] at (0,0) {};
\node[query,label=below left:$b$] at (10,10) {};
%%%%%%%%%%%%%%%%%%%%%%%%%%%%%%%%%%%%%%%%%%%%%%%%%%%%%%%%%%%%%%%%%%%%%%%%%%%%%%%
\end{tikzpicture}
\end{lrbox}

%%%%%%%%%%%%%%%%%%%%%%%%%%%%%%%%%%%%%%%%%%%%%%%%%%%%%%%%%%%%%%%%%%%%%%%%%%%%%%%
% ROW 1 COL 2/2: 04_dn_set.tex
%%%%%%%%%%%%%%%%%%%%%%%%%%%%%%%%%%%%%%%%%%%%%%%%%%%%%%%%%%%%%%%%%%%%%%%%%%%%%%%
\newsavebox\dnset
\begin{lrbox}{\dnset}
\begin{tikzpicture}[scale=\myscale]
\tikzset{every node}=[font=\Huge]
%%%%%%%%%%%%%%%%%%%%%%%%%%%%%%%%%%%%%%%%%%%%%%%%%%%%%%%%%%%%%%%%%%%%%%%%%%%%%%%
% Phantom arrows for equal spacing of subfigs
\backarrow{10}{10}{white}
\forwardarrow{0}{0}{white}
%%%%%%%%%%%%%%%%%%%%%%%%%%%%%%%%%%%%%%%%%%%%%%%%%%%%%%%%%%%%%%%%%%%%%%%%%%%%%%%
% fill in lhs (early so in background)
\draw[draw=none, fill=lightgray] (10,10) rectangle (0,5);
\draw[draw=none, fill=lightgray] (10,10) rectangle (5,0);
%%%%%%%%%%%%%%%%%%%%%%%%%%%%%%%%%%%%%%%%%%%%%%%%%%%%%%%%%%%%%%%%%%%%%%%%%%%%%%%
% MAIN SQUARE
\draw (0, 0) rectangle (10, 10);
%%%%%%%%%%%%%%%%%%%%%%%%%%%%%%%%%%%%%%%%%%%%%%%%%%%%%%%%%%%%%%%%%%%%%%%%%%%%%%%
\draw[dashed] (5,5) -- (0,5);
\draw[dashed] (5,5) -- (5,0);
%%%%%%%%%%%%%%%%%%%%%%%%%%%%%%%%%%%%%%%%%%%%%%%%%%%%%%%%%%%%%%%%%%%%%%%%%%%%%%%
\mydnarrow{5}{5}
\myleftarrow{5}{5}
%%%%%%%%%%%%%%%%%%%%%%%%%%%%%%%%%%%%%%%%%%%%%%%%%%%%%%%%%%%%%%%%%%%%%%%%%%%%%%%
\node[query, label=above right:\midp] at (5,5) {};
\node[query,label=above right:$a$] at (0,0) {};
\node[query,label=below left:$b$] at (10,10) {};
%%%%%%%%%%%%%%%%%%%%%%%%%%%%%%%%%%%%%%%%%%%%%%%%%%%%%%%%%%%%%%%%%%%%%%%%%%%%%%%
\end{tikzpicture}
\end{lrbox}

%%%%%%%%%%%%%%%%%%%%%%%%%%%%%%%%%%%%%%%%%%%%%%%%%%%%%%%%%%%%%%%%%%%%%%%%%%%%%%%
% ROW 2 COL 1/3: 04_dn_right_right_v1.tex
%%%%%%%%%%%%%%%%%%%%%%%%%%%%%%%%%%%%%%%%%%%%%%%%%%%%%%%%%%%%%%%%%%%%%%%%%%%%%%%
\newsavebox\rowtwocolone
\begin{lrbox}{\rowtwocolone}
\begin{tikzpicture}[scale=\myscale]
\tikzset{every node}=[font=\Huge]
%%%%%%%%%%%%%%%%%%%%%%%%%%%%%%%%%%%%%%%%%%%%%%%%%%%%%%%%%%%%%%%%%%%%%%%%%%%%%%%
% Phantom arrows for equal spacing of subfigs
\backarrow{10}{10}{white}
\forwardarrow{0}{0}{white}
%%%%%%%%%%%%%%%%%%%%%%%%%%%%%%%%%%%%%%%%%%%%%%%%%%%%%%%%%%%%%%%%%%%%%%%%%%%%%%%
% MAIN SQUARE
\draw (0, 0) rectangle (10, 10);
%%%%%%%%%%%%%%%%%%%%%%%%%%%%%%%%%%%%%%%%%%%%%%%%%%%%%%%%%%%%%%%%%%%%%%%%%%%%%%%
\backarrow{5}{5}{darkpastelgreen}
\forwardarrow{10}{5}{darkpastelgreen}
%%%%%%%%%%%%%%%%%%%%%%%%%%%%%%%%%%%%%%%%%%%%%%%%%%%%%%%%%%%%%%%%%%%%%%%%%%%%%%%
\mydnarrow{5}{5}
\myrightarrow{5}{5}
%%%%%%%%%%%%%%%%%%%%%%%%%%%%%%%%%%%%%%%%%%%%%%%%%%%%%%%%%%%%%%%%%%%%%%%%%%%%%%%
\node[query,label=left:\midp] at (5,5) {};
\node[query,label=above left:$p$] at (10,5) {};
\node[query,label=above right:$a$] at (0,0) {};
\node[query,label=below left:$b$] at (10,10) {};
%%%%%%%%%%%%%%%%%%%%%%%%%%%%%%%%%%%%%%%%%%%%%%%%%%%%%%%%%%%%%%%%%%%%%%%%%%%%%%%
\end{tikzpicture}
\end{lrbox}

%%%%%%%%%%%%%%%%%%%%%%%%%%%%%%%%%%%%%%%%%%%%%%%%%%%%%%%%%%%%%%%%%%%%%%%%%%%%%%%
% ROW 2 COL 2/3: 04_dn_right_right_v2.tex
%%%%%%%%%%%%%%%%%%%%%%%%%%%%%%%%%%%%%%%%%%%%%%%%%%%%%%%%%%%%%%%%%%%%%%%%%%%%%%%
\newsavebox\rowtwocoltwo
\begin{lrbox}{\rowtwocoltwo}
\begin{tikzpicture}[scale=\myscale]
\tikzset{every node}=[font=\Huge]
%%%%%%%%%%%%%%%%%%%%%%%%%%%%%%%%%%%%%%%%%%%%%%%%%%%%%%%%%%%%%%%%%%%%%%%%%%%%%%%
% Phantom arrows for equal spacing of subfigs
\backarrow{10}{10}{white}
\forwardarrow{0}{0}{white}
%%%%%%%%%%%%%%%%%%%%%%%%%%%%%%%%%%%%%%%%%%%%%%%%%%%%%%%%%%%%%%%%%%%%%%%%%%%%%%%
% MAIN SQUARE
\draw (0, 0) rectangle (10, 10);
%%%%%%%%%%%%%%%%%%%%%%%%%%%%%%%%%%%%%%%%%%%%%%%%%%%%%%%%%%%%%%%%%%%%%%%%%%%%%%%
\backarrow{5}{5}{darkpastelgreen}
\backarrow{10}{5}{darkpastelgreen}
\backarrow{10}{7}{darkpastelgreen}
%%%%%%%%%%%%%%%%%%%%%%%%%%%%%%%%%%%%%%%%%%%%%%%%%%%%%%%%%%%%%%%%%%%%%%%%%%%%%%%
\mydnarrow{5}{5}
\myrightarrow{5}{5}
\myrightarrow{10}{5}
\myrightarrow{10}{7}
\myuparrow{10}{7}
%%%%%%%%%%%%%%%%%%%%%%%%%%%%%%%%%%%%%%%%%%%%%%%%%%%%%%%%%%%%%%%%%%%%%%%%%%%%%%%
\node[query,label=above left:\midp] at (5,5) {};
\node[query,label=left:$p$] at (10,5) {};
\node[query, label=left:$d$] at (10, 7) {};
\node[query,label=above right:$a$] at (0,0) {};
\node[query,label=below left:$b$] at (10,10) {};
%%%%%%%%%%%%%%%%%%%%%%%%%%%%%%%%%%%%%%%%%%%%%%%%%%%%%%%%%%%%%%%%%%%%%%%%%%%%%%%
\end{tikzpicture}
\end{lrbox}

%%%%%%%%%%%%%%%%%%%%%%%%%%%%%%%%%%%%%%%%%%%%%%%%%%%%%%%%%%%%%%%%%%%%%%%%%%%%%%%
% ROW 2 COL 3/3: 04_dn_right_right_rule_out_top.tex
%%%%%%%%%%%%%%%%%%%%%%%%%%%%%%%%%%%%%%%%%%%%%%%%%%%%%%%%%%%%%%%%%%%%%%%%%%%%%%%
\newsavebox\rowtwocolthr
\begin{lrbox}{\rowtwocolthr}
\begin{tikzpicture}[scale=\myscale]
\tikzset{every node}=[font=\Huge]
%%%%%%%%%%%%%%%%%%%%%%%%%%%%%%%%%%%%%%%%%%%%%%%%%%%%%%%%%%%%%%%%%%%%%%%%%%%%%%%
% Phantom arrows for equal spacing of subfigs
\backarrow{10}{10}{white}
\forwardarrow{0}{0}{white}
%%%%%%%%%%%%%%%%%%%%%%%%%%%%%%%%%%%%%%%%%%%%%%%%%%%%%%%%%%%%%%%%%%%%%%%%%%%%%%%
% fill in lhs (early so in background)
\draw[white, fill=lightgray] (0,5) rectangle (10,10);
\draw[dashed] (0,5) -- (10,5);
%%%%%%%%%%%%%%%%%%%%%%%%%%%%%%%%%%%%%%%%%%%%%%%%%%%%%%%%%%%%%%%%%%%%%%%%%%%%%%%
% MAIN SQUARE
\draw (0, 0) rectangle (10, 10);
%%%%%%%%%%%%%%%%%%%%%%%%%%%%%%%%%%%%%%%%%%%%%%%%%%%%%%%%%%%%%%%%%%%%%%%%%%%%%%%
\backarrow{5}{5}{darkpastelgreen}
\backarrow{10}{5}{darkpastelgreen}
%%%%%%%%%%%%%%%%%%%%%%%%%%%%%%%%%%%%%%%%%%%%%%%%%%%%%%%%%%%%%%%%%%%%%%%%%%%%%%%
\mydnarrow{5}{5}
\myrightarrow{5}{5}
\myleftarrow{10}{5}
%%%%%%%%%%%%%%%%%%%%%%%%%%%%%%%%%%%%%%%%%%%%%%%%%%%%%%%%%%%%%%%%%%%%%%%%%%%%%%%
\node[query,label=below left:\midp] at (5,5) {};
\node[query,label=below left:$p$] at (10,5) {};
\node[query,label=above right:$a$] at (0,0) {};
\node[query,label=below left:$b$] at (10,10) {};
%%%%%%%%%%%%%%%%%%%%%%%%%%%%%%%%%%%%%%%%%%%%%%%%%%%%%%%%%%%%%%%%%%%%%%%%%%%%%%%
\end{tikzpicture}
\end{lrbox}

%%%%%%%%%%%%%%%%%%%%%%%%%%%%%%%%%%%%%%%%%%%%%%%%%%%%%%%%%%%%%%%%%%%%%%%%%%%%%%%
% ROW 3 COL 1/3: 04_dn_right_dn_v1.tex
%%%%%%%%%%%%%%%%%%%%%%%%%%%%%%%%%%%%%%%%%%%%%%%%%%%%%%%%%%%%%%%%%%%%%%%%%%%%%%%
\newsavebox\rowthrcolone
\begin{lrbox}{\rowthrcolone}
\begin{tikzpicture}[scale=\myscale]
\tikzset{every node}=[font=\Huge]
%%%%%%%%%%%%%%%%%%%%%%%%%%%%%%%%%%%%%%%%%%%%%%%%%%%%%%%%%%%%%%%%%%%%%%%%%%%%%%%
% Phantom arrows for equal spacing of subfigs
\backarrow{10}{10}{white}
\forwardarrow{0}{0}{white}
%%%%%%%%%%%%%%%%%%%%%%%%%%%%%%%%%%%%%%%%%%%%%%%%%%%%%%%%%%%%%%%%%%%%%%%%%%%%%%%
% MAIN SQUARE
\draw (0, 0) rectangle (10, 10);
%%%%%%%%%%%%%%%%%%%%%%%%%%%%%%%%%%%%%%%%%%%%%%%%%%%%%%%%%%%%%%%%%%%%%%%%%%%%%%%
\forwardarrow{5}{5}{darkpastelgreen}
\backarrow{5}{0}{darkpastelgreen}
%%%%%%%%%%%%%%%%%%%%%%%%%%%%%%%%%%%%%%%%%%%%%%%%%%%%%%%%%%%%%%%%%%%%%%%%%%%%%%%
\mydnarrow{5}{5}
\myrightarrow{5}{5}
%%%%%%%%%%%%%%%%%%%%%%%%%%%%%%%%%%%%%%%%%%%%%%%%%%%%%%%%%%%%%%%%%%%%%%%%%%%%%%%
\node[query,label=above left:\midp] at (5,5) {};
\node[query,label=above left:\botp] at (5,0) {};
\node[query,label=above right:$a$] at (0,0) {};
\node[query,label=below left:$b$] at (10,10) {};
%%%%%%%%%%%%%%%%%%%%%%%%%%%%%%%%%%%%%%%%%%%%%%%%%%%%%%%%%%%%%%%%%%%%%%%%%%%%%%%
\end{tikzpicture}
\end{lrbox}

%%%%%%%%%%%%%%%%%%%%%%%%%%%%%%%%%%%%%%%%%%%%%%%%%%%%%%%%%%%%%%%%%%%%%%%%%%%%%%%
% ROW 3 COL 2/3: 04_dn_right_dn_v2.tex
%%%%%%%%%%%%%%%%%%%%%%%%%%%%%%%%%%%%%%%%%%%%%%%%%%%%%%%%%%%%%%%%%%%%%%%%%%%%%%%
\newsavebox\rowthrcoltwo
\begin{lrbox}{\rowthrcoltwo}
\begin{tikzpicture}[scale=\myscale]
\tikzset{every node}=[font=\Huge]
%%%%%%%%%%%%%%%%%%%%%%%%%%%%%%%%%%%%%%%%%%%%%%%%%%%%%%%%%%%%%%%%%%%%%%%%%%%%%%%
% Phantom arrows for equal spacing of subfigs
\backarrow{10}{10}{white}
\forwardarrow{0}{0}{white}
%%%%%%%%%%%%%%%%%%%%%%%%%%%%%%%%%%%%%%%%%%%%%%%%%%%%%%%%%%%%%%%%%%%%%%%%%%%%%%%
% MAIN SQUARE
\draw (0, 0) rectangle (10, 10);
%%%%%%%%%%%%%%%%%%%%%%%%%%%%%%%%%%%%%%%%%%%%%%%%%%%%%%%%%%%%%%%%%%%%%%%%%%%%%%%
\forwardarrow{5}{5}{darkpastelgreen}
\forwardarrow{5}{0}{darkpastelgreen}
%%%%%%%%%%%%%%%%%%%%%%%%%%%%%%%%%%%%%%%%%%%%%%%%%%%%%%%%%%%%%%%%%%%%%%%%%%%%%%%
\mydnarrow{5}{5}
\myrightarrow{5}{5}
\mydnarrow{5}{0}
%%%%%%%%%%%%%%%%%%%%%%%%%%%%%%%%%%%%%%%%%%%%%%%%%%%%%%%%%%%%%%%%%%%%%%%%%%%%%%%
\node[query, label=above left:\midp] at (5,5) {};
\node[query, label=above:\botp] at (5,0) {};
\node[query,label=above right:$a$] at (0,0) {};
\node[query,label=below left:$b$] at (10,10) {};
%%%%%%%%%%%%%%%%%%%%%%%%%%%%%%%%%%%%%%%%%%%%%%%%%%%%%%%%%%%%%%%%%%%%%%%%%%%%%%%
\end{tikzpicture}
\end{lrbox}

%%%%%%%%%%%%%%%%%%%%%%%%%%%%%%%%%%%%%%%%%%%%%%%%%%%%%%%%%%%%%%%%%%%%%%%%%%%%%%%
% ROW 3 COL 3/3: 04_dn_right_dn_rule_out_left.tex
%%%%%%%%%%%%%%%%%%%%%%%%%%%%%%%%%%%%%%%%%%%%%%%%%%%%%%%%%%%%%%%%%%%%%%%%%%%%%%%
\newsavebox\rowthrcolthr
\begin{lrbox}{\rowthrcolthr}
\begin{tikzpicture}[scale=\myscale]
\tikzset{every node}=[font=\Huge]
%%%%%%%%%%%%%%%%%%%%%%%%%%%%%%%%%%%%%%%%%%%%%%%%%%%%%%%%%%%%%%%%%%%%%%%%%%%%%%%
% Phantom arrows for equal spacing of subfigs
\backarrow{10}{10}{white}
\forwardarrow{0}{0}{white}
%%%%%%%%%%%%%%%%%%%%%%%%%%%%%%%%%%%%%%%%%%%%%%%%%%%%%%%%%%%%%%%%%%%%%%%%%%%%%%%
% fill in lhs (early so in background)
\draw[white, fill=lightgray] (0,0) rectangle (5,10);
\draw[dashed] (5,0) -- (5,10);
%%%%%%%%%%%%%%%%%%%%%%%%%%%%%%%%%%%%%%%%%%%%%%%%%%%%%%%%%%%%%%%%%%%%%%%%%%%%%%%
% MAIN SQUARE
\draw (0, 0) rectangle (10, 10);
%%%%%%%%%%%%%%%%%%%%%%%%%%%%%%%%%%%%%%%%%%%%%%%%%%%%%%%%%%%%%%%%%%%%%%%%%%%%%%%
\forwardarrow{5}{5}{darkpastelgreen}
\forwardarrow{5}{0}{darkpastelgreen}
%%%%%%%%%%%%%%%%%%%%%%%%%%%%%%%%%%%%%%%%%%%%%%%%%%%%%%%%%%%%%%%%%%%%%%%%%%%%%%%
\mydnarrow{5}{5}
\myrightarrow{5}{5}
\myuparrow{5}{0}
%%%%%%%%%%%%%%%%%%%%%%%%%%%%%%%%%%%%%%%%%%%%%%%%%%%%%%%%%%%%%%%%%%%%%%%%%%%%%%%
\node[query, label=above left:\midp] at (5,5) {};
\node[query, label=above right:\botp] at (5,0) {};
\node[query,label=above right:$a$] at (0,0) {};
\node[query,label=below left:$b$] at (10,10) {};
%%%%%%%%%%%%%%%%%%%%%%%%%%%%%%%%%%%%%%%%%%%%%%%%%%%%%%%%%%%%%%%%%%%%%%%%%%%%%%%
\end{tikzpicture}
\end{lrbox}

%%%%%%%%%%%%%%%%%%%%%%%%%%%%%%%%%%%%%%%%%%%%%%%%%%%%%%%%%%%%%%%%%%%%%%%%%%%%%%%
% ROW 4 COL 1/3: 04_up_left_left_v1.tex
%%%%%%%%%%%%%%%%%%%%%%%%%%%%%%%%%%%%%%%%%%%%%%%%%%%%%%%%%%%%%%%%%%%%%%%%%%%%%%%
\newsavebox\rowfoucolone
\begin{lrbox}{\rowfoucolone}
\begin{tikzpicture}[scale=\myscale]
\tikzset{every node}=[font=\Huge]
%%%%%%%%%%%%%%%%%%%%%%%%%%%%%%%%%%%%%%%%%%%%%%%%%%%%%%%%%%%%%%%%%%%%%%%%%%%%%%%
% Phantom arrows for equal spacing of subfigs
\backarrow{10}{10}{white}
\forwardarrow{0}{0}{white}
%%%%%%%%%%%%%%%%%%%%%%%%%%%%%%%%%%%%%%%%%%%%%%%%%%%%%%%%%%%%%%%%%%%%%%%%%%%%%%%
% MAIN SQUARE
\draw (0, 0) rectangle (10, 10);
%%%%%%%%%%%%%%%%%%%%%%%%%%%%%%%%%%%%%%%%%%%%%%%%%%%%%%%%%%%%%%%%%%%%%%%%%%%%%%%
\backarrow{5}{5}{darkpastelgreen}
\forwardarrow{5}{10}{darkpastelgreen}
%%%%%%%%%%%%%%%%%%%%%%%%%%%%%%%%%%%%%%%%%%%%%%%%%%%%%%%%%%%%%%%%%%%%%%%%%%%%%%%
\myuparrow{5}{5}
\myleftarrow{5}{5}
%%%%%%%%%%%%%%%%%%%%%%%%%%%%%%%%%%%%%%%%%%%%%%%%%%%%%%%%%%%%%%%%%%%%%%%%%%%%%%%
\node[query,label=below:\midp] at (5,5) {};
\node[query,label=below right:\topp] at (5,10) {};
\node[query,label=above right:$a$] at (0,0) {};
\node[query,label=below left:$b$] at (10,10) {};
%%%%%%%%%%%%%%%%%%%%%%%%%%%%%%%%%%%%%%%%%%%%%%%%%%%%%%%%%%%%%%%%%%%%%%%%%%%%%%%
\end{tikzpicture}
\end{lrbox}

%%%%%%%%%%%%%%%%%%%%%%%%%%%%%%%%%%%%%%%%%%%%%%%%%%%%%%%%%%%%%%%%%%%%%%%%%%%%%%%
% ROW 4 COL 2/3: 04_up_left_left_v2.tex
%%%%%%%%%%%%%%%%%%%%%%%%%%%%%%%%%%%%%%%%%%%%%%%%%%%%%%%%%%%%%%%%%%%%%%%%%%%%%%%
\newsavebox\rowfoucoltwo
\begin{lrbox}{\rowfoucoltwo}
\begin{tikzpicture}[scale=\myscale]
\tikzset{every node}=[font=\Huge]
%%%%%%%%%%%%%%%%%%%%%%%%%%%%%%%%%%%%%%%%%%%%%%%%%%%%%%%%%%%%%%%%%%%%%%%%%%%%%%%
% Phantom arrows for equal spacing of subfigs
\backarrow{10}{10}{white}
\forwardarrow{0}{0}{white}
%%%%%%%%%%%%%%%%%%%%%%%%%%%%%%%%%%%%%%%%%%%%%%%%%%%%%%%%%%%%%%%%%%%%%%%%%%%%%%%
% MAIN SQUARE
\draw (0, 0) rectangle (10, 10);
%%%%%%%%%%%%%%%%%%%%%%%%%%%%%%%%%%%%%%%%%%%%%%%%%%%%%%%%%%%%%%%%%%%%%%%%%%%%%%%
\backarrow{5}{5}{darkpastelgreen}
\backarrow{5}{10}{darkpastelgreen}
%%%%%%%%%%%%%%%%%%%%%%%%%%%%%%%%%%%%%%%%%%%%%%%%%%%%%%%%%%%%%%%%%%%%%%%%%%%%%%%
\myuparrow{5}{5}
\myleftarrow{5}{5}
\myuparrow{5}{10}
%%%%%%%%%%%%%%%%%%%%%%%%%%%%%%%%%%%%%%%%%%%%%%%%%%%%%%%%%%%%%%%%%%%%%%%%%%%%%%%
\node[query, label=below:\midp] at (5,5) {};
\node[query, label=below:\topp] at (5,10) {};
\node[query,label=above right:$a$] at (0,0) {};
\node[query,label=below left:$b$] at (10,10) {};
%%%%%%%%%%%%%%%%%%%%%%%%%%%%%%%%%%%%%%%%%%%%%%%%%%%%%%%%%%%%%%%%%%%%%%%%%%%%%%%
\end{tikzpicture}
\end{lrbox}

%%%%%%%%%%%%%%%%%%%%%%%%%%%%%%%%%%%%%%%%%%%%%%%%%%%%%%%%%%%%%%%%%%%%%%%%%%%%%%%
% ROW 4 COL 3/3: 04_up_left_left_rule_out_right.tex
%%%%%%%%%%%%%%%%%%%%%%%%%%%%%%%%%%%%%%%%%%%%%%%%%%%%%%%%%%%%%%%%%%%%%%%%%%%%%%%
\newsavebox\rowfoucolthr
\begin{lrbox}{\rowfoucolthr}
\begin{tikzpicture}[scale=\myscale]
\tikzset{every node}=[font=\Huge]
%%%%%%%%%%%%%%%%%%%%%%%%%%%%%%%%%%%%%%%%%%%%%%%%%%%%%%%%%%%%%%%%%%%%%%%%%%%%%%%
% Phantom arrows for equal spacing of subfigs
\backarrow{10}{10}{white}
\forwardarrow{0}{0}{white}
%%%%%%%%%%%%%%%%%%%%%%%%%%%%%%%%%%%%%%%%%%%%%%%%%%%%%%%%%%%%%%%%%%%%%%%%%%%%%%%
% fill in lhs (early so in background)
\draw[white, fill=lightgray] (5,0) rectangle (10,10);
\draw[dashed] (5,0) -- (5,10);
%%%%%%%%%%%%%%%%%%%%%%%%%%%%%%%%%%%%%%%%%%%%%%%%%%%%%%%%%%%%%%%%%%%%%%%%%%%%%%%
% MAIN SQUARE
\draw (0, 0) rectangle (10, 10);
%%%%%%%%%%%%%%%%%%%%%%%%%%%%%%%%%%%%%%%%%%%%%%%%%%%%%%%%%%%%%%%%%%%%%%%%%%%%%%%
\backarrow{5}{5}{darkpastelgreen}
\backarrow{5}{10}{darkpastelgreen}
%%%%%%%%%%%%%%%%%%%%%%%%%%%%%%%%%%%%%%%%%%%%%%%%%%%%%%%%%%%%%%%%%%%%%%%%%%%%%%%
\myuparrow{5}{5}
\myleftarrow{5}{5}
\mydnarrow{5}{10}
%%%%%%%%%%%%%%%%%%%%%%%%%%%%%%%%%%%%%%%%%%%%%%%%%%%%%%%%%%%%%%%%%%%%%%%%%%%%%%%
\node[query,label=below left:\midp] at (5,5) {};
\node[query,label=below left:\topp] at (5,10) {};
\node[query,label=above right:$a$] at (0,0) {};
\node[query,label=below left:$b$] at (10,10) {};
%%%%%%%%%%%%%%%%%%%%%%%%%%%%%%%%%%%%%%%%%%%%%%%%%%%%%%%%%%%%%%%%%%%%%%%%%%%%%%%
\end{tikzpicture}
\end{lrbox}

%%%%%%%%%%%%%%%%%%%%%%%%%%%%%%%%%%%%%%%%%%%%%%%%%%%%%%%%%%%%%%%%%%%%%%%%%%%%%%%
% ROW 5 COL 1/3: 04_up_left_up_v1.tex
%%%%%%%%%%%%%%%%%%%%%%%%%%%%%%%%%%%%%%%%%%%%%%%%%%%%%%%%%%%%%%%%%%%%%%%%%%%%%%%
\newsavebox\rowfivcolone
\begin{lrbox}{\rowfivcolone}
\begin{tikzpicture}[scale=\myscale]
\tikzset{every node}=[font=\Huge]
%%%%%%%%%%%%%%%%%%%%%%%%%%%%%%%%%%%%%%%%%%%%%%%%%%%%%%%%%%%%%%%%%%%%%%%%%%%%%%%
% Phantom arrows for equal spacing of subfigs
\backarrow{10}{10}{white}
\forwardarrow{0}{0}{white}
%%%%%%%%%%%%%%%%%%%%%%%%%%%%%%%%%%%%%%%%%%%%%%%%%%%%%%%%%%%%%%%%%%%%%%%%%%%%%%%
% MAIN SQUARE
\draw (0, 0) rectangle (10, 10);
%%%%%%%%%%%%%%%%%%%%%%%%%%%%%%%%%%%%%%%%%%%%%%%%%%%%%%%%%%%%%%%%%%%%%%%%%%%%%%%
\forwardarrow{5}{5}{darkpastelgreen}
\backarrow{0}{5}{darkpastelgreen}
%%%%%%%%%%%%%%%%%%%%%%%%%%%%%%%%%%%%%%%%%%%%%%%%%%%%%%%%%%%%%%%%%%%%%%%%%%%%%%%
\myuparrow{5}{5}
\myleftarrow{5}{5}
%%%%%%%%%%%%%%%%%%%%%%%%%%%%%%%%%%%%%%%%%%%%%%%%%%%%%%%%%%%%%%%%%%%%%%%%%%%%%%%
\node[query,label=right:\midp] at (5,5) {};
\node[query,label=below right:\leftp] at (0,5) {};
\node[query,label=above right:$a$] at (0,0) {};
\node[query,label=below left:$b$] at (10,10) {};
%%%%%%%%%%%%%%%%%%%%%%%%%%%%%%%%%%%%%%%%%%%%%%%%%%%%%%%%%%%%%%%%%%%%%%%%%%%%%%%
\end{tikzpicture}
\end{lrbox}

%%%%%%%%%%%%%%%%%%%%%%%%%%%%%%%%%%%%%%%%%%%%%%%%%%%%%%%%%%%%%%%%%%%%%%%%%%%%%%%
% ROW 5 COL 2/3: 04_up_left_up_v2.tex
%%%%%%%%%%%%%%%%%%%%%%%%%%%%%%%%%%%%%%%%%%%%%%%%%%%%%%%%%%%%%%%%%%%%%%%%%%%%%%%
\newsavebox\rowfivcoltwo
\begin{lrbox}{\rowfivcoltwo}
\begin{tikzpicture}[scale=\myscale]
\tikzset{every node}=[font=\Huge]
%%%%%%%%%%%%%%%%%%%%%%%%%%%%%%%%%%%%%%%%%%%%%%%%%%%%%%%%%%%%%%%%%%%%%%%%%%%%%%%
% Phantom arrows for equal spacing of subfigs
\backarrow{10}{10}{white}
\forwardarrow{0}{0}{white}
%%%%%%%%%%%%%%%%%%%%%%%%%%%%%%%%%%%%%%%%%%%%%%%%%%%%%%%%%%%%%%%%%%%%%%%%%%%%%%%
% MAIN SQUARE
\draw (0, 0) rectangle (10, 10);
%%%%%%%%%%%%%%%%%%%%%%%%%%%%%%%%%%%%%%%%%%%%%%%%%%%%%%%%%%%%%%%%%%%%%%%%%%%%%%%
\forwardarrow{5}{5}{darkpastelgreen}
\forwardarrow{0}{5}{darkpastelgreen}
%%%%%%%%%%%%%%%%%%%%%%%%%%%%%%%%%%%%%%%%%%%%%%%%%%%%%%%%%%%%%%%%%%%%%%%%%%%%%%%
\myuparrow{5}{5}
\myleftarrow{5}{5}
\myleftarrow{0}{5}
%%%%%%%%%%%%%%%%%%%%%%%%%%%%%%%%%%%%%%%%%%%%%%%%%%%%%%%%%%%%%%%%%%%%%%%%%%%%%%%
\node[query, label=right:\midp] at (5,5) {};
\node[query, label=right:\leftp] at (0,5) {};
\node[query,label=above right:$a$] at (0,0) {};
\node[query,label=below left:$b$] at (10,10) {};
%%%%%%%%%%%%%%%%%%%%%%%%%%%%%%%%%%%%%%%%%%%%%%%%%%%%%%%%%%%%%%%%%%%%%%%%%%%%%%%
\end{tikzpicture}
\end{lrbox}

%%%%%%%%%%%%%%%%%%%%%%%%%%%%%%%%%%%%%%%%%%%%%%%%%%%%%%%%%%%%%%%%%%%%%%%%%%%%%%%
% ROW 5 COL 3/3: 04_up_left_up_rule_out_bot.tex
%%%%%%%%%%%%%%%%%%%%%%%%%%%%%%%%%%%%%%%%%%%%%%%%%%%%%%%%%%%%%%%%%%%%%%%%%%%%%%%
\newsavebox\rowfivcolthr
\begin{lrbox}{\rowfivcolthr}
\begin{tikzpicture}[scale=\myscale]
\tikzset{every node}=[font=\Huge]
%%%%%%%%%%%%%%%%%%%%%%%%%%%%%%%%%%%%%%%%%%%%%%%%%%%%%%%%%%%%%%%%%%%%%%%%%%%%%%%
% Phantom arrows for equal spacing of subfigs
\backarrow{10}{10}{white}
\forwardarrow{0}{0}{white}
%%%%%%%%%%%%%%%%%%%%%%%%%%%%%%%%%%%%%%%%%%%%%%%%%%%%%%%%%%%%%%%%%%%%%%%%%%%%%%%
% fill in lhs (early so in background)
\draw[white, fill=lightgray] (0,0) rectangle (10,5);
\draw[dashed] (0,5) -- (10,5);
%%%%%%%%%%%%%%%%%%%%%%%%%%%%%%%%%%%%%%%%%%%%%%%%%%%%%%%%%%%%%%%%%%%%%%%%%%%%%%%
% MAIN SQUARE
\draw (0, 0) rectangle (10, 10);
%%%%%%%%%%%%%%%%%%%%%%%%%%%%%%%%%%%%%%%%%%%%%%%%%%%%%%%%%%%%%%%%%%%%%%%%%%%%%%%
\forwardarrow{5}{5}{darkpastelgreen}
\forwardarrow{0}{5}{darkpastelgreen}
%%%%%%%%%%%%%%%%%%%%%%%%%%%%%%%%%%%%%%%%%%%%%%%%%%%%%%%%%%%%%%%%%%%%%%%%%%%%%%%
\myuparrow{5}{5}
\myleftarrow{5}{5}
\myrightarrow{0}{5}
%%%%%%%%%%%%%%%%%%%%%%%%%%%%%%%%%%%%%%%%%%%%%%%%%%%%%%%%%%%%%%%%%%%%%%%%%%%%%%%
\node[query, label=above right:\midp] at (5,5) {};
\node[query, label=above right:\leftp] at (0,5) {};
\node[query,label=above right:$a$] at (0,0) {};
\node[query,label=below left:$b$] at (10,10) {};
%%%%%%%%%%%%%%%%%%%%%%%%%%%%%%%%%%%%%%%%%%%%%%%%%%%%%%%%%%%%%%%%%%%%%%%%%%%%%%%
\end{tikzpicture}
\end{lrbox}

%%%%%%%%%%%%%%%%%%%%%%%%%%%%%%%%%%%%%%%%%%%%%%%%%%%%%%%%%%%%%%%%%%%%%%%%%%%%%%%
%%%%%%%%%%%%%%%%%%%%%%%%%%%%%%%%%%%%%%%%%%%%%%%%%%%%%%%%%%%%%%%%%%%%%%%%%%%%%%%
%%%%%%%%%%%%%%%%%%%%%%%%%%%%%%%%%%%%%%%%%%%%%%%%%%%%%%%%%%%%%%%%%%%%%%%%%%%%%%%
% THE MAIN FIGURE
%%%%%%%%%%%%%%%%%%%%%%%%%%%%%%%%%%%%%%%%%%%%%%%%%%%%%%%%%%%%%%%%%%%%%%%%%%%%%%%
%%%%%%%%%%%%%%%%%%%%%%%%%%%%%%%%%%%%%%%%%%%%%%%%%%%%%%%%%%%%%%%%%%%%%%%%%%%%%%%
%%%%%%%%%%%%%%%%%%%%%%%%%%%%%%%%%%%%%%%%%%%%%%%%%%%%%%%%%%%%%%%%%%%%%%%%%%%%%%%


\begin{tikzpicture}
%% ROW 1
%\subfig{0}{0}{Case 1: $\midp \in \Up(f_s)$}{\upset}
%\subfig{10}{0}{Case 2: $\midp \in \Down(f_s)$}{\dnset}

%% ROW 2
%\subfig{0}{-10}{Case 3.a.i: \midp and \rightp VOP}{\rowtwocolone}
%\subfig{10}{-10}{Case 3.a.ii: apply Lemma~\ref{lem:logn}}{\rowtwocoltwo}
%\subfig{20}{-10}{Case 3.a.iii: (\midp, \rightp) is a DSW}{\rowtwocolthr}

%% ROW 3
%\subfig{0}{-20}{Case 3.b.i: \midp and \botp VOP}{\rowthrcolone}
%\subfig{10}{-20}{Case 3.b.ii: apply Lemma~\ref{lem:logn}}{\rowthrcoltwo}
%\subfig{20}{-20}{Case 3.b.iii: (\botp, \midp) is an USW}{\rowthrcolthr}

%% ROW 4
%\subfig{0}{-30}{Case 4.a.i: \midp and \topp VOP}{\rowfoucolone}
%\subfig{10}{-30}{Case 4.a.ii: apply Lemma~\ref{lem:logn}}{\rowfoucoltwo}
%\subfig{20}{-30}{Case 4.a.iii: (\midp, \topp) is a DSW}{\rowfoucolthr}

%% ROW 5
%\subfig{0}{-40}{Case 4.b.i: \midp and \leftp VOP}{\rowfivcolone}
%\subfig{10}{-40}{Case 4.b.ii: apply Lemma~\ref{lem:logn}}{\rowfivcoltwo}
%\subfig{20}{-40}{Case 4.b.iii: (\leftp, \midp) is an USW}{\rowfivcolthr}

\pgfmathsetmacro{\rheight}{-12}

% ROW 1
% \subfigprime{0}{0}{Case 1:}{$\midp \in \Up(f_s)$}{\upset}
% \subfigprime{8}{0}{Case 2:}{$\midp \in \Down(f_s)$}{\dnset}

% ROW 2
% \subfigprime{0}{\rheight}{Case 3.a.i:}{VOP: \midp, \rightp}{\rowtwocolone}
  \subfigprime{0}{\rheight}{If $f(\mi)_3 \geq \mi_3$ and }{$f(p)_j > p_j$ then $d \in \Up(f)$.}{\rowtwocoltwo}
  \subfigprime{16}{\rheight}{If $f(\mi)_3 \geq \mi_3$ and $f(p)_j \leq p_j$}{ then $(\mi, d)$ is a down-set witness.}{\rowtwocolthr}

% ROW 3
% \subfigprime{24}{\rheight}{Case 3.b.i:}{VOP: \midp, \botp}{\rowthrcolone}
% \subfigprime{32}{\rheight}{Case 3.b.ii:}{apply Lemma~\ref{lem:logn}}{\rowthrcoltwo}
% \subfigprime{40}{\rheight}{Case 3.b.iii:}{USW: (\botp, \midp)}{\rowthrcolthr}

% % ROW 4
% \subfigprime{0}{2*\rheight}{Case 4.a.i:}{VOP: \midp, \topp}{\rowfoucolone}
% \subfigprime{8}{2*\rheight}{Case 4.a.ii:}{apply Lemma~\ref{lem:logn}}{\rowfoucoltwo}
% \subfigprime{16}{2*\rheight}{Case 4.a.iii:}{DSW: (\midp, \topp)}{\rowfoucolthr}

% % ROW 5
% \subfigprime{24}{2*\rheight}{Case 4.b.i:}{VOP: \midp, \leftp}{\rowfivcolone}
% \subfigprime{32}{2*\rheight}{Case 4.b.ii:}{apply Lemma~\ref{lem:logn}}{\rowfivcoltwo}
% \subfigprime{40}{2*\rheight}{Case 4.b.iii:}{USW: (\leftp, \midp)}{\rowfivcolthr}



\end{tikzpicture}

}
  \caption{In the main case where $u \leq \mi$, $d \geq \mi$, and $a_i - b_i \geq 2$ for $i \in \{ 1, 2\}$,
  the search space can be halved as in the proof of \cref{innerMainCase}. $\mi$ is queried, and the only non-trivial
  case is when $\mi \not\in \Up(f_s) \cup \Down(f_s)$. The diagrams show the case when $f(\mi)_3 \geq \mi_3$.
  The case $f(\mi)_3 \leq \mi_3$ then is similar. The horizontal dimension is taken to be dimension $j$ from the proof,
  and vertical is $i$. Diagram source: \citep{fasterTarski}}.
\end{figure}
\vspace{-10pt}
I must now justify the assumption in \cref{innerMainCase} that 
$u \leq \mi$ and $d \geq \mi$.
\begin{lemma}[\citep{fasterTarski}]\label{innerOtherCase}
  Suppose for some $i \in \{1, 2\}$ that $u_i \not\leq \mi_i$.
  Then a point $p \in [N]^3$ can be found such that $p_i \leq \mi_i$ and $(a, p)$ is a valid down-set
  witness, or $(p, u)$ is a valid down-set witness. 
  If $d_i \not\geq \mi$ then a point $q \in [N]^3$ can be found such that $q_i \geq \mi$
  and $(q, b)$ is a valid up-set
  witness, or $(d, q)$ is a valid up-set witness. 
\end{lemma}
\begin{proof}
  Suppose for some $i \in \{1, 2\}$ that $u_i > \mi_i$. Then
  by definition of a witness, if $j \in \{1, 2\}$ and $i \neq j$ then $u_j = a_j$.
  Put $p_j = u_j$, $p_i = \mi_i$. Then since $u \geq p$ by monotonicity
  and definition of a witness I have $f(u)_3 \geq u_3$. If $f(p)_i \leq p_i$ then
  $(a, p)$ is a valid down set witness that satisfies $p_i \leq \mi_i$. If
  $f(p)_i \geq p_i$ then $(p, u)$ is a valid down-set witness. \\
  The case with $d_i < \mi_i$ is similar; I take $p_j = d_j$, $p_i = \mi_i$ and $p_3 = \mi_3$
  and find either $(p, b)$ is a down-set witness or $(b, p)$ is a down-set witness.
\end{proof}
\vspace{-10pt}
\begin{figure}[ht]
  \centering
  \scalebox{0.4}{%%%%%%%%%%%%%%%%%%%%%%%%%%%%%%%%%%%%%%%%%%%%%%%%%%%%%%%%%%%
% STEP 1
%%%%%%%%%%%%%%%%%%%%%%%%%%%%%%%%%%%%%%%%%%%%%%%%%%%%%%%%%%%

\pgfmathsetmacro{\d}{2.5}
\pgfmathsetmacro{\dd}{10-\d}

%%%%%%%%%%%%%%%%%%%%%%%%%%%%%%%%%%%%%%%%%%%%%%%%%%%%%%%%%%%%%%%%%%%%%%%%%%%%%%%
% ROW 1 COL 1/3: 03_top_v.tex
%%%%%%%%%%%%%%%%%%%%%%%%%%%%%%%%%%%%%%%%%%%%%%%%%%%%%%%%%%%%%%%%%%%%%%%%%%%%%%%
\newsavebox\rowonecolone
\begin{lrbox}{\rowonecolone}
\begin{tikzpicture}[scale=\myscale]
\tikzset{every node}=[font=\Huge]
%%%%%%%%%%%%%%%%%%%%%%%%%%%%%%%%%%%%%%%%%%%%%%%%%%%%%%%%%%%%%%%%%%%%%%%%%%%%%%%
% Phantom arrows for equal spacing of subfigprimes
\backarrow{10}{10}{white}
\forwardarrow{0}{0}{white}
%%%%%%%%%%%%%%%%%%%%%%%%%%%%%%%%%%%%%%%%%%%%%%%%%%%%%%%%%%%%%%%%%%%%%%%%%%%%%%%
% MAIN SQUARE
\draw (0, 0) rectangle (10, 10);
%%%%%%%%%%%%%%%%%%%%%%%%%%%%%%%%%%%%%%%%%%%%%%%%%%%%%%%%%%%%%%%%%%%%%%%%%%%%%%%
\backarrow{10}{10}{darkpastelgreen}
\backarrow{\d}{10}{darkpastelgreen}
\forwardarrow{5}{10}{darkpastelgreen}
%%%%%%%%%%%%%%%%%%%%%%%%%%%%%%%%%%%%%%%%%%%%%%%%%%%%%%%%%%%%%%%%%%%%%%%%%%%%%%%
\myleftarrow{10}{10}
\myrightarrow{\d}{10}
%%%%%%%%%%%%%%%%%%%%%%%%%%%%%%%%%%%%%%%%%%%%%%%%%%%%%%%%%%%%%%%%%%%%%%%%%%%%%%%
\node[query,label=below:$d$] at (\d,10) {};
\node[query,label={[xshift=-2mm] below right:$p$}] at (5,10) {};
\node[query,label=above right:$a$] at (0,0) {};
\node[query,label=below left:$b$] at (10,10) {};
\end{tikzpicture}
\end{lrbox}

%%%%%%%%%%%%%%%%%%%%%%%%%%%%%%%%%%%%%%%%%%%%%%%%%%%%%%%%%%%%%%%%%%%%%%%%%%%%%%%
% ROW 1 COL 2/3: 03_top_rule_out_rhs.tex
%%%%%%%%%%%%%%%%%%%%%%%%%%%%%%%%%%%%%%%%%%%%%%%%%%%%%%%%%%%%%%%%%%%%%%%%%%%%%%%
\newsavebox\rowonecoltwo
\begin{lrbox}{\rowonecoltwo}
\begin{tikzpicture}[scale=\myscale]
\tikzset{every node}=[font=\Huge]
%%%%%%%%%%%%%%%%%%%%%%%%%%%%%%%%%%%%%%%%%%%%%%%%%%%%%%%%%%%%%%%%%%%%%%%%%%%%%%%
% Phantom arrows for equal spacing of subfigprimes
\backarrow{10}{10}{white}
\forwardarrow{0}{0}{white}
%%%%%%%%%%%%%%%%%%%%%%%%%%%%%%%%%%%%%%%%%%%%%%%%%%%%%%%%%%%%%%%%%%%%%%%%%%%%%%%
% fill in lhs (early so in background)
\draw[draw=none, fill=lightgray] (5,0) rectangle (10,10);
\draw[dashed] (5,10) -- (5,0);
%%%%%%%%%%%%%%%%%%%%%%%%%%%%%%%%%%%%%%%%%%%%%%%%%%%%%%%%%%%%%%%%%%%%%%%%%%%%%%%
% MAIN SQUARE
\draw (0, 0) rectangle (10, 10);
%%%%%%%%%%%%%%%%%%%%%%%%%%%%%%%%%%%%%%%%%%%%%%%%%%%%%%%%%%%%%%%%%%%%%%%%%%%%%%%
\backarrow{10}{10}{darkpastelgreen}
\backarrow{\d}{10}{darkpastelgreen}
\backarrow{5}{10}{darkpastelgreen}
%%%%%%%%%%%%%%%%%%%%%%%%%%%%%%%%%%%%%%%%%%%%%%%%%%%%%%%%%%%%%%%%%%%%%%%%%%%%%%%
\myleftarrow{10}{10}
\myrightarrow{\d}{10}
\myleftarrow{5}{10}
%%%%%%%%%%%%%%%%%%%%%%%%%%%%%%%%%%%%%%%%%%%%%%%%%%%%%%%%%%%%%%%%%%%%%%%%%%%%%%%
\node[query,label=below right:$p$] at (5,10) {};
\node[query,label=below:$d$] at (\d,10) {};
\node[query,label=above right:$a$] at (0,0) {};
\node[query,label=below left:$b$] at (10,10) {};
\end{tikzpicture}
\end{lrbox}

%%%%%%%%%%%%%%%%%%%%%%%%%%%%%%%%%%%%%%%%%%%%%%%%%%%%%%%%%%%%%%%%%%%%%%%%%%%%%%%
% ROW 1 COL 3/3: 03_top_rule_out_lhs.tex
%%%%%%%%%%%%%%%%%%%%%%%%%%%%%%%%%%%%%%%%%%%%%%%%%%%%%%%%%%%%%%%%%%%%%%%%%%%%%%%
\newsavebox\rowonecolthr
\begin{lrbox}{\rowonecolthr}
\begin{tikzpicture}[scale=\myscale]
\tikzset{every node}=[font=\Huge]
%%%%%%%%%%%%%%%%%%%%%%%%%%%%%%%%%%%%%%%%%%%%%%%%%%%%%%%%%%%%%%%%%%%%%%%%%%%%%%%
% Phantom arrows for equal spacing of subfigprimes
\backarrow{10}{10}{white}
\forwardarrow{0}{0}{white}
%%%%%%%%%%%%%%%%%%%%%%%%%%%%%%%%%%%%%%%%%%%%%%%%%%%%%%%%%%%%%%%%%%%%%%%%%%%%%%%
% MAIN SQUARE
\draw (0, 0) rectangle (10, 10);
%%%%%%%%%%%%%%%%%%%%%%%%%%%%%%%%%%%%%%%%%%%%%%%%%%%%%%%%%%%%%%%%%%%%%%%%%%%%%%%
\backarrow{10}{10}{darkpastelgreen}
\backarrow{\d}{10}{darkpastelgreen}
\backarrow{5}{10}{darkpastelgreen}
%%%%%%%%%%%%%%%%%%%%%%%%%%%%%%%%%%%%%%%%%%%%%%%%%%%%%%%%%%%%%%%%%%%%%%%%%%%%%%%
\myleftarrow{10}{10}
\myrightarrow{\d}{10}
\myrightarrow{5}{10}
%%%%%%%%%%%%%%%%%%%%%%%%%%%%%%%%%%%%%%%%%%%%%%%%%%%%%%%%%%%%%%%%%%%%%%%%%%%%%%%
\node[query,label={[xshift=-2mm] below right:$p$}] at (5,10) {};
\node[query,label=below:$d$] at (\d,10) {};
\node[query,label=above right:$a$] at (0,0) {};
\node[query,label=below left:$b$] at (10,10) {};
\end{tikzpicture}
\end{lrbox}

%%%%%%%%%%%%%%%%%%%%%%%%%%%%%%%%%%%%%%%%%%%%%%%%%%%%%%%%%%%%%%%%%%%%%%%%%%%%%%%
%%%%%%%%%%%%%%%%%%%%%%%%%%%%%%%%%%%%%%%%%%%%%%%%%%%%%%%%%%%%%%%%%%%%%%%%%%%%%%%
%%%%%%%%%%%%%%%%%%%%%%%%%%%%%%%%%%%%%%%%%%%%%%%%%%%%%%%%%%%%%%%%%%%%%%%%%%%%%%%
%%%%%%%%%%%%%%%%%%%%%%%%%%%%%%%%%%%%%%%%%%%%%%%%%%%%%%%%%%%%%%%%%%%%%%%%%%%%%%%
%%%%%%%%%%%%%%%%%%%%%%%%%%%%%%%%%%%%%%%%%%%%%%%%%%%%%%%%%%%%%%%%%%%%%%%%%%%%%%%

%%%%%%%%%%%%%%%%%%%%%%%%%%%%%%%%%%%%%%%%%%%%%%%%%%%%%%%%%%%%%%%%%%%%%%%%%%%%%%%
% ROW 2 COL 1/3: 03_right_v.tex
%%%%%%%%%%%%%%%%%%%%%%%%%%%%%%%%%%%%%%%%%%%%%%%%%%%%%%%%%%%%%%%%%%%%%%%%%%%%%%%
% \newsavebox\rowtwocolone
\begin{lrbox}{\rowtwocolone}
\begin{tikzpicture}[scale=\myscale]
\tikzset{every node}=[font=\Huge]
%%%%%%%%%%%%%%%%%%%%%%%%%%%%%%%%%%%%%%%%%%%%%%%%%%%%%%%%%%%%%%%%%%%%%%%%%%%%%%%
% Phantom arrows for equal spacing of subfigprimes
\backarrow{10}{10}{white}
\forwardarrow{0}{0}{white}
%%%%%%%%%%%%%%%%%%%%%%%%%%%%%%%%%%%%%%%%%%%%%%%%%%%%%%%%%%%%%%%%%%%%%%%%%%%%%%%
% MAIN SQUARE
\draw (0, 0) rectangle (10, 10);
%%%%%%%%%%%%%%%%%%%%%%%%%%%%%%%%%%%%%%%%%%%%%%%%%%%%%%%%%%%%%%%%%%%%%%%%%%%%%%%
\backarrow{10}{10}{darkpastelgreen}
\backarrow{10}{\d}{darkpastelgreen}
\forwardarrow{10}{5}{darkpastelgreen}
%%%%%%%%%%%%%%%%%%%%%%%%%%%%%%%%%%%%%%%%%%%%%%%%%%%%%%%%%%%%%%%%%%%%%%%%%%%%%%%
\mydnarrow{10}{10}
\myuparrow{10}{\d}
%%%%%%%%%%%%%%%%%%%%%%%%%%%%%%%%%%%%%%%%%%%%%%%%%%%%%%%%%%%%%%%%%%%%%%%%%%%%%%%
\node[query,label=left:$d$] at (10,\d) {};
\node[query,label={[yshift=-3mm] above left:\rightp}] at (10,5) {};
\node[query,label=above right:$a$] at (0,0) {};
\node[query,label=below left:$b$] at (10,10) {};
\end{tikzpicture}
\end{lrbox}

%%%%%%%%%%%%%%%%%%%%%%%%%%%%%%%%%%%%%%%%%%%%%%%%%%%%%%%%%%%%%%%%%%%%%%%%%%%%%%%
% ROW 2 COL 2/3: 03_right_rule_out_top.tex
%%%%%%%%%%%%%%%%%%%%%%%%%%%%%%%%%%%%%%%%%%%%%%%%%%%%%%%%%%%%%%%%%%%%%%%%%%%%%%%
% \newsavebox\rowtwocoltwo
\begin{lrbox}{\rowtwocoltwo}
\begin{tikzpicture}[scale=\myscale]
\tikzset{every node}=[font=\Huge]
%%%%%%%%%%%%%%%%%%%%%%%%%%%%%%%%%%%%%%%%%%%%%%%%%%%%%%%%%%%%%%%%%%%%%%%%%%%%%%%
% Phantom arrows for equal spacing of subfigprimes
\backarrow{10}{10}{white}
\forwardarrow{0}{0}{white}
%%%%%%%%%%%%%%%%%%%%%%%%%%%%%%%%%%%%%%%%%%%%%%%%%%%%%%%%%%%%%%%%%%%%%%%%%%%%%%%
% fill in lhs (early so in background)
\draw[draw=none, fill=lightgray] (0,5) rectangle (10,10);
\draw[dashed] (0,5) -- (10,5);
%%%%%%%%%%%%%%%%%%%%%%%%%%%%%%%%%%%%%%%%%%%%%%%%%%%%%%%%%%%%%%%%%%%%%%%%%%%%%%%
% MAIN SQUARE
\draw (0, 0) rectangle (10, 10);
%%%%%%%%%%%%%%%%%%%%%%%%%%%%%%%%%%%%%%%%%%%%%%%%%%%%%%%%%%%%%%%%%%%%%%%%%%%%%%%
\backarrow{10}{10}{darkpastelgreen}
\backarrow{10}{\d}{darkpastelgreen}
\backarrow{10}{5}{darkpastelgreen}
%%%%%%%%%%%%%%%%%%%%%%%%%%%%%%%%%%%%%%%%%%%%%%%%%%%%%%%%%%%%%%%%%%%%%%%%%%%%%%%
\mydnarrow{10}{10}
\myuparrow{10}{\d}
\mydnarrow{10}{5}
%%%%%%%%%%%%%%%%%%%%%%%%%%%%%%%%%%%%%%%%%%%%%%%%%%%%%%%%%%%%%%%%%%%%%%%%%%%%%%%
\node[query, label={[yshift=-2mm] above left:\rightp}] at (10,5) {};
\node[query,label=left:$d$] at (10,\d) {};
\node[query,label=above right:$a$] at (0,0) {};
\node[query,label=below left:$b$] at (10,10) {};
\end{tikzpicture}
\end{lrbox}

%%%%%%%%%%%%%%%%%%%%%%%%%%%%%%%%%%%%%%%%%%%%%%%%%%%%%%%%%%%%%%%%%%%%%%%%%%%%%%%
% ROW 2 COL 3/3: 03_right_rule_out_bot.tex
%%%%%%%%%%%%%%%%%%%%%%%%%%%%%%%%%%%%%%%%%%%%%%%%%%%%%%%%%%%%%%%%%%%%%%%%%%%%%%%
% \newsavebox\rowtwocolthr
\begin{lrbox}{\rowtwocolthr}
\begin{tikzpicture}[scale=\myscale]
\tikzset{every node}=[font=\Huge]
%%%%%%%%%%%%%%%%%%%%%%%%%%%%%%%%%%%%%%%%%%%%%%%%%%%%%%%%%%%%%%%%%%%%%%%%%%%%%%%
% Phantom arrows for equal spacing of subfigprimes
\backarrow{10}{10}{white}
\forwardarrow{0}{0}{white}
%%%%%%%%%%%%%%%%%%%%%%%%%%%%%%%%%%%%%%%%%%%%%%%%%%%%%%%%%%%%%%%%%%%%%%%%%%%%%%%
% MAIN SQUARE
\draw (0, 0) rectangle (10, 10);
%%%%%%%%%%%%%%%%%%%%%%%%%%%%%%%%%%%%%%%%%%%%%%%%%%%%%%%%%%%%%%%%%%%%%%%%%%%%%%%
\backarrow{10}{10}{darkpastelgreen}
\backarrow{10}{\d}{darkpastelgreen}
\backarrow{10}{5}{darkpastelgreen}
%%%%%%%%%%%%%%%%%%%%%%%%%%%%%%%%%%%%%%%%%%%%%%%%%%%%%%%%%%%%%%%%%%%%%%%%%%%%%%%
\mydnarrow{10}{10}
\myuparrow{10}{\d}
\myuparrow{10}{5}
%%%%%%%%%%%%%%%%%%%%%%%%%%%%%%%%%%%%%%%%%%%%%%%%%%%%%%%%%%%%%%%%%%%%%%%%%%%%%%%
\node[query, label={[yshift=-4mm] above left:\rightp}] at (10,5) {};
\node[query,label=left:$d$] at (10,\d) {};
\node[query,label=above right:$a$] at (0,0) {};
\node[query,label=below left:$b$] at (10,10) {};

\end{tikzpicture}
\end{lrbox}

%%%%%%%%%%%%%%%%%%%%%%%%%%%%%%%%%%%%%%%%%%%%%%%%%%%%%%%%%%%%%%%%%%%%%%%%%%%%%%%
%%%%%%%%%%%%%%%%%%%%%%%%%%%%%%%%%%%%%%%%%%%%%%%%%%%%%%%%%%%%%%%%%%%%%%%%%%%%%%%
%%%%%%%%%%%%%%%%%%%%%%%%%%%%%%%%%%%%%%%%%%%%%%%%%%%%%%%%%%%%%%%%%%%%%%%%%%%%%%%
%%%%%%%%%%%%%%%%%%%%%%%%%%%%%%%%%%%%%%%%%%%%%%%%%%%%%%%%%%%%%%%%%%%%%%%%%%%%%%%
%%%%%%%%%%%%%%%%%%%%%%%%%%%%%%%%%%%%%%%%%%%%%%%%%%%%%%%%%%%%%%%%%%%%%%%%%%%%%%%

%%%%%%%%%%%%%%%%%%%%%%%%%%%%%%%%%%%%%%%%%%%%%%%%%%%%%%%%%%%%%%%%%%%%%%%%%%%%%%%
% ROW 3 COL 1/3: 03_bot_v.tex
%%%%%%%%%%%%%%%%%%%%%%%%%%%%%%%%%%%%%%%%%%%%%%%%%%%%%%%%%%%%%%%%%%%%%%%%%%%%%%%
% \newsavebox\rowthrcolone
\begin{lrbox}{\rowthrcolone}
\begin{tikzpicture}[scale=\myscale]
\tikzset{every node}=[font=\Huge]
%%%%%%%%%%%%%%%%%%%%%%%%%%%%%%%%%%%%%%%%%%%%%%%%%%%%%%%%%%%%%%%%%%%%%%%%%%%%%%%
% Phantom arrows for equal spacing of subfigprimes
\backarrow{10}{10}{white}
\forwardarrow{0}{0}{white}
%%%%%%%%%%%%%%%%%%%%%%%%%%%%%%%%%%%%%%%%%%%%%%%%%%%%%%%%%%%%%%%%%%%%%%%%%%%%%%%
% MAIN SQUARE
\draw (0, 0) rectangle (10, 10);
%%%%%%%%%%%%%%%%%%%%%%%%%%%%%%%%%%%%%%%%%%%%%%%%%%%%%%%%%%%%%%%%%%%%%%%%%%%%%%%
\forwardarrow{0}{0}{darkpastelgreen}
\forwardarrow{\dd}{0}{darkpastelgreen}
\backarrow{5}{0}{darkpastelgreen}
%%%%%%%%%%%%%%%%%%%%%%%%%%%%%%%%%%%%%%%%%%%%%%%%%%%%%%%%%%%%%%%%%%%%%%%%%%%%%%%
\myrightarrow{0}{0}
\myleftarrow{\dd}{0}
%%%%%%%%%%%%%%%%%%%%%%%%%%%%%%%%%%%%%%%%%%%%%%%%%%%%%%%%%%%%%%%%%%%%%%%%%%%%%%%
%%%%%%%%%%%%%%%%%%%%%%%%%%%%%%%%%%%%%%%%%%%%%%%%%%%%%%%%%%%%%%%%%%%%%%%%%%%%%%%
\node[query,label=above:$u$] at (\dd,0) {};
\node[query,label={[xshift=2mm] above left:$p$}] at (5,0) {};
\node[query,label=above right:$a$] at (0,0) {};
\node[query,label=below left:$b$] at (10,10) {};
\end{tikzpicture}
\end{lrbox}

%%%%%%%%%%%%%%%%%%%%%%%%%%%%%%%%%%%%%%%%%%%%%%%%%%%%%%%%%%%%%%%%%%%%%%%%%%%%%%%
% ROW 3 COL 2/3: 03_bot_rule_out_lhs.tex
%%%%%%%%%%%%%%%%%%%%%%%%%%%%%%%%%%%%%%%%%%%%%%%%%%%%%%%%%%%%%%%%%%%%%%%%%%%%%%%
% \newsavebox\rowthrcoltwo
\begin{lrbox}{\rowthrcoltwo}
\begin{tikzpicture}[scale=\myscale]
\tikzset{every node}=[font=\Huge]
%%%%%%%%%%%%%%%%%%%%%%%%%%%%%%%%%%%%%%%%%%%%%%%%%%%%%%%%%%%%%%%%%%%%%%%%%%%%%%%
% Phantom arrows for equal spacing of subfigprimes
\backarrow{10}{10}{white}
\forwardarrow{0}{0}{white}
%%%%%%%%%%%%%%%%%%%%%%%%%%%%%%%%%%%%%%%%%%%%%%%%%%%%%%%%%%%%%%%%%%%%%%%%%%%%%%%
% fill in lhs (early so in background)
\draw[draw=none, fill=lightgray] (0,0) rectangle (5,10);
\draw[dashed] (5,10) -- (5,0);
%%%%%%%%%%%%%%%%%%%%%%%%%%%%%%%%%%%%%%%%%%%%%%%%%%%%%%%%%%%%%%%%%%%%%%%%%%%%%%%
% MAIN SQUARE
\draw (0, 0) rectangle (10, 10);
%%%%%%%%%%%%%%%%%%%%%%%%%%%%%%%%%%%%%%%%%%%%%%%%%%%%%%%%%%%%%%%%%%%%%%%%%%%%%%%
\forwardarrow{0}{0}{darkpastelgreen}
\forwardarrow{\dd}{0}{darkpastelgreen}
\forwardarrow{5}{0}{darkpastelgreen}
%%%%%%%%%%%%%%%%%%%%%%%%%%%%%%%%%%%%%%%%%%%%%%%%%%%%%%%%%%%%%%%%%%%%%%%%%%%%%%%
\myrightarrow{0}{0}
\myleftarrow{\dd}{0}
\myrightarrow{5}{0}
%%%%%%%%%%%%%%%%%%%%%%%%%%%%%%%%%%%%%%%%%%%%%%%%%%%%%%%%%%%%%%%%%%%%%%%%%%%%%%%
\node[query,label=above:$u$] at (\dd,0) {};
\node[query,label={[xshift=1mm] above left:$p$}] at (5,0) {};
\node[query,label=above right:$a$] at (0,0) {};
\node[query,label=below left:$b$] at (10,10) {};
\end{tikzpicture}
\end{lrbox}

%%%%%%%%%%%%%%%%%%%%%%%%%%%%%%%%%%%%%%%%%%%%%%%%%%%%%%%%%%%%%%%%%%%%%%%%%%%%%%%
% ROW 3 COL 3/3: 03_bot_rule_out_rhs.tex
%%%%%%%%%%%%%%%%%%%%%%%%%%%%%%%%%%%%%%%%%%%%%%%%%%%%%%%%%%%%%%%%%%%%%%%%%%%%%%%
% \newsavebox\rowthrcolthr
\begin{lrbox}{\rowthrcolthr}
\begin{tikzpicture}[scale=\myscale]
\tikzset{every node}=[font=\Huge]
%%%%%%%%%%%%%%%%%%%%%%%%%%%%%%%%%%%%%%%%%%%%%%%%%%%%%%%%%%%%%%%%%%%%%%%%%%%%%%%
% Phantom arrows for equal spacing of subfigprimes
\backarrow{10}{10}{white}
\forwardarrow{0}{0}{white}
%%%%%%%%%%%%%%%%%%%%%%%%%%%%%%%%%%%%%%%%%%%%%%%%%%%%%%%%%%%%%%%%%%%%%%%%%%%%%%%
% MAIN SQUARE
\draw (0, 0) rectangle (10, 10);
%%%%%%%%%%%%%%%%%%%%%%%%%%%%%%%%%%%%%%%%%%%%%%%%%%%%%%%%%%%%%%%%%%%%%%%%%%%%%%%
\forwardarrow{0}{0}{darkpastelgreen}
\forwardarrow{\dd}{0}{darkpastelgreen}
\forwardarrow{5}{0}{darkpastelgreen}
%%%%%%%%%%%%%%%%%%%%%%%%%%%%%%%%%%%%%%%%%%%%%%%%%%%%%%%%%%%%%%%%%%%%%%%%%%%%%%%
\myrightarrow{0}{0}
\myleftarrow{\dd}{0}
\myleftarrow{5}{0}
%%%%%%%%%%%%%%%%%%%%%%%%%%%%%%%%%%%%%%%%%%%%%%%%%%%%%%%%%%%%%%%%%%%%%%%%%%%%%%%
\node[query,label={[xshift=-2mm] above:$p$}] at (5,0) {};
\node[query,label=above:$u$] at (\dd,0) {};
\node[query,label=above right:$a$] at (0,0) {};
\node[query,label=below left:$b$] at (10,10) {};
\end{tikzpicture}
\end{lrbox}

%%%%%%%%%%%%%%%%%%%%%%%%%%%%%%%%%%%%%%%%%%%%%%%%%%%%%%%%%%%%%%%%%%%%%%%%%%%%%%%
% ROW 4 COL 1/3: 03_left_v.tex
%%%%%%%%%%%%%%%%%%%%%%%%%%%%%%%%%%%%%%%%%%%%%%%%%%%%%%%%%%%%%%%%%%%%%%%%%%%%%%%
% \newsavebox\rowfoucolone
\begin{lrbox}{\rowfoucolone}
\begin{tikzpicture}[scale=\myscale]
\tikzset{every node}=[font=\Huge]
%%%%%%%%%%%%%%%%%%%%%%%%%%%%%%%%%%%%%%%%%%%%%%%%%%%%%%%%%%%%%%%%%%%%%%%%%%%%%%%
% Phantom arrows for equal spacing of subfigprimes
\backarrow{10}{10}{white}
\forwardarrow{0}{0}{white}
%%%%%%%%%%%%%%%%%%%%%%%%%%%%%%%%%%%%%%%%%%%%%%%%%%%%%%%%%%%%%%%%%%%%%%%%%%%%%%%
% MAIN SQUARE
\draw (0, 0) rectangle (10, 10);
%%%%%%%%%%%%%%%%%%%%%%%%%%%%%%%%%%%%%%%%%%%%%%%%%%%%%%%%%%%%%%%%%%%%%%%%%%%%%%%
\forwardarrow{0}{0}{darkpastelgreen}
\forwardarrow{0}{\dd}{darkpastelgreen}
\backarrow{0}{5}{darkpastelgreen}
%%%%%%%%%%%%%%%%%%%%%%%%%%%%%%%%%%%%%%%%%%%%%%%%%%%%%%%%%%%%%%%%%%%%%%%%%%%%%%%
\myuparrow{0}{0}
\mydnarrow{0}{\dd}
%%%%%%%%%%%%%%%%%%%%%%%%%%%%%%%%%%%%%%%%%%%%%%%%%%%%%%%%%%%%%%%%%%%%%%%%%%%%%%%
\node[query,label={[yshift=1mm] below right:\leftp}] at (0,5) {};
\node[query,label=right:$u$] at (0,\dd) {};
\node[query,label=above right:$a$] at (0,0) {};
\node[query,label=below left:$b$] at (10,10) {};
\end{tikzpicture}
\end{lrbox}

%%%%%%%%%%%%%%%%%%%%%%%%%%%%%%%%%%%%%%%%%%%%%%%%%%%%%%%%%%%%%%%%%%%%%%%%%%%%%%%
% ROW 4 COL 2/3: 03_left_rule_out_bot.tex
%%%%%%%%%%%%%%%%%%%%%%%%%%%%%%%%%%%%%%%%%%%%%%%%%%%%%%%%%%%%%%%%%%%%%%%%%%%%%%%
% \newsavebox\rowfoucoltwo
\begin{lrbox}{\rowfoucoltwo}
\begin{tikzpicture}[scale=\myscale]
\tikzset{every node}=[font=\Huge]
%%%%%%%%%%%%%%%%%%%%%%%%%%%%%%%%%%%%%%%%%%%%%%%%%%%%%%%%%%%%%%%%%%%%%%%%%%%%%%%
% Phantom arrows for equal spacing of subfigprimes
\backarrow{10}{10}{white}
\forwardarrow{0}{0}{white}
%%%%%%%%%%%%%%%%%%%%%%%%%%%%%%%%%%%%%%%%%%%%%%%%%%%%%%%%%%%%%%%%%%%%%%%%%%%%%%%
% fill in lhs (early so in background)
\draw[draw=none, fill=lightgray] (0,0) rectangle (10,5);
\draw[dashed] (0,5) -- (10,5);
%%%%%%%%%%%%%%%%%%%%%%%%%%%%%%%%%%%%%%%%%%%%%%%%%%%%%%%%%%%%%%%%%%%%%%%%%%%%%%%
% MAIN SQUARE
\draw (0, 0) rectangle (10, 10);
%%%%%%%%%%%%%%%%%%%%%%%%%%%%%%%%%%%%%%%%%%%%%%%%%%%%%%%%%%%%%%%%%%%%%%%%%%%%%%%
\forwardarrow{0}{0}{darkpastelgreen}
\forwardarrow{0}{\dd}{darkpastelgreen}
\forwardarrow{0}{5}{darkpastelgreen}
%%%%%%%%%%%%%%%%%%%%%%%%%%%%%%%%%%%%%%%%%%%%%%%%%%%%%%%%%%%%%%%%%%%%%%%%%%%%%%%
\myuparrow{0}{0}
\mydnarrow{0}{\dd}
\myuparrow{0}{5}
%%%%%%%%%%%%%%%%%%%%%%%%%%%%%%%%%%%%%%%%%%%%%%%%%%%%%%%%%%%%%%%%%%%%%%%%%%%%%%%
\node[query,label=below right:\leftp] at (0,5) {};
\node[query,label=right:$u$] at (0,\dd) {};
\node[query,label=above right:$a$] at (0,0) {};
\node[query,label=below left:$b$] at (10,10) {};
\end{tikzpicture}
\end{lrbox}

%%%%%%%%%%%%%%%%%%%%%%%%%%%%%%%%%%%%%%%%%%%%%%%%%%%%%%%%%%%%%%%%%%%%%%%%%%%%%%%
% ROW 4 COL 3/3: 03_left_rule_out_top.tex
%%%%%%%%%%%%%%%%%%%%%%%%%%%%%%%%%%%%%%%%%%%%%%%%%%%%%%%%%%%%%%%%%%%%%%%%%%%%%%%
% \newsavebox\rowfoucolthr
\begin{lrbox}{\rowfoucolthr}
\begin{tikzpicture}[scale=\myscale]
\tikzset{every node}=[font=\Huge]
%%%%%%%%%%%%%%%%%%%%%%%%%%%%%%%%%%%%%%%%%%%%%%%%%%%%%%%%%%%%%%%%%%%%%%%%%%%%%%%
% Phantom arrows for equal spacing of subfigprimes
\backarrow{10}{10}{white}
\forwardarrow{0}{0}{white}
%%%%%%%%%%%%%%%%%%%%%%%%%%%%%%%%%%%%%%%%%%%%%%%%%%%%%%%%%%%%%%%%%%%%%%%%%%%%%%%
% MAIN SQUARE
\draw (0, 0) rectangle (10, 10);
%%%%%%%%%%%%%%%%%%%%%%%%%%%%%%%%%%%%%%%%%%%%%%%%%%%%%%%%%%%%%%%%%%%%%%%%%%%%%%%
\forwardarrow{0}{0}{darkpastelgreen}
\forwardarrow{0}{\dd}{darkpastelgreen}
\forwardarrow{0}{5}{darkpastelgreen}
%%%%%%%%%%%%%%%%%%%%%%%%%%%%%%%%%%%%%%%%%%%%%%%%%%%%%%%%%%%%%%%%%%%%%%%%%%%%%%%
\myuparrow{0}{0}
\mydnarrow{0}{\dd}
\mydnarrow{0}{5}
%%%%%%%%%%%%%%%%%%%%%%%%%%%%%%%%%%%%%%%%%%%%%%%%%%%%%%%%%%%%%%%%%%%%%%%%%%%%%%%
\node[query, label={[yshift=-1mm] right:\leftp}] at (0,5) {};
\node[query,label=right:$u$] at (0,\dd) {};
\node[query,label=above right:$a$] at (0,0) {};
\node[query,label=below left:$b$] at (10,10) {};
\end{tikzpicture}
\end{lrbox}

%%%%%%%%%%%%%%%%%%%%%%%%%%%%%%%%%%%%%%%%%%%%%%%%%%%%%%%%%%%%%%%%%%%%%%%%%%%%%%%
%%%%%%%%%%%%%%%%%%%%%%%%%%%%%%%%%%%%%%%%%%%%%%%%%%%%%%%%%%%%%%%%%%%%%%%%%%%%%%%
%%%%%%%%%%%%%%%%%%%%%%%%%%%%%%%%%%%%%%%%%%%%%%%%%%%%%%%%%%%%%%%%%%%%%%%%%%%%%%%
%%%%%%%%%%%%%%%%%%%%%%%%%%%%%%%%%%%%%%%%%%%%%%%%%%%%%%%%%%%%%%%%%%%%%%%%%%%%%%%
%%%%%%%%%%%%%%%%%%%%%%%%%%%%%%%%%%%%%%%%%%%%%%%%%%%%%%%%%%%%%%%%%%%%%%%%%%%%%%%

%%%%%%%%%%%%%%%%%%%%%%%%%%%%%%%%%%%%%%%%%%%%%%%%%%%%%%%%%%%%%%%%%%%%%%%%%%%%%%%
% THE OVERALL GRID
%%%%%%%%%%%%%%%%%%%%%%%%%%%%%%%%%%%%%%%%%%%%%%%%%%%%%%%%%%%%%%%%%%%%%%%%%%%%%%%
\begin{tikzpicture}

% ROW 1
% \subfigprime{0}{0}{Case 1.a:}{$d$ and $\topp$ VOP}{\rowonecolone}
%   \subfigprime{0}{0}{If $f(p)_i \leq p_i$ then}{$(d, p)$ is a down-set witness}{\rowonecoltwo}
% \subfigprime{16}{0}{If $f(p)_i \geq p_i$ then}{$(p, b)$ is a down-set witness}{\rowonecolthr}

% ROW 2
% \subfigprime{0}{-12}{Case 2.a:}{$d$ and $\rightp$ VOP}{\rowtwocolone}
% \subfigprime{10}{-12}{Case 2.b:}{$(d, \rightp)$ is a DSW}{\rowtwocoltwo}
% \subfigprime{20}{-12}{Case 2.c:}{$(\rightp, b)$ is a DSW}{\rowtwocolthr}

% % ROW 3
% \subfigprime{0}{-24}{Case 3.a:}{$u$ and $\botp$ VOP}{\rowthrcolone}
\subfigprime{10}{-24}{If $f(p)_i \leq p_i$ then}{$(u, p)$ is an up-set witness}{\rowthrcoltwo}
\subfigprime{20}{-24}{If $f(p)_i \geq p_i$ then}{$(p, a)$ is an up-set witness}{\rowthrcolthr}

% % ROW 4
% \subfigprime{0}{-36}{Case 4.a:}{$u$ and $\leftp$ VOP}{\rowfoucolone}
% \subfigprime{10}{-36}{Case 4.b:}{$(u, \leftp)$ is a USW}{\rowfoucoltwo}
% \subfigprime{20}{-36}{Case 4.c:}{$(\leftp, a)$ is a USW}{\rowfoucolthr}

\end{tikzpicture}
}
  \caption{The main case described in \cref{innerMainCase} breaks down if $u \not\leq \mi$ or $d \not\geq \mi$.
  This can be rectified via \cref{innerOtherCase}. The horizontal dimension is taken
  to be $i$ from the proof of \cref{innerOtherCase}, and vertical $j$. Diagram source: \citep{fasterTarski}}
\end{figure}
Combining
the previous two lemmas, a 'normal' iteration of the inner algorithm is as follows.
Check if $u < \mi$ or $d > \mi$. If so, using \cref{innerOtherCase} either find a new
set of witnesses with a search space that is half the size of the previous, or a new of
witnesses such that $u \leq \mi$ and $d \geq \mi$ and continue to the next iteration.
If $u \geq \mi$ and $d \leq \mi$ then using \cref{innerMainCase} find a new set of witnesses
such that the new search space is at most half the size of the previous.
With the additional fact from \citep{fasterTarski} that the search space can be halfed in a constant amount of
queries when the search space as a width of at most 1, this gives an $O(\log N)$ query algorithm
for the $\trsks(N, 3)$ problem.

