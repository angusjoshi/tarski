\section{Overview}
Recent developments have been made in upper bounds for the $\trsk$ problem.
The critical component to the algorithmic improvements is
the surprising result from \citep{fasterTarski} that $\trsk(N, 3)$
can be solved in at most $O(\log^2 N)$ queries.
The idea is roughly as follows; suppose I have
a monotone function $f : [N]^3 \to [N]^3$ and an algorithm which
given a coordinate $i \in \{1, 2, 3\}$ and a value $v \in [N]$
returns a monotone point $x \in [N]^3$ with the guarantee
that $x_i = v$. If this algorithm takes $q(N)$ queries in the worst case,
then a fixpoint of $f$ can be found in $O(\log N \cdot q(N))$ in the
same fashion as \cref{dQiYiAlg}. This sub-problem will prove to be
important, so I will define it in it's own right.
\newcommand{\trsks}{\textsc{FindMonotone}}
\begin{definition}[\trsks]
  The problem $\trsks(N, d)$ is, given oracle access to a monotone function $f : [N]^d \to [N]^d$,
  a coordinate $i \in \{1, ..., d\}$, and value $v \in [N]$, find a monotone point $x \in [N]^d$ such that
  $x_i = v$. 
\end{definition}
The breakthrough in \citep{fasterTarski} was in detailing a so-called inner-algorithm which leads to
the following result.
\begin{theorem}[\citep{fasterTarski}] \label{tightThreeDimension}
  The query complexity of $\trsks(N, 3)$ is $O(\log N)$.
\end{theorem}
Fearnley et al. also gave a method of decomposition, allowing for $\trsk$ in higher
dimensions to be decomposed into a product of sorts of lower
dimensional problems. This can be seen as a generalization of the method in \cref{dQiYiAlg}.
Dang et al. solve $\trsk(N, d)$ by solving an instance $A$ of $\trsk(N, 1)$ 
where every query of the monotone function from $\trsk(N, 1)$ at $v$ is answered by recursively finding
a fixpoint in the $d-1$ dimensional slice at the $d$-th coordinate with value $v$. It then follows that
a soution to $A$ can be used to recover a solution of the entire problem. The algorithm of
Fearnley et al. shows that it is possible to make more general decompositions using a method
which I call \emph{fixpoint decomposition}. There is slight complication with making this precise
which will be detailed in a later section of this chapter.
\begin{theorem}[Fixpoint Decomposition, \citep{fasterTarski}]\label{fixDecomp}
  For positive integers $a, b \in \zpos$, given an algorithm $A$
  which can solve $\trsk(N, a)$ in $p(N, a)$ queries, and an algorithm $B$
  which can solve $\trsk(N, b)$ in $p(N, b)$ queries, the problem
  $\trsk(N, a + b)$ can be solved in $O(p(N, a)\cdot q(N, b))$ queries.
\end{theorem}
Chen and Li take this one step further; instead of decomposing the $\trsk$ problem
they show that it is possible to make a decomposition on $\trsks$ using a method
I will call \emph{monotone decomposition}. The details
of this are more complicated still than \cref{fixDecomp}, and will be shared in a later
section.
\begin{theorem}[Monotone Decomposition, \citep{chenLi}]
  For positive integers $a, b \in \zpos$, given an algorithm $A$
  which can solve $\trsks(N, a)$ in $p(N, a)$ queries, and an algorithm $B$
  which an solve $\trsks(N, b)$ in $q(N, b)$ queries, the problem $\trsks(N, a + b)$
  can be solved in $O((b + 1) \cdot p(N, a) \cdot q(N, b))$ queries.
\end{theorem}

