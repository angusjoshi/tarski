\section{Fixpoint Decomposition} \label{fixDecompChapter}
Throughout I will fix $a, b \in \zpos$,
$d = a + b$, and $f : [N]^a \times [N]^b \to [N]^a \times [N]^b$.
Suppose I have an algorithm $A$ which can solve $\trsk(N, a)$,
and an algorithm $B$ which can solve $\trsk(N, b)$. A naive approach 
for finding a fixpoint of $f$ would be to define a
function on the right hand side of the lattice
$f_r : [N]^b \to [N]^b$ where given $x_r \in [N]^b$ the value of 
$f_r(x_r)$
is computed by defining the slice $f_l : [N]^a \to [N]^a$
such that $f_l(x_l) = f((x_l, x_r))$, finding a fixpoint $x_l^* \in [N]^a$
of $f_l$, then using $f(x_l^*, x_r)_{-l}$ as the result of $f_r(x_r)$.
The punchline is then that if $x_r$ is a fixpoint of $f_r$, then if
$x_l^*$ was the fixpoint of $f_l$ associated with $x_r$ the point
$(x_l^*, x_r)$ is a fixpoint of $f$. This does not work however -
there is no guarantee that $f_r$ is monotone. That is,
if points $x_r, x_r' \in [N]^b$ are queried by algorithm $B$
with $x_r \leq x_r'$ it is not necessarily the case that the associated
fixpoints of $f_l$ $x_l^*$ and $x_l'^* \in [N]^a$ satisfy $x_l^* \leq x_l'^*$, so
monotonocity of $f$ does not carry over. The trick will thus be
to find a way to guarantee that $x_l*$ and $x_l'^*$ satisfy 
$x_l^* \leq x_l'^*$ whenever $x_r \leq x_r'$. Fortunately
this is achieveable; in \cref{fixDecompAlg} the issue is solved
by carefully choosing bounds of the sublattice to search in based
on previously computed points. The correctness of which will be
the concern of the remainder of this section.
An abuse of notation $x_{-r}$ or $x_{-l}$ is often used to denote
the first $a$ coordinates, or last $b$ coordinates of $x$ respectively.
\begin{algorithm}[h]
  \caption{\citep{fasterTarski}}\label{fixDecompAlg}
  \begin{algorithmic}[1]
    \Procedure{FixpointDecomposition}{\\
      \qquad monotone $f : [N]^a \times [N]^b \to [N]^a \times [N]^b$,\\
      \qquad algorithm $A$ for solving $\trsk(N, a)$, \\
      \qquad algorithm $B$ for solving $\trsk(N, b)$ 
    }
    \State $\mathrm{prev} \gets \emptyset$
    \Procedure{$f_r$}{$x_r \in [N]^b$}
      \Procedure{$f_l$}{$x_l \in [N]^a$}
        \Return $f((x_l, x_r))_{-r}$.
      \EndProcedure
      \State $\bot_l \gets \bigvee\{p_l : (p_l, p_r) \in \mathrm{prev}, \; p_r \leq x_r \}$
      \State $\top_r \gets \bigwedge\{p_l : (p_l, p_r) \in \mathrm{prev}, \; p_r \geq x_r \}$
      \State $x_l^* \gets A(f_l)$ with bounds $\bot_l$, $\top_l$
      % \State using algorithm $A$ find a fixpoint $x_l^*$ of $f_l$ in the sublattice with bounds
      % $\bot_l$, $\top_l$.
      \State $\mathrm{prev} \gets \mathrm{prev} \cup \{(x_l^*, x_r)\}$
      \State \Return $f((x_l^*, x_r))_{-l}$.
    \EndProcedure
    \State $x_r^* \gets B(f_r)$
    % \State run algorithm $B$ on $f_r$ to find fixpoint $x_r^*$
    \State \Return $(x_l^*, x_r^*)$ where $x_l^*$ is the fixpoint of $f_l$ 
    found when evaluating $f_r(x_r^*)$.
  \EndProcedure
  \end{algorithmic}
\end{algorithm}

\begin{lemma}\label{orderPreserved}
  If $(p_l, p_r), (p_l', p_r') \in \mathrm{prev}$ with $p_r \leq p_r'$ then
  $p_l \leq p_l'$
\end{lemma}
\begin{proof}
  Suppose without loss of generality that $p_r$ was queried by algorithm $B$
  before $p_r'$. Then when $p_r'$ is queried, $p_l$ is an element of
  $\{pl : (p_l, p_r) \in \mathrm{prev} : p_r \leq p_r'\}$ whence $\bot_l \geq p_l$,
  so the fixpoint found by algorithm $A$ satisfies $x_l^* \geq p_l$.
\end{proof}
\begin{lemma}
  At line 7 of \cref{fixDecompAlg} $\bot_l \leq \top_l$.
\end{lemma}
\begin{proof}
  By \cref{orderPreserved} I have for all 
  $p_l \in \{p_l : (p_l, p_r) \in \mathrm{prev} : p_r \leq x_r \}$ and
  $p_l' \in \{p_l : (p_l, p_r) \in \mathrm{prev} : p_r \geq x_r \}$ that
  $p_l \leq p_l'$. Then by definition of $\wedge$ and $\vee$ $\bot_l \leq \top_l$.
\end{proof}
\begin{lemma}\label{leftRestricts}
  At line 7 of \cref{fixDecompAlg} I have $f_l(\bot_l) \geq \bot_l$ and $f_l(\top_l) \leq \top_l$.
\end{lemma}
\begin{proof}
  I prove the first case and the second is similar. Suppose not. That is,
  for some $i \in [a]$ I have $f_l(\bot_l)_i < (\bot_l)_i$. If 
  $L = \{p_l : (p_l, p_r) \in \mathrm{prev} : p_r \leq x_r \}$ is empty then
  $\vee L = \vec{1}$ and I contradict the definition of $f$. If $L$ is non-empty then by definition
  of finite joins, there is some $(p_l, p_r) \in \mathrm{prev}$ such that $(p_l)_i = (\bot_l)_i$
  and $(p_l, p_r) \leq (x_l, x_r)$. But $f(((p_l, p_r))_{-r})_i = (p_l)_i$ so
  I contradict monotinicity.
\end{proof}
\begin{lemma}
  At line 7 of $\cref{fixDecompAlg}$ $f_l$ is a monotone function on the lattice
  $[\bot_l, \top_l]$.
\end{lemma}
\begin{proof}
  That $f_l$ is monotone follows from an inductive application of \cref{sliceMonotone}.
  Then the rest follows from \cref{leftRestricts} and \cref{restricts}.
\end{proof}
\begin{prop}
  The point $(x_l^*, x_r^*)$ returned by \cref{fixDecompAlg} is a fixpoint of $f$.
\end{prop}
\begin{proof}
  \cref{orderPreserved} guarantees that $f_r$ is monotone, and hence
  on line 10 a fixpoint $x_r^*$ can be found. Then by construction
  $(x_l^*, x_r^*)$ is clearly a fixpoint of $f$.
\end{proof}
\begin{proofOf}{\cref{fixDecomp}}
  Suppose algorithm $A$ takes at most $p(N, a)$ queries and algorithm $B$ takes at most
  $q(N, a)$ queries to find a fixpoint. Then every query of $f_r$ by algorithm $A$
  makes at most $q(N, b)$ queries to $f$ to find a fixpoint of $f_l$, so
  given algorithm $A$ makes a most $p(N, a)$ queries, the entire algorithms
  makes at most $p(N, a) \cdot q(N, b)$ queries to $f$.
\end{proofOf}

