\section{Lattices, Monotone Functions, and Fixpoints}
For the purposes of this project, I'm primarily concerned with an order-theoretic characterization of lattices.
I begin with some definitions.
\begin{definition}[Poset]\label{posetdef}
  A \emph{partially ordered set}, or \emph{poset} is a set $S$ with a binary relation $\leq$ on $S$ such that the following axioms hold,
  \begin{itemize}
    \item For all $s \in S$, $s \leq s$ (reflexivity),
    \item for all $s, t, u \in S$, if $s \leq t$ and $t \leq u$ then $s \leq u$ (transitivity),
    \item for all $s, t \in S$, if $s \leq t$ and $t \leq s$ then $s = t$ (antisymmetry).
  \end{itemize}
\end{definition}
\begin{definition}[Least/Greatest Upper/Lower Bounds]
  Let $(S, \leq)$ be a partially ordered set and $A \subseteq S$. A \emph{greatest lower bound} of $A$ is
  a $l \in S$ such that for all $a \in A$ $a \geq l$, and whenever $l' \in S$ also satisfies for all $a \in A$
  $a \geq l'$ I have $l \geq l'$. When it exists is denoted $l = \bigwedge A$. 
  A \emph{least upper bound} of $A$ is a $u \in S$ such that for all $a \in A$ $a \leq u$, and whenever $u' \in S$ also satisfies for all $a \in A$
  $a \leq u'$ I have $u \leq u'$. When it exists it is denoted $u = \bigvee A$.
\end{definition}
\begin{remark}
  Least upper bounds and greatest lower bounds need not exist in general. Take for example $(\Z, \leq)$ which
  is claimed to be a partial ordering. $\Z$ is a subset of $\Z$, but clearly $\Z$ has no upper bounds.
  Boundedness is also in general not suffiecient. 
\end{remark}
\begin{lemma}[Least/Greatest Upper/Lower bounds are unique]
  Let $L$ be a lattice, $A \subseteq L$ and $u, u'$ least upper bounds of $A$. Then $u = u'$.
  If $l, l'$ are greatest lower bounds of $A$ then $l = l'$.
\end{lemma}
\begin{proof}
  The second part of the definition on a least upper bound gives $u \leq u'$ and $u' \leq u$. Then
  antisymmetry of a partial order gives $u = u'$. That $l = l'$ follows similarly.
\end{proof}
\begin{definition}[Complete Lattice]
  Let $(L, \leq)$ be a partially ordered set. $(L, \leq)$ is a \emph{complete lattice} if for all
  $A \subseteq L$ there is a least upper bound $u = \bigvee A$ and a greatest lower bound $l = \bigwedge A$.
\end{definition}
\begin{definition}[Product Lattice]
  Given two complete lattices $(L, \leq)$ and $(L', \leq')$ the \emph{product lattice} $(L \times L, \leq \times \leq')$
  is the cartesian product of the underlying sets with $(l, l') \leq \times \leq' (m, m')$ if and only if
  $l \leq m$ and $l' \leq m'$ for all $l, m \in L$ and $l', m' \in L'$.
\end{definition}
That a finite product of complete lattices is also a complete lattice is taken as fact.
The notion of a complete sublattice will in turn be useful as well.
\begin{definition}[Sublattice]
  Let $(L, \leq)$ be a lattice. A complete sublattice $(M, \leq)$ is a
  subset $M \subseteq L$ such that for all $N \subseteq M$ I have $\bigvee N \in M$
  and $\bigwedge N \in M$.
\end{definition}
For the rest of the dissertation where the ordering is clear from context, I will denote a lattice purely by
it's underlying set.
\begin{notation}
  For $N \in \mathbb{Z}_{\geq 1}$ I use the notation $[N] = \{1, ..., N\}$.
\end{notation}
It is clear that any finite totally ordered set is a complete lattice.
\begin{cor}
  For $d \in \mathbb{Z}_{\geq 1}$ let $[N]^d = \prod_{i=1}^d [N]$. Then $[N]^d$ is a lattice. 
  The ordering is defined as $(l_1, ..., l_d) \leq (l'_1, ..., l'_d)$
  if and only if $l_i \leq l'_i$ for each $i \in \{1, ..., d\}$.
\end{cor}
At times I will also need continuous lattices. I take the result from elementary analysis
that all bounded subsets of the real numbers have a least upper bound and greatest lower bound
in the real numbers to find that $[a, b] \subseteq \R$ is a complete lattice for all $a, b \in \R$
under the usual ordering.
\begin{cor}
  For $d \in \mathbb{Z}_{\geq 1}$ let $[0, 1]^d = \prod_{i=1}^d [0, 1]$. Then $[0, 1]^d$ is a lattice. 
  The ordering is defined as $(l_1, ..., l_d) \leq (l'_1, ..., l'_d)$
  if and only if $l_i \leq l'_i$ for each $i \in \{1, ..., d\}$.
\end{cor}
The focus will be on finding so-called fixpoints of monotone functions on a lattice.
\begin{definition}[Monotone Function]
  Let $L$ be a lattice. Then a function $f : L \to L$ is \emph{monotone} if whenever $l, l' \in L$ with
  $l \leq l'$ I have $f(l) \leq f(l')$.
\end{definition}
\begin{definition}[Fixpoint]
  Let $S$ be a set and $f : S \to S$. Then $s \in S$ is a \emph{fixpoint} of $f$ if $f(s) = s$.
\end{definition}
The notion of a monotone point will also be useful.
\begin{definition}[Monotone Point]
  Let $L$ be a lattice and $f : L \to L$ be monotone. Then $x \in L$ is a \emph{monotone point}
  if either $f(x) \geq x$ or $f(x) \leq x$. It is \emph{monotone up} if $f(x) \geq x$ and \emph{monotone
  down} if $f(x) \leq x$.
\end{definition}
\begin{notation}
  Let $L$ be a complete lattice and $a, b \in L$. The notation $[a, b] = \{x \in L : a \leq x \leq b\}$ is sometimes
  used. The notations $[a, \infty) = \{x \in L : a \leq x\}$
  and $(-\infty, b] = \{x \in L : x \leq b\}$ are also used. 
  The reader can verify that these all define complete sublattices of $L$.
\end{notation}
And now for Tarski.
\begin{theorem}[Tarski's Fixed Point Theorem \citep{tarski}]
  Let $L$ be a complete lattice and $f : L \to L$ a monotone function. Define 
  $\Fix(f) = \{x \in L : f(x) = x\}$. Then $\Fix(f)$ is a complete lattice under
  the same ordering as $L$. In particular, $\Fix(f)$ contains a least fixpoint $l = \bigwedge \Fix(f)$.
\end{theorem}

\section{Matrix Games}
Basic theory of matrix games is required for \cref{shapleyChap} so is included here. Some prior knowledge on game theory is assumed.
\begin{notation}
  Let $k$ be a field and $n, m \in \zpos$. $\mathbf{Mat}(k, n \times m)$ denotes the set of all $n \times m$
  matrices with entries in $k$.
\end{notation}
\begin{definition}[Matrix Game]
  A \emph{matrix game} is a zero-sum game played by two players called the maximizer and minimizer respectively with
  strategy sets $[n]$ and $[m]$ respectively. Players choose an action $i \in [n]$ and $j \in [m]$, then receive
  payoff $A_{i, j}$ and $-A_{i, j}$ respectively where $A \in \mathbf{Mat}(\Q, n \times m)$.
  A \emph{mixed strategy} for the maximizer is a probability distribution on $[n]$. That is,
  a vector $x \in \Q^n$ such that $x_i \geq 0$ for each $i \in [n]$ and $\sum_{i} x_i = 1$. The
  set of all mixed strategies for the maximizer is denoted $X$.
  A mixed strategy for the minimizer is defined similarly as a probability distrubution $y \in \Q^n$ on $[m]$.
  The set of all mixed strategies for the minimizer is denoted $Y$.
  The \emph{expected payoff} of a matrix game $A$ under mixed strategies $x \in X$ and $y \in Y$ is defined
  to be $U(x, y) = x^T A y$ which is easily seen to be the expected payoff the maximizer receives
  when both players independently and simultaneously choose random strategies according to probability
  distributions $x$ and $y$.
\end{definition}
They key result in the theory of matrix games is the so-called minimax theorem due to Von Nuemann. A
proof can be found in \citep[Chapter 15]{matrixGamesChvatal}.
\begin{theorem}[Minimax, \citep{matrixGamesChvatal}]
  Let $A \in \mathbf{Mat}(\Q, n \times m)$ be a matrix game, $X$ and $Y$ the sets of mixed
  strategies for the maximizer and minimizer respectively. Then,
  \begin{align*}
    \max_{x \in X} \min_{y \in Y} x^T A y = \min_{y \in Y}\max_{x \in X} x^T A y.
  \end{align*}
\end{theorem}
This yields a notion of 'value' of a matrix game.
\begin{definition} \label{matrixGameVal}
  Let $A \in \mathbf{Mat}(\Q, n \times m)$ be a matrix game, $X$ and $Y$ the sets of mixed
  for the maximizer and minimizer respectively. The \emph{value} of $A$ is defined 
  $\val(A) = \max_{x \in X} \min_{y \in Y} x^T A y$.
\end{definition}
Fortunately computing the value of a matrix game is computationally not hard.
\begin{prop}
  Let $A \in \mathbf{Mat}(\Q, n \times m)$ be a matrix game. Then 
  $\val(A)$ can be computed in time polynomial with the encoding size of $A$.
\end{prop}
For details on this the reader is again referred to \citep[Chapter 15]{matrixGamesChvatal} where
it is shown the the problem can be reduced to a sufficiently small linear programming problem,
for which polynomial time algorithms are known.
