\section{Basic Algorithms}
Tarski's fixed point theorem gives rise to a natural computational problem.
\begin{definition}[$\trsk$]
  The problem $\trsk(N, d)$ is, given oracle access to a monotone function $f : [N]^d \to [N]^d$, find a point $x^* \in [N]^d$ such that $f(x^*) = x^*$.
\end{definition}
And now for the first basic algorithm for solving \trsk.
\begin{notation}
  For $k \in \znn$ the notation $\vec{k} = (k, ..., k)$. It is assumed that
  the dimensionality of this 'vector' is clear from context.
\end{notation}
\begin{algorithm}
  \caption{Kleene Tarski Iteration}
  \begin{algorithmic}[1]\label{kleeneTarski}
  \Procedure{KleeneTarski}{monotone $f : [N]^d \to [N]^d$}
  \State $x \gets \vec{1}$
  \While{$x \neq f(x)$} 
    \State $x \gets f(x)$
  \EndWhile
  \Return $x$
  \EndProcedure
  \end{algorithmic}
\end{algorithm}

Correctness of the algorithm if it terminates is clear, so all that is needed it a bound on it's runtime.
\begin{lemma}
  \textsc{KleeneTarski} always terminates in time $O(Nd)$.
\end{lemma}
\begin{proof}
  As in the proof of \cref{fixExist}, for all $i \in \znn$, $f^i(\vec{1}) \leq f^{i+1}(\vec{1})$. If $f^i(\vec{1}) = f^{i+1}(\vec{1})$
  then $f^i(\vec{1})$ is a fixpoint. So suppose for a contradiction for some $j > Nd$ that for all $i \leq j$, $f^i(\vec{1}) < f^{i+1}(\vec{1})$. 
  By integrality, $\|f^{i+1}(x)\|_1 \geq \|f^i(x)\|_1 + 1$. It follows that $\|f_j(x)\|_1 > Nd$. But this implies that
  $\|f_j(x)\|_1 > \|\vec{N}\|_1$, which is a contradiction of the definition of $f$. So for some $j \leq Nd$, $f^j(\vec{1}) = f^{j+1}(\vec{1})$.
\end{proof}
\begin{theorem}
  The query complexity of $\trsk(N,d)$ is $O(Nd)$.
\end{theorem}
The fixpoint returned by $\textsc{KleeneTarski}$ isn't just any old fixpoint.
\begin{lemma}\label{kleeneLfp}
  Let $x$ be the fixpoint returned by $\textsc{KleeneTarski}$. Then $x$ is the least-fixpoint
  of $f$. That is, for all other fixpoints $y \in [N]^d$, $y \leq x$.
\end{lemma}
\begin{proof}
  Let $(f^i(\vec{1}))_{i \in \znn}$ be the sequence of points generated in $\textsc{KleeneTarski}$.
  I will show inductively that $f^i(\vec{1}) \leq y$ for all $i \in \znn$. For the base case,
  by construction of the lattice $f^0(\vec{1}) = \vec{1} \leq y$. Suppose $f^{i-1}(\vec{1}) \leq y$.
  Then by monotonicity of $f$, $f(f^{i-1}(\vec{1})) = f^i(\vec{1}) \leq f(y)$. But $y$ is a fixpoint
  so $f(y) = y$ and $f^i(\vec{1}) \leq y$.
\end{proof}
It should be emphasized that this is algorithm does \emph{not} run in time polynomial with the encoding size
of problem instances. For numbers of size $2^n$ can be represented with a $n$ bits of information. 
At the time of writing
it is not known whether or not this problem is solvable in polynomial time.
Etessami et al. gave the current best known lower bound on the query complexity of $\textsc{Tarski}$, along with other complexity-theoretic results
on the problem.
\begin{theorem}[\citep{lowerBound}]
  The query complexity of $\trsk(N, d)$ is $\Omega(\log^2N)$.
\end{theorem}
Dang, Qi, and Ye gave an algorithm for solving the $\trsk$ problem\citep{dangQiYe} using a variant of the well
known binary search algorithm. The details of their algorithm are instructive
to the workings on the improved algorithms detailed  later, so they are given below.
\begin{notation}
  Given a tuple $x = (x_1, ..., x_n)$ for $i \in [n]$ the notation $x_{-i} = (x_1, ..., x_{i-1}, x_{i+1}, ..., x_n)$. That is, it drops
  the $i$-th coordinate of the tuple.
\end{notation}
\begin{definition}[Slice] \label{sliceDef}
  Let $(S_i)_{i \in [d]}$ be totally ordered sets, $L = \prod_{i \in [d]} S_i$
  be their product lattice, and $f : L \to L$ be monotone. 
  Then a \emph{slice} of $f$ is a choice of coordinate $i \in [d]$,
  and a choice of value $x_i \in S_i$, defining a new function 
  $f_s : L_{-i} \to L_{-i}$ with
  $f_s((l_1, ..., l_{d-1})) = f((l_1, ..., l_{i-1},  x_i, l_i, ..., l_{d-1}))_{-i}$.
\end{definition}
\begin{lemma} \label{sliceMonotone}
  Let $f : \nd \to \nd$ be monotone. Then for any $i \in [d]$, $x_i \in [N]$ the slice $f_s : [N]^{d-1} \to [N]^{d-1}$ at $i$ with value $x_i$  
  is monotone. 
\end{lemma}
\begin{proof}
  Suppose $l, l' \in [N]^{d-1}$ with $l = (l_1, ..., l_{d-1})$, $l' = (l'_1, ..., l'_{d-1})$, and $l \leq l'$.
  By reflexivity, $x_i \leq x_i$, so $(l_1, ... , x_i, l_i, ..., l_{d-1}) \leq (l'_1, ... , x_i, l'_i, ..., l_{d-1})$,
  and $f_s(l) \leq f_s(l')$ follows by monotonicity of $f$.
\end{proof}
\begin{lemma}\label{restricts}
  Let $f : [N]^d \to [N]^d$ is monotone, and $x \in [N]^d$ be such that $x \leq f(x)$. Then
  $f$ restricts to a monotone function $\restr{f}{[x, \infty)} : [x, \infty) \to [x, \infty)$. Similarly,
  if $x \geq f(x)$ then $f$ restricts to a monotone function $\restr{f}{(-\infty, x]} : (-\infty, x] \to (-\infty, x]$.
\end{lemma}
\begin{proof}
  I need to show that if $x \leq f(x)$ then for all $y \in [x, \infty)$, $f(y) \in [x, \infty)$. By construction,
  $y \geq x$, and by monotonicity $f(y) \geq f(x)$. But $f(x) \geq x$, so $f(y) \geq x$, and $f(y) \in [x, \infty)$. The second part
  is similar.
\end{proof}
\begin{lemma}\label{d1Case}
  Let $f : [N] \to [N]$ be monotone. Then a fixpoint of $f$ can be found in $O(\log N)$ queries of $f$.
\end{lemma}
\begin{proof}
  Choose $x = \lfloor \frac{N}{2} \rfloor$. $[N]$ is totally ordered, so exactly one of the following hold; 
  $x < f(x)$, $x = f(x)$, $x > f(x)$.
  If $x = f(x)$ then I'm done. If $x < f(x)$ then by \cref{restricts} $f$ restricts to a monotone function $\restr{f}{[x, \infty)}$, 
  and a fixpoint of $\restr{f}{[x, \infty)}$ is clearly also a fixpoint of $f$. Similarly, if $x > f(x)$ then $f$ restricts to
  $\restr{f}{(\infty, x]}$. This enables a recursion on the smaller sublattice. Finally,
  noting that a fixpoint can be found trivially in the one-point set in a constant number of queries,
  since the search space is halved every recursive call
  the algorithm terminates in $O(\log N)$ queries.
\end{proof}
The algorithm of Dang, Qi, and Ye can be seen in \cref{dQiYiAlg}.

\begin{algorithm}[h]
  \caption{\citep{dangQiYe}}\label{dQiYiAlg}
  \begin{algorithmic}[1]
  \Procedure{DangQiYe}{monotone $f : [N]^d \to [N]^d$}
    \State $\bot \gets \vec{1}$
    \State $\top \gets \vec{N}$
    \State \Return \Call{DangQiYeRec}{$f$, $\bot$, $\top$}
  \EndProcedure
  \Procedure{DangQiYeRec}{monotone $f : [N]^d \to [N]^d$, $\bot$, $\top$}
  \While{true}
    \State $\mi_d \gets \lfloor \frac{\bot_d + \top_d}{2} \rfloor$
    \State $f_s \gets$ the slice of $f$ at $d$ with value $\mi_d$
    \State $\vec{x_s} \gets$ \Call{DangQiYe}{$f_s$, $\bot_{-d}$, $\top_{-d}$}
    \State $\vec{x} \gets ((\vec{x_s})_1, ..., (\vec{x_s})_{d-1}, \mi_d)$
    \If{$\vec{x}_d = f(\vec{x})_d$}
      \State \Return{$\vec{x}$}
    \EndIf
    \If{$\mi_d < f(\vec{x})_d$}
      \State $\bot \gets \vec{x}$
    \EndIf
    \If{$\mi_d > f(\vec{x})_d$}
      \State $\top \gets \vec{x}$
    \EndIf
  \EndWhile
  \EndProcedure
  \end{algorithmic}
\end{algorithm}

\begin{lemma}
  \textsc{DangQiYe} returns a fixpoint of $f$ if it terminates.
\end{lemma}
\begin{proof}
  The algorithm only returns if it satisfies the condition on line 11. At this point, $\vec{x_s}$ is a fixpoint
  of $f_s$, so it follows that $f(\vec{x})_i = \vec{x}_i$ for $i \in [d-1]$. The condition ensures
  that also $\vec{x}_d = f(\vec{x})_d$, and $\vec{x}$ is a fixpoint of $f$.
\end{proof}
\begin{lemma}
  \textsc{DangQiYe} terminates in at most $O(\log^d N)$ queries to $f$.
\end{lemma}
\begin{proof}
  By induction. The base case follows from \cref{d1Case}. Suppose \textsc{DangQiYe} uses at most
  $O(\log^{d-1}N)$ queries to solve the $d-1$ dimensional case. If the condition
  on line 11 fails, note that $\vec{x}$ is a monotone point, so \cref{restricts} guarantees
  that the function restricts to the smaller lattice bounded by lines 14 or 16.
  The $d$-th dimension is shrunk by a factor of $\frac{1}{2}$ every iteration of the loop,
  so will require at most $\log N$ recursive calls to the $d-1$ dimensional algorithm.
  So the algorithm terminates using at most $O(\log N) \cdot O(\log^{d-1} N) = O(\log^d N)$ queries to $f$.
\end{proof}
\begin{theorem}[\citep{dangQiYe}]
  The query complexity of $\trsk(N, d)$ is $O(\log^d N)$.
\end{theorem}
Simply combining the theorems of Dang, Qi, and Ye, and Etessami et al. gives a tight
bound on the 2 dimensional problem.
\begin{cor}[\citep{lowerBound}]
  The query complexity of $\trsk(N, 2)$ is $\Theta(\log^2 N)$.
\end{cor}
There have recently been improvements on this upper bound. In paricular,
Fearnley et al. gave an algorithm for solving the 3 dimensional version
of the problem. This used the crucial fact that, in contrast to the
Dang, Qi, and Ye algorithm, restricting to a half-space does \emph{not} require
a point which is fixed in $n-1$ dimensions, but a point which is monotone
in all $n$ dimensions. They gave a so-called 'inner algorithm'
which in three dimensions can find a monotone point with a particular
value in one coordinate in $O(\log N)$ queries. Chen and Li showed
how to decompose the problem of finding monotone
points in higher dimensions - crucially using the inner algorithm - to a product of sorts of 2 dimensional
prolems, giving the current best known upper bound on the $\trsk$ problem.

\begin{theorem}[\citep{fasterTarski}]
  The query complexity of $\trsk(N, 3)$ is $\Theta(\log^2 N)$.
\end{theorem}
\begin{theorem}[\citep{chenLi}]
  The query complexity of $\trsk(N, d)$ is $O(\log^{\lceil (d+1)/2 \rceil} N)$.
\end{theorem}
The details of these algorithms will be shared in a later section.
\begin{remark}
  The problem of computing any fixpoint of a monotone function on a general lattice
  is known to be intractable. In particular, it was shown in \citep{changComplexity}
  that for any randomized algorithm there are general lattices of $N$ elements 
  which require $\Omega(N)$ queries with high probability to find a fixpoint.
  Focussing on a specific case such as $[N]^d$ is therefore justified.
  Further, the case of computing least fixpoints of even one dimensional
  functions $f : [2^n] \to [2^n]$ is proven to be NP-hard in \citep{lowerBound}
  so would seem that focussing on find any fixpoint is the best course of action.
\end{remark}
