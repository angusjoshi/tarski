\chapter{Background}

\section{Lattices, Monotone Functions, and Fixpoints}
A definition seems like a good place to start.
\begin{definition}[Lattice]\label{latedef}
  A \emph{lattice} is a set $L$, and two binary operators $\wedge, \vee : L \times L \to L$ called \emph{join} and \emph{meet} respectively,
  such that the following axioms hold. For all $a, b, c \in L$,
  \begin{itemize}
    \item $a \wedge (a \vee b) = a$ and $a \vee (a \wedge b) = b$ (absorption),
    \item $a \wedge b = b \wedge a$ and $a \vee b = b \vee a$ (commutativity),
    \item $a \wedge (b \wedge c) = (a \wedge b) \wedge c$ and $a \vee (b \vee c) = (a \vee b) \vee a$ (associativity).
  \end{itemize}
\end{definition}
For the purposes of this project, I'm primarily concerned with an order-theoretic characterization of lattices. 
\begin{definition}[Poset]\label{posetdef}
  A \emph{partially ordered set}, or \emph{poset} is a set $S$ with a binary relation $\leq$ on $S$ such that the following axioms hold,
  \begin{itemize}
    \item For all $s \in S$, $s \leq s$ (reflexivity),
    \item for all $s, t, u \in S$, if $s \leq t$ and $t \leq u$ then $s \leq u$ (transitivity),
    \item for all $s, t \in S$, if $s \leq t$ and $t \leq s$ then $s = t$ (antisymmetry).
  \end{itemize}
\end{definition}
As it turns out, any lattice structure induces a partial order on it's underlying set.
\begin{lemma}\label{joinMeetIdempotent}
  Let $(L, \wedge, \vee)$ be a lattice. Then for all $l \in L$, $l \wedge l = l$ and $l \vee l = l$.
\end{lemma}
\begin{proof}
  Let $l \in L$. Then by two applications of the absorption laws, $l = l \vee (l \wedge (l \vee l)) = l \wedge l$. $l \vee l = l$ follows
  by duality.
\end{proof}
\begin{prop}\label{latOrd}
  Let $(L, \wedge, \vee)$ be a lattice. Then for all $l, l' \in L$, the binary relation defined by $l \leq l'$ if and only if $l \vee l' = l'$
  is a partial order on $L$.
\end{prop}
\begin{proof}
  Reflexivity follows immediately by \cref{joinMeetIdempotent}. For transitivity, let $l, m, n \in L$ and suppose
  $l \vee m = m$ and $m \vee n = n$. Then using associativity of $\vee$,
  \begin{align*}
    l \vee n = l \vee (m \vee n) = (l \vee m) \vee n = m \vee n = n.
  \end{align*}
  For antisymmetry, suppose $l \vee m = m$ and $m \vee l = l$. By commutativity of $\vee$, 
  \begin{align*}
    l = m \vee l = l \vee m = m.
  \end{align*}
\end{proof}
\begin{remark}
  Although \cref{latOrd} only uses half of the structure of the lattice to define a partial order,
  there is a dual definition where $l \leq l'$ if and only if $l \wedge l' = l$, and the two are equivalent by the absorption laws.
\end{remark}
The partial ordering induced by a lattice is in fact a very special case of a general partial order.
\begin{definition}[Least/Greatest Lower Bounds]
  Let $S$ be a set with partial ordering $\leq$. For $s, s' \in S$, $t$ is the \emph{greatest lower bound} of $s$ and $s'$
  if $s \geq t$, $s' \geq t$ and whenever $t' \in S$ satisfies $s \geq t'$ and $s' \geq t'$ I have $t \geq t'$.
  The \emph{least upper bound} is dual to the greatest lower bound.
\end{definition}
\begin{definition}[Lattice Ordering]
  Let $L$ be a set. A \emph{lattice ordering} on $L$ is a partial ordering $\leq$ on $L$ such that for all $l, l' \in L$, $l$ and $l'$
  have a unique greatest lower bound, and least upper bound.
\end{definition}
\begin{prop}
  Let $(L, \wedge, \vee)$ be a lattice. Then the binary relation $\leq$ on $L$ defined by $l \leq l'$ if and only if
  $l \vee l' = l'$ is a lattice ordering on $L$.
\end{prop}
\begin{proof}
  By \cref{latOrd}, $\leq$ is a partial order on $L$. So let $l, l' \in L$. I will show that the least upper bound of
  $l$ and $l'$ is precisely $l \vee l'$. Firstly, $l \vee (l \vee l') = (l \vee l) \vee l' = l \vee l'$, and $l \leq (l \vee l')$.
  $l' \leq (l \vee l')$ follows similarly. Now suppose $m \in L$ satisfies $l \vee m = m$ and $l' \vee m = m$. Then,
  \begin{align*}
    (l \vee l') \vee m = l \vee (l' \vee m) = l \vee m = m,
  \end{align*}
  and $l \vee l' \leq m$. Finally for uniqueness, if $m, m'$ are least upper bounds of $l$ and $l'$, then $m \leq m'$ and $m' \leq m$,
  so by antisymmetry, $m = m'$. That $l \wedge l'$ is the greatest lower bound of $l$ and $l'$ follows by duality.
\end{proof}
\begin{definition}[Product Lattice]
  Given two lattices $\mathcal{L} = (L, \wedge, \vee)$ and $\mathcal{L'} = (L', \wedge', \vee')$ the \emph{product lattice} 
  is $\mathcal{L} \times \mathcal{L'} = (L \times L', \wedge \times \wedge', \vee \times \vee')$ where for $(l, l'), (m, m') \in \mathcal{L} \times \mathcal{L'}$
  the product meet is defined $(l, l') (\wedge \times \wedge') (m, m') = ((l \wedge m), (l' \wedge' m'))$. Join is defined similarly.
\end{definition}
\begin{prop}
  Let $\mathcal{L} = (L, \wedge, \vee)$ and $\mathcal{L'} = (L', \wedge', \vee')$. The product lattice $\mathcal{L} \times \mathcal{L'}$ is a lattice.
\end{prop}
\begin{proof}
  Let $(l, l'), (m, m'), (n, n') \in L \times L'$. For absorption, 
  \begin{align*}
    (l, l') (\vee \times \vee') ((l, l') (\wedge \times \wedge') (m, m')) &= (l \vee (l \wedge m), l' \vee (l' \wedge m')) \\
                                                                          &= (l, l').
  \end{align*}
  The other absorption law, commutativity, and associativity follow similarly.
\end{proof}
\begin{definition}[Total Order]
  A partially ordered set $(S, \leq)$ is \emph{totally ordered} if whenever $a, b \in S$ at least one of $a \leq b$ or $b \leq a$ hold.
  This gives binary operators $\max$ and $\min$ on $S$, where $\max (a, b) = \begin{cases} b, & a \leq b \\ a, & \text{otherwise,}  \end{cases}$
    $\min (a, b) = \begin{cases} a, & a \leq b \\ b, & \text{otherwise.}  \end{cases}$
\end{definition}
\begin{prop}
  Let $(S, \leq)$ be a total order. Then $(S, \min, \max)$ is a lattice.
\end{prop}
\begin{proof}
  For absorption, let $a, b, c \in S$. Then 
  \begin{align*}
    \min(a, \max(a, b)) = \begin{cases} \min(a, b), & a \leq b \\ \min(a, a), & \text{otherwise}\end{cases} = a.
  \end{align*}
  The other absorption law follows similarly. Commutativity is clear, and for associativity of $\min$,
  \begin{align*}
    \min(a, \min(b, c)) = \begin{cases} \min(a, b), &  b \leq c \\ \min(a, c), & \text{otherwise}\end{cases} = \min(\min(a, b), c).
  \end{align*}
  Associativity of $\max$ follows similarly.
\end{proof}
For the rest of the dissertation where join, meet, and $\leq$ are clear from context, I will denote a lattice purely by it's underlying set.
\begin{notation}
  For $N \in \mathbb{Z}_{\geq 1}$ I use the notation $[N] = \{1, ..., N\}$.
\end{notation}
It's clear that $[N]$ is totally ordered with the standard ordering on $\mathbb{Z}$.
\begin{cor}
  For $d \in \mathbb{Z}_{\geq 1}$ let $[N]^d = \prod_{i=1}^d [N]$. Then $[N]^d$ is a lattice. Join and meet
  are given by coordinate-wise $\max$ and $\min$ respectively, and the lattice ordering is defined as $(l_1, ..., l_d) \leq (l'_1, ..., l'_d)$
  if and only if $l_i \leq l'_i$ for each $i \in \{1, ..., d\}$.
\end{cor}
The focus will be on finding so-called fixopints of monotone functions on a lattice.
\begin{definition}[Monotone Function]
  Let $L$ be a lattice. Then a function $f : L \to L$ is \emph{monotone} if whenever $l, l' \in L$ with
  $l \leq l'$ I have $f(l) \leq f(l')$.
\end{definition}
\begin{definition}[Fixpoint]
  Let $S$ be a set and $f : S \to S$. Then $s \in S$ is a \emph{fixpoint} of $f$ if $f(s) = s$.
\end{definition}
\newpage

\section{Fixpoint Existence and Computation}
Tarski gave a theorem on the existence of fixpoints of monotone functions on complete lattices\citep{tarski} (which I have not defined). This theorem
is in reality stronger than is needed for the finite case. I present a proof of the existence of fixpoints in a finite lattice.
\begin{theorem}[\citep{tarski}]\label{fixExist}
  Let $f : [N]^d \to [N]^d$ be monotone. Then there is a point $x^* \in [N]^d$ such that $f(x^*) = x^*$.
\end{theorem}
\begin{proof}
  Firstly, note that for all $x \in [N]^d$ the point $\vec{1} = (1, ..., 1) \leq x$, and in particular $\vec{1} \leq f(\vec{1})$.
  By an induction combined with monotonicity I find for all $i \in \znn$, $f^i (\vec{1}) \leq f^{i+1} (\vec{1})$. Suppose for a contradiction that there
  is no point $x^* \in [N]^d$ such that $f(x^*) = x^*$. Then for all $i \in \znn$, $f^i (\vec{1}) \neq f^{i+1}(\vec{1})$ which implies
  that $f^i (\vec{1}) < f^{i+1}(\vec{1})$. This gives infinitely many distinct points in $[N]^d$. But $[N]^d$ is finite, which is a contradiction.
  It follows that
  there is a fixpoint of $f$ in $[N]^d$.
\end{proof}
This gives rise to a natural problem; how can such a fixpoint be found?
\begin{definition}[$\trsk$]
  The problem $\trsk(N, d)$ is, given oracle access to a monotone function $f : [N]^d \to [N]^d$, find a point $x^* \in [N]^d$ such that $f(x^*) = x^*$.
\end{definition}
The proof of \cref{fixExist} implicitly contains our first algorithm for fixpoint computation.
\begin{algorithm}
  \caption{Kleene Tarski Iteration}
  \begin{algorithmic}[1]
  \Procedure{KleeneTarski}{monotone $f : [N]^d \to [N]^d$}
  \State $x \gets \vec{1}$
  \While{$x \neq f(x)$} 
    \State $x \gets f(x)$
  \EndWhile
  \Return $x$
  \EndProcedure
  \end{algorithmic}
\end{algorithm}

Correctness of the algorithm if it terminates is clear, so all that is needed it a bound on it's runtime.
\begin{lemma}
  \textsc{KleeneTarski} always terminates in time $O(Nd)$.
\end{lemma}
\begin{proof}
  As in the proof of \cref{fixExist}, for all $i \in \znn$, $f^i(\vec{1}) \leq f^{i+1}(\vec{1})$. If $f^i(\vec{1}) = f^{i+1}(\vec{1})$
  then $f^i(\vec{1})$ is a fixpoint. So suppose for a contradiction for some $j > Nd$ that for all $i \leq j$, $f^i(\vec{1}) < f^{i+1}(\vec{1})$. 
  By integrality, $\|f^{i+1}(x)\|_1 \geq \|f^i(x)\|_1 + 1$. It follows that $\|f_j(x)\|_1 > Nd$. But this implies that
  $\|f_j(x)\|_1 > \|\vec{N}\|_1$, which is a contradiction of the definition of $f$. So for some $j \leq Nd$, $f^j(\vec{1}) = f^{j+1}(\vec{1})$.
\end{proof}
\begin{theorem}
  The query complexity of $\trsk(N,d)$ is $O(Nd)$.
\end{theorem}
It should be emphasized that this is \emph{not} a polynomial-time algorithm for solving the $\trsk$ problem, as a number
$N$ can be represented with $\log n$ bits of information. 
Etessami et al. gave the current best known lower bound on the query complexity of $\textsc{Tarski}$, along with other complexity-theoretic results
on the problem.
\begin{theorem}[\citep{lowerBound}]
  The query complexity of $\trsk(N, d)$ is $\Omega(\log^2N)$.
\end{theorem}
Dang, Qi, and Ye gave an algorithm for solving the $\trsk$ problem\citep{dangQiYe} using a variant of the well
known binary search algorithm. The details of their algorithm are instructive
to the workings on the improved algorithms detailed  later, so they are given below.
\begin{notation}
  Given a tuple $x = (x_1, ..., x_n)$ for $i \in [n]$ the notation $x_{-i} = (x_1, ..., x_{i-1}, x_{i+1}, ..., x_n)$. That is, it drops
  the $i$-th coordinate of the tuple.
\end{notation}
\begin{definition}[Slice]
  Let $f : \nd \to \nd$ be monotone. Then a \emph{slice} of $f$ is a choice of coordinate $i \in [d]$,
  and a choice of value $x_i \in [N]$, defining a new function $f_s : [N]^{d-1} \to [N]^{d-1}$ with
  $f_s((l_1, ..., l_{d-1})) = f((l_1, ..., l_{i-1},  x_i, l_i, ..., l_{d-1}))_{-i}$.
\end{definition}
\begin{lemma}
  Let $f : \nd \to \nd$ be monotone. Then for any $i \in [d]$, $x_i \in [N]$ the slice $f_s : [N]^{d-1} \to [N]^{d-1}$ at $i$ with value $x_i$  
  is monotone. 
\end{lemma}
\begin{proof}
  Suppose $l, l' \in [N]^{d-1}$ with $l = (l_1, ..., l_{d-1})$, $l' = (l'_1, ..., l'_{d-1})$, and $l \leq l'$.
  By reflexivity, $x_i \leq x_i$, so $(l_1, ... , x_i, l_i, ..., l_{d-1}) \leq (l'_1, ... , x_i, l'_i, ..., l_{d-1})$,
  and $f_s(l) \leq f_s(l')$ follows by monotonicity of $f$.
\end{proof}
\begin{notation}
  Let $L$ be a lattice. Then for $x \in L$ the notation $L_{\geq x}$ is a sub-lattice of $L$ with underlying set
  $\{l \in L | l \geq x\}$. It is clear that $L_{\geq x}$ is also a lattice, with the same joins, meets, and ordering as $L$.
\end{notation}
\begin{lemma}\label{restricts}
  Let $f : [N]^d \to [N]^d$ is monotone, and $x \in [N]^d$ be such that $x \leq f(x)$. Then
  $f$ restricts to a monotone function $\restr{f}{[N]^d_{\geq x}} : [N]^d_{\geq x} \to [N]^d_{\geq x}$. Similarly,
  if $x \geq f(x)$ then $f$ restricts to a monotone function $\restr{f}{[N]^d_{\leq x}} : [N]^d_{\leq x} \to [N]^d_{\leq x}$.
\end{lemma}
\begin{proof}
  I need to show that if $x \leq f(x)$ then for all $y \in [N]^d_{\geq x}$, $f(y) \in [N]^d_{\geq x}$. By construction,
  $y \geq x$, and by monotonicity $f(y) \geq f(x)$. But $f(x) \geq x$, so $f(y) \geq x$, and $f(y) \in [N]^d_{\geq x}$. The second part
  follows by duality.
\end{proof}
\begin{lemma}
  Let $f : [N] \to [N]$ be monotone. Then a fixpoint of $f$ can be found in $O(\log N)$ queries of $f$.
\end{lemma}
\begin{proof}
  Choose $x = \lfloor \frac{N}{2} \rfloor$. $[N]$ is totally ordered, so exactly one of the following hold; $x < f(x)$, $x = f(x)$, $x > f(x)$.
  If $x = f(x)$ then I'm done. If $x < f(x)$ then by \cref{restricts} $f$ restricts to a monotone function $\restr{f}{[N]^d_{\geq x}}$, 
  and a fixpoint of $\restr{f}{[N]^d_{\geq x}}$ is clearly also a fixpoint of $f$. Similarly, if $x > f(x)$ then $f$ restricts to
  $\restr{f}{[N]^d_{\leq x}}$. This enables a recursion on the smaller sublattice. Finally,
  noting that a fixpoint can be found trivially in the one-point set in a constant number of queries,
  since the search space is halved every recursive call
  the algorithm terminates in $O(\log N)$ queries.
\end{proof}
The algorithm of Dang, Qi, and Ye using these facts is given below.

\begin{algorithm}[t]
  \caption{Dang, Qi, Ye}
  \begin{algorithmic}[1]
  \Procedure{DangQiYe}{monotone $f : [N]^d \to [N]^d$}
  \State $l \gets 1$
  \State $h \gets N$
  \While{true}
    \State $x \gets \lfloor \frac{l + h}{2} \rfloor$
    \State $f_s \gets$ the slice of $f$ at $d$ with value $x$
    \State $\vec{x_s} \gets$ \Call{DangQiYe}{$f_s$}
    \State $\vec{x} \gets ((\vec{x_s})_1, ..., (\vec{x_s})_{d-1}, x)$
    \If{$\vec{x}_d = f(\vec{x})_d$}
      \State \Return{$\vec{x}$}
    \EndIf
    \If{$\vec{x}_d < f(\vec{x})_d$}
      \State $l \gets x$
    \EndIf
    \If{$\vec{x}_d > f(\vec{x})_d$}
      \State $h \gets x$
    \EndIf
  \EndWhile
  \EndProcedure
  \end{algorithmic}
\end{algorithm}

\begin{lemma}
  \textsc{DangQiYe} returns a fixpoint of $f$ if it terminates.
\end{lemma}
\begin{proof}
  The algorithm only returns if it satisfies the condition on line 9. At this point, $\vec{x_s}$ is a fixpoint
  of $f_s$, so it follows that $f(\vec{x})_i = \vec{x}_i$ for $i \in [d-1]$. The condition ensures
  that also $\vec{x}_d = f(\vec{x})_d$, and $\vec{x}$ is a fixpoint of $f$.
\end{proof}
\begin{lemma}
  \textsc{DangQiYe} termintes in at most $O(\log^d N)$ queries to $f$.
\end{lemma}
\begin{proof}
\end{proof}
\newpage
\section{The Arrival Problem}
asdfsdaf
