\chapter{Introduction}
\section{Overview}
Fixpoint theorems have proven to be fundamental in the study of equilibria in
game theory and economic theory more generally. One such theorem due to Alfred Tarski in \citep{tarski}
proves the existence of fixpoints of monotone functions on complete lattices, and
many problems in algorithmic game theory reduce to computing such a fixpoint\citep{lowerBound}. Recent years have seen
a flurry of results in the computational complexity of tarski fixed-point computation \citep{dangQiYe, lowerBound, fasterTarski, chenLi},
though many problems remain open. \\
In this dissertation I discuss a variant of tarki fixed-point computation I call the $\trsk$ problem, three
related problems in algorithmic game theory which reduce to $\trsk$, and the state of the art in upper bounds for $\trsk$.
The currently best-known algorithms are also implemented and tested on random instances of the three described problems, reaching
the conclusion that the recent improvements in asymptotic upper bounds do not translate to practical performance benefit in the cases
tested here.
\section{Contributions}
The main contributions of this dissertation are as follows,
\begin{itemize}
  \item an exposition of the theoretical development on upper bounds for the $\trsk$ problem, and reduction from related problems is given with highlights including,
    proofs of the main lemmas for the so-called inner algorithm in \citep{fasterTarski} are proven in a simplified context in \cref{innerAlgChap}, and a 
    complete reduction from the $\arr$ problem to $\trsk$ is given in \cref{arr},
  \item a minor error in \citep[Proposition 6.1.]{lowerBound} is pointed out, with suggested rectification included in \cref{shapleyDiscretization},
  \item a \href{https://www.github.com/angusjoshi/tarski}{novel implementation} of the recently developed complex algorithms for the $\trsk$ problem is tested
    on randomly generated instances of $\arr$, simple stochastic games, and shapley's stochastic games, giving
    evidence that the improved asymptotic upper bounds do not result in a practical performance increase.
\end{itemize}
