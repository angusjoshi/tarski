\section{Simple Stochastic Games}\label{ssgs}
Simple stochastic games, as defined in \citep{condon}, are a class of
zero-sum games played on graphs with two players called the maximizer and minimizer
respectively. For the purposes of this dissertation I will consider only $\beta$-stopping
simple stochastic games which roughly speaking are simple stochastic games
where at every stage the game automatically halts with probability $\beta$.
Condon shows in \citep{condon} that these games necessarily have a rational value for rationally described instances
of the problem, and further that
the value can be achieved in pure stationary strategies which is to say
that players can achieve optimal expected payoff by fixing a deterministic
action for every one of their vertices prior to play commencing.
The relationship to $\trsk$ then comes from \citep{lowerBound}, where Etessami et al. show
that computing the \emph{exact} value of (not necessarily $\beta$-stopping) simple stochastic games as well as a pure stationary strategy profile
to achieve this value is 
polynomial-time reducible to $\trsk$. This section will lay out the required definitions,
and describe the aforementioned reduction to $\trsk$ in the special case of 
$\beta$-stopping simple stochastic games.

\newcommand{\vmax}{V_{\max}}
\newcommand{\vmin}{V_{\min}}
\begin{definition}[$\beta$-stopping Simple Stochastic Game]
  A \emph{$\beta$-stopping simple stochastic game} is a directed graph $G = (V, V_p, \vmax, \vmin, E, v_0, t, \beta)$ with designated start vertex $v_0 \in V$,
  target vertex $t \in V$, $\beta \in (0, 1] \cap \Q$,
  a partition of $V \setminus \{t\}$ into three disjoint subsets $V_p, V_{\min}, V_{\max}$,
  and a mapping $p : V_p \times V \to [0, 1] \cap \Q$
  such that for all $v_p \in V_p$,
  $v \in V$ if $(v_p, v) \not\in E$ I have $p(v_p, v) = 0$, for each
  $v_p \in V_p$ $\sum_{v \in V} p(v_p, v) = 1$, and for every $v \in V \setminus \{t\}$ there
  is necessarily an edge $(v, w) \in E$ for some $w \in V$. A \emph{play} in a simple stochastic game
  transpires as follows. A token is placed on the initial vertex of the game $v_0$.
  Let $v_i$ be the vertex on which the token currently lies. Then at each step, the game halts with probability $\beta$. If it did not halt and $v_i \in \vmax \; (\vmin)$
  then the maximizer (minimzer) chooses and edge $(v_i, v_{i + 1}) \in E$ for some $v_{i + 1} \in V$. If
  $v_i \in V_p$ then an edge $(v_i, w) \in E$ is chosen randomly with probability $p(v_i, w)$. If $v_i = t$
  then the game halts. 
  When the game halts, the maximizer gets a payoff of $1$ if the game reached
  $t$, and $0$ otherwise. The payoff of the minimizer is the negative of the maximizer's.
\end{definition}

\begin{figure}[h]
  \centering
  \tikzset{every picture/.style={line width=0.75pt}} %set default line width to 0.75pt        
\begin{tikzpicture}[x=0.6pt,y=0.6pt,yscale=-1,xscale=1,
  main/.style = {draw, circle, 
  fill={rgb, 255:red, 0; green, 0; blue, 0 }, text=white},
  node distance=3cm,
  ->,
  >={Stealth[round,sep]},
  legend/.style={draw, circle,
  fill={rgb, 255:red, 0; green, 0; blue, 0 },
  text=black
  }
  ]

  \node[main] (v0) {$v_0$};
  \node[main, shape=diamond] (v1) [right of=v0] {$v_1$};
  \node[main, shape=rectangle] (v2) [below right of=v0] {$v_2$};
  \node[main] (v3) [right of=v1] {$v_3$};
  \node[main, shape=star] (t) [below right of=v3] {$t$};

  \draw (v0) -- (v1) node[near start, below right, text=black]{$\frac{1}{3}$};
  \draw (v0) -- (v2) node[near start, below, text=black]{$\frac{2}{3}$};

  \draw (v3) -- (t) node[near start, below left, text=black]{$\frac{2}{5}$};
  \draw (v3) -- (v2) node[near start, below right, text=black]{$\frac{2}{5}$};
  \draw (v3) to[bend left=45, min distance=1cm] node[near start, above right, text=black]{$\frac{1}{5}$} (v0);

  \draw (v1) -- (v3);
  \draw (v1) -- (v2); 

  \draw (v2) edge[bend right=60, min distance=1.5cm] (v0);
  \draw (v2) -- (t);

  \matrix [draw] at (current bounding box.north east) {
    \node[legend, label=right:\scriptsize Chance] {}; \\
    \node[legend, shape=diamond,label=right:\scriptsize Maximizer] {}; \\
    \node[legend, shape=rectangle,label=right:\scriptsize Minimizer] {}; \\
  };

\end{tikzpicture}

  \caption{In $\beta$-stopping simple stochastic games, one of the two players aims to maximize the probability
  of play reaching the target $t$, while the other aims to minimize it.}
\end{figure}

It is clear that since $\beta > 0$ the game eventually halts with probability $1$. 
\begin{definition}[Pure Stationary Strategy]
  Let $G = (V, V_p, \vmax, \vmin, E, v_0, t)$ be a simple stochastic game. A \emph{pure stationary strategy}
  for the maximizer is a mapping $\sigma : \vmax \to V$ with the requirement that for all $v \in \vmax$
  $(v, \sigma(v)) \in E$. The set of all such pure stationary strategies for the maximizer is denoted
  $S$. A pure stationary strategy for the minimizer is a map $\tau : \vmin \to V$ such that
  for all $v \in \vmin$ 
  $(v, \tau(v)) \in E$. The set of all such pure stationary strategies for the minimizer is denoted $T$. 
  A \emph{pure stationary strategy profile} is a pair $(\sigma, \tau) \in S \times T$.
\end{definition}
Once you fix a pure stationary strategy profile $(\sigma, \tau)$
in a simple stochastic game, the resultant process is easily seen
to be a discrete markov chain with two absorbing states corresponding to
the to the game reaching $t$ or halting elsewhere. The probability
of reaching $t$ from any particular vertex can then be computed 
by solving a system of linear equations, leading to our definition of the expected value under 
a fixed pure stationary strategy profile.
\begin{definition}[Expected Value]
  Let $G = (V, V_p, \vmax, \vmin, E, v_0, t)$ be a simple stochastic game.
  The \emph{expected value} of
  a particular vertex $i \in V$ under the pure strategy profile 
  $(\sigma, \tau)$ written $v_{\sigma, \tau}(i)$ is the probability
  of absorption at $t$ in the resulting markov chain after fixing actions
  according to $\sigma$ and $\tau$.
  The \emph{game expected value} is the value of the start vertex $v_{\sigma, \tau}(v_0)$.
\end{definition}
The content of the following result is that simple stochastic games have a well defined
notion of value that, importantly, can be achieved in \emph{pure stationary strategies}.
This is to say that both players can achieve optimal expected payoff in the game
by fixing a deterministic action at every node in the game prior to it starting.
\begin{theorem}[\citep{condon}, lemma 6] \label{ssgHasValue}
  Let $G = (V, V_p, \vmax, \vmin, E, v_0, t)$ be a $\beta$-stopping simple 
  stochastic game. And $S, T$ the pure strategy stationary strategy sets
  for the maxmimizer and minimizer respectively.
  Then for each $i \in V$,
  \begin{align*}
    \max_{\sigma \in S} \min_{\tau \in T} v_{\sigma, \tau} (i) = 
  \min_{\tau \in T} \max_{\sigma \in S} v_{\sigma, \tau} (i).
  \end{align*}
  Further if $q_i^* = \max_{\sigma \in S} \min_{\tau \in T} v_{\sigma, \tau} (i)$
  then for each $i \in V$, $q_i^* \in \mathbb{Q}$.
\end{theorem}
The trick is then to construct a monotone function $F : [0, 1]^d \to [0, 1]^d$
such that $F(x) = x$ if and only if $x = (q_1, ..., q_d)$.
\begin{definition}[$\beta$-stopping Simple Stochastic Game Monotone Function]
  Let $G = (V, V_p, \vmax, \vmin, E, v_0, t)$ be a $\beta$-stopping simple stochastic game, 
  $d = |V|$, and $(v_i)_{i \in [d]}$ some enumeration of the vertices in $V$. 
  The \emph{$\beta$-stopping simple stochastic game monotone function} is a function
  $F : [0, 1]^d \to [0, 1]^d$ defined coordinatewise as,
  \begin{align*}
    F((x_1, ..., x_d))_i = 
      \left( 1 - \beta \right) \cdot
      \begin{cases}
        1, & v_i = t \\
        \max\{x_j : (v_i, v_j) \in E\}, & v_i \in \vmax \\
        \min\{x_j : (v_i, v_j) \in E\}, & v_i \in \vmin \\
        \sum_{v_j \in V} p(v_i, v_j) \cdot x_j, & v_i \in V_p.
      \end{cases}
  \end{align*}
\end{definition}
This is really what you would expect from the function corresponding to a $\beta$-stopping
simple stochastic game; the value of a maximizer (minimizer) vertex is the maximum (minimum)
of it's successors, the value of a chance node is a weighted average of the value of the successors
according to the probability of transitioning, and the value of the target is 1. The factor
of $(1 - \beta)$ at the front corresponds exactly to the fact the game halts with probability $\beta$
at every step.
\begin{lemma}\label{ssgUnique}
  Let $F : [0, 1]^d \to [0, 1]^d$ be a \emph{$\beta$-stopping simple stochastic game monotone function} and $x, x' \in [0, 1]^d$.
  Then $F$ has a \emph{unique} fixpoint $x^* \in [0, 1]^d$.
\end{lemma}
\begin{proof}
  It can be calculated that for all $x, x' \in [0, 1]^d$ if $\|\cdot \|_\infty$ is the $\ell^\infty$ norm then
  $\|F(x) - F(x')\|_\infty \leq (1 - \beta)\|x - x'\|_{\infty}$. By definition $\beta$ is in the range $(0, 1]$, so 
  by the banach fixpoint theorem $F$ has a unique
  fixpoint.
\end{proof}
\begin{lemma}
  Let $F : [0, 1]^d \to [0, 1]^d$ be a $\beta$-stopping simple stochastic game monotone function. 
  Then $F$ is monotone in the usual coordinatewise ordering.
\end{lemma}
\begin{proof}
  Fix some enumeration $(v_i)_{i \in [d]}$ of $V$ and $x, x' \in [0, 1]^d$ with $x \leq x'$.
  Then for each $i \in [d]$ I consider cases on $v_i$. If $v_i = t$ then $F(x)_i = F(x')_i$ and
  the claim holds. If $v_i \in \vmax$ suppose that $F(x)_i > F(x')_i$. 
  Then for some $j \in [d]$ with $(v_i, v_j) \in E$ I have for all $j' \in [d]$
  with $(v_i, v_{j'}) \in E$ that $x_j > x_{j'}$. I can put $j' = j$ to find $x_j > x'_{j}$ which is a contradiction.
  The case $v_i \in \vmin$ is similar, and if $v_i \in V_p$ then,
  \begin{align*}
    F(x)_i = (1 - \beta) \sum_{v_j \in V} p(v_i, v_j) x_j \leq (1 - \beta) \sum_{v_j \in V} p(v_i, v_j) x'_j = F(x')_i.
  \end{align*}
\end{proof}
It is shown in \citep{condon} that if $q^* = (q_1^*, ..., q_d^*)$ then $q^* = F(q^*)$, from which it follows
by \cref{ssgUnique} that $q^*$ is the unique fixpoint of $F$. The last stage in reduction to $\trsk$
is to discretize the function $F : [0, 1]^d \to [0, 1]^d$ to a function $f : [N]_0^d \to [N]_0^d$
such that a fixpoint of $f$ can be converted to an approximate fixpoint of $F$.
This is proved using ideas from \citep{nashComp} and \citep{lowerBound} in the following.
\begin{lemma}[Discretized $\beta$-stopping Simple Stochastic Game Monotone Function]
 Let $G = (V, V_p, \vmax, \vmin, E, v_0, t)$ be a $\beta$-stopping simple 
  stochastic game,
  $F : [0, 1]^d \to [0, 1]^d$ it's corresponding monotone function, and 
  $\varepsilon \in \rpos$.
  Let $M = \left \lfloor \frac{1}{\beta \varepsilon}\right \rfloor$
  and define 
  $f : [M]_0^d \to [M]_0^d$ coordinatewise with 
  $f(x)_i = \left \lfloor F(\beta \varepsilon \cdot x)_i \cdot 
    \frac{1}{\beta \varepsilon}\right \rfloor$.
    If $x^* \in [0, 1]^d$ is the unique fixpoint of $F$
    and $x = f(x)$ then $\|\beta \varepsilon \cdot x - x^*\|_\infty < \varepsilon$.
    Further, $f$ is monotone.
\end{lemma}
\newcommand{\bed}{\beta \varepsilon \cdot}
\begin{proof}
  Let $x \in [M]_0^d$, suppose $x = f(x)$ and $F(x^*) = x^*$.
  From $x = f(x)$ I have by definition of $\lfloor \cdot \rfloor$
  for each $i \in [d]$,
  \begin{align*}
  1 > 
    F(\beta \varepsilon \cdot x)_i \cdot \frac{1}{\beta \varepsilon} - x_i
  \geq 0.
  \end{align*}
  This implies by definition of $\|\cdot\|_\infty$ and homogoneity of the norm that,
  \begin{align*}
    \|\beta \varepsilon \cdot x - F(\beta \varepsilon \cdot x) \|_\infty < \beta \varepsilon.
  \end{align*}
  So I calculate,
  \begin{align*}
    \|\bed x - x^* \|_\infty &\leq 
     \| \bed x - F(\bed x)\|_\infty + \|F(\bed x) - x^* \|_\infty  \\
    &= \| \bed x - F(\bed x)\|_\infty + \|F(\bed x) - F(x^*) \|_\infty  \\
    &< \beta \varepsilon + (1 - \beta)\|\bed x - x^*\|_\infty.
  \end{align*}
  Rearranging and dividing through by $\beta$ gives the first part of the claim.
  That $f$ is monotone easily follows from monotonicity of $F$, and that $\lfloor \cdot \rfloor$ and
  multiplication by non-negative constants preserve monotonicity.
\end{proof}
The culmination of this section is the following result.
\begin{theorem}
  Let $G$ be a $\beta$-stopping simple stochastic game and $q^* \in \Q$ the value of the game.
  For all $\varepsilon \in \rpos$ finding a $q \in \Q$ such that $|q - q^*| < \varepsilon$
  is polynomial time reducible to $\trsk$.
\end{theorem}
\begin{remark}
  In \citep{lowerBound} Etessami et al. show a stronger result; in a slightly more general
  model of simple stochastic games that don't necessarily have the $\beta$ stopping property used above,
  the problem of finding the exact value of the game $q^*$ is polynomial-time reducible to $\trsk$.
  The simplified stopping game approximation result is shown above instead to simplify exposition,
  as well as to simplify implementation\footnote{Computing the exact value of non-stopping games requires working with
  numbers on the order of $2^{\Omega(n^2)}$, where $n$ is the number of vertices in the graph. For $n = 15$ for example this requires $225$ bits
  of precision, which is larger than fits in a machine word on my laptop.} during the practical testing chapter of this dissertation.
\end{remark}
%notes: make less brackets, particularly in the introducotry part
% at a citation at the bit describing value of the game? i.e. 'the resultant
% process is easily seen to be a markov chain'
