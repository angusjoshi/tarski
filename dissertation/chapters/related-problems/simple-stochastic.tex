\section{Simple Stochastic Games}
Simple stochastic games, as defined in \citep{condon}, are a class of
zero-sum games played on graphs with two players called the maximizer and minimizer
respectively. The vertices of the graph are partitioned into three disjoint
subsets with one for each player and one for chance, and there is a designated
max sink and min sinks within the vertex set where the game terminates.
A token is placed on an initial vertex and at each stage of the game,
if the current vertex belongs to a particular player they may choose an outgoing edge to transition through.
If the current node belongs to chance then the next vertex is chosen randomly according to the distribution
of the weights of the current vertex's outgoing edges, which are defined to be non-negative and sum to one.
The goal of the maximizer is then to maximize the probability that the game reaches the max sink, and similarly
the minimizer tries to maximize the probability that the game reaches the min sink or never halts (or equivalently to minimize the
probability that it reaches the max sink). \\
Condon shows in \citep{condon} that these games necessarily have 
