\section{Simple Stochastic Games}
Simple stochastic games, as defined in \citep{condon}, are a class of
zero-sum games played on graphs with two players called the maximizer and minimizer
respectively. For the purposes of this dissertation I will consider only $\beta$-stopping
simple stochastic games (the meaning of which will be defined in turn).
Condon shows in \citep{condon} that these games necessarily have a rational value for rationally described instances
of the problem, and further that
the value can be achieved in pure stationary strategies (all of these concepts will be made precise).
The relationship to $\trsk$ then comes from \citep{lowerBound}, where Etessami et al. show
that computing the \emph{exact} value of (not necessarily $\beta$-stopping) simple stochastic games as well as a pure stationary strategy profile
to achieve this value is 
polynomial-time reducible to $\trsk$. This section will lay out the required definitions,
and describe the aforementioned reduction to $\trsk$ in the special case of $\beta$-stopping simple stochastic games.

\newcommand{\vmax}{V_{\max}}
\newcommand{\vmin}{V_{\min}}
\begin{definition}[Simple Stochastic Game]
  A \emph{simple stochastic game} is a directed graph $G = (V, V_p, \vmax, \vmin, E, v_0, t, \beta)$ with designated start vertex $v_0 \in V$,
  target vertex $t \in V$, $\beta \in (0, 1] \cap \Q$,
  a partition of $V \setminus \{t\}$ into three disjoint subsets $V_p, V_{\min}, V_{\max}$,
  and a mapping $p : V_p \times V \to [0, 1]$ such that for all $v_p \in V_p$,
  $v \in V$ if $(v_p, v) \not\in E$ I have $p(v_p, v) = 0$, for each
  $v_p \in V_p$ $\sum_{v \in V} p(v_p, v) = 1$, and for every $v \in V \setminus \{t\}$ there
  is necessarily an edge $(v, w) \in E$ for some $w \in V$. A \emph{play} in a simple stochastic game
  transpires as follows. A token is placed on the initial vertex of the game $v_0$.
  Let $v_i$ be the vertex on which the token currently lies. Then at each step, the game halts with probability $\beta$. If it did not halt and $v_i \in \vmax \; (\vmin)$
  then the maximizer (minimzer) chooses and edge $(v_i, v_{i + 1}) \in E$ for some $v_{i + 1} \in V$. If
  $v_i \in V_p$ then an edge $(v_i, w) \in E$ is chosen randomly with probability $p(v_i, w)$. If $v_i = t$
  then the game halts. The payoffs of to the maximizer is $1$ if the game reaches $t$, and $0$ otherwise.
  The minimizer achieves the negative payoff of the maximizer.
\end{definition}

\begin{figure}[h]
  \centering
  \tikzset{every picture/.style={line width=0.75pt}} %set default line width to 0.75pt        
\begin{tikzpicture}[x=0.6pt,y=0.6pt,yscale=-1,xscale=1,
  main/.style = {draw, circle, 
  fill={rgb, 255:red, 0; green, 0; blue, 0 }, text=white},
  node distance=3cm,
  ->,
  >={Stealth[round,sep]},
  legend/.style={draw, circle,
  fill={rgb, 255:red, 0; green, 0; blue, 0 },
  text=black
  }
  ]

  \node[main] (v0) {$v_0$};
  \node[main, shape=diamond] (v1) [right of=v0] {$v_1$};
  \node[main, shape=rectangle] (v2) [below right of=v0] {$v_2$};
  \node[main] (v3) [right of=v1] {$v_3$};
  \node[main, shape=star] (t) [below right of=v3] {$t$};

  \draw (v0) -- (v1) node[near start, below right, text=black]{$\frac{1}{3}$};
  \draw (v0) -- (v2) node[near start, below, text=black]{$\frac{2}{3}$};

  \draw (v3) -- (t) node[near start, below left, text=black]{$\frac{2}{5}$};
  \draw (v3) -- (v2) node[near start, below right, text=black]{$\frac{2}{5}$};
  \draw (v3) to[bend left=45, min distance=1cm] node[near start, above right, text=black]{$\frac{1}{5}$} (v0);

  \draw (v1) -- (v3);
  \draw (v1) -- (v2); 

  \draw (v2) edge[bend right=60, min distance=1.5cm] (v0);
  \draw (v2) -- (t);

  \matrix [draw] at (current bounding box.north east) {
    \node[legend, label=right:\scriptsize Chance] {}; \\
    \node[legend, shape=diamond,label=right:\scriptsize Maximizer] {}; \\
    \node[legend, shape=rectangle,label=right:\scriptsize Minimizer] {}; \\
  };

\end{tikzpicture}

  \caption{In $\beta$-stopping simple stochastic games, one of the two players aims to maximize the probability
  of play reaching the target $t$, while the other aims to minimize it.}
\end{figure}

It is clear that since $\beta > 0$ the game eventually halts with probability $1$. 
\begin{definition}[Pure Stationary Strategy]
  Let $G = (V, V_p, \vmax, \vmin, E, v_0, t)$ be a simple stochastic game. A \emph{pure stationary strategy}
  for the maximizer is a mapping $\sigma : \vmax \to V$ with the requirement that for all $v \in \vmax$
  $(v, \sigma(v)) \in E$. The set of all such pure stationary strategies for the maximizer is denoted
  $S$. A pure stationary strategy for the minimizer is a map $\tau : \vmin \to V$ such that
  for all $v \in \vmin$ 
  $(v, \tau(v)) \in E$. The set of all such pure stationary strategies for the minimizer is denoted $T$. 
  A \emph{pure stationary strategy profile} is a pair $(\sigma, \tau) \in S \times T$.
\end{definition}
\begin{theorem}[\citep{condon}, lemma 6]
  Let $G$ be a $\beta$-stopping simple stochastic game.
  Then $\max_{\sigma \in S} \min_{\tau \in T}$
\end{theorem}
