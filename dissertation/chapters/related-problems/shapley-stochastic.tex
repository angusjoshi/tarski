\section{Shapley's Stochastic Games}
Shapley's stochastic games, or just stochastic games as described in \citep{shapley}
are a class of zero-sum game played on a set of states with two players
called the maximizer and minimizer respectively.
At each state the players concurrently choose an action,  
and then receive a payoff which sums to zero based on the joint
action of the two players. The game then halts with some fixed probability.
If it didn't halt then a next state is chosen randomly with probability distribution
dependent on the joint action chosen.
It is shown in \citep{shapley} that stochastic games necessarily have an optimal expected value,
and further that this value can be achieved in stationary strategies. Unlike
simple stochastic games however, the value need not be rational or the strategies
achieving this value pure. In \citep{lowerBound} it is shown that the  problem
of finding a rational number $\varepsilon$ close to the actual value of the game
is polynomial-time reducible to $\trsk$ which will be described in this section.
\begin{figure}[h]
  \centering
  \tikzset{every picture/.style={line width=0.75pt}} %set default line width to 0.75pt        
\begin{tikzpicture}[x=0.6pt,y=0.6pt,yscale=-1,xscale=1,
  main/.style = {draw, circle, 
   text=black},
  node distance=6cm,
  ->,
  >={Stealth[round,sep]},
  legend/.style={draw, circle,
  fill={rgb, 255:red, 0; green, 0; blue, 0 },
  text=black}]

  \node[main, circle split] (v0) 
    {$v_1$ \qquad $\begin{bmatrix}
    1 & 2 \\ 
    3 & 4
    \end{bmatrix}$ 
    \nodepart{lower} $\begin{bmatrix}
      \begin{bmatrix} 0 & \frac{1}{3} & \frac{2}{3} \end{bmatrix}
        & \begin{bmatrix} \frac{1}{5} & \frac{2}{5} & \frac{2}{5}\end{bmatrix} \\
          \begin{bmatrix}\frac{2}{3} & \frac{1}{3} & 0 \end{bmatrix} 
            & \begin{bmatrix} 0 & \frac{1}{7} & \frac{6}{7}\end{bmatrix}
    \end{bmatrix}$};
  \node[main, circle split] (v1) [below right of=v0]
    {$v_2$ \qquad $\begin{bmatrix}
    -4 & 1 \\ 
    -1 & 5
    \end{bmatrix}$ 
    \nodepart{lower} $\begin{bmatrix}
      \begin{bmatrix} \frac{2}{7} & \frac{1}{7} & \frac{4}{7} \end{bmatrix}
        & \begin{bmatrix} \frac{1}{3} & 0 & \frac{2}{3}\end{bmatrix} \\
          \begin{bmatrix}\frac{1}{4} & \frac{1}{2} & \frac{1}{4} \end{bmatrix} 
            & \begin{bmatrix} \frac{1}{5} & \frac{1}{5} & \frac{2}{5}\end{bmatrix}
    \end{bmatrix}$};
  \node[main, circle split] (v2) [above right of=v1]
    {$v_3$ \qquad $\begin{bmatrix}
    1 & -1 \\ 
    3 & 0
    \end{bmatrix}$ 
    \nodepart{lower} $\begin{bmatrix}
      \begin{bmatrix} \frac{1}{4} & \frac{1}{4} & \frac{1}{2} \end{bmatrix}
        & \begin{bmatrix} 0 & \frac{1}{2} & \frac{1}{2}\end{bmatrix} \\
          \begin{bmatrix}\frac{2}{3} & 0 & \frac{1}{3} \end{bmatrix} 
            & \begin{bmatrix} \frac{1}{7} & \frac{6}{7} & 0\end{bmatrix}
    \end{bmatrix}$};
  % \node[main, circle split] (v1) 
  %   {$\begin{bmatrix}1 & 2 \\ 3 & 4\end{bmatrix}$ \nodepart{lower} $x$};
  % \node[main, circle split] (v2) 
  %   {$\begin{bmatrix}1 & 2 \\ 3 & 4\end{bmatrix}$ \nodepart{lower} $x$};

  % \draw (v0) -- (v1) node[near start, below right, text=black]{$\frac{1}{3}$};
  % \draw (v0) -- (v2) node[near start, below, text=black]{$\frac{2}{3}$};

  % \draw (v3) -- (t) node[near start, below left, text=black]{$\frac{2}{5}$};
  % \draw (v3) -- (v2) node[near start, below right, text=black]{$\frac{2}{5}$};
  % \draw (v3) to[bend left=45, min distance=1cm] node[near start, above right, text=black]{$\frac{1}{5}$} (v0);

  % \draw (v1) -- (v3);
  % \draw (v1) -- (v2); 

  % \draw (v2) edge[bend right=60, min distance=1.5cm] (v0);
  % \draw (v2) -- (t);
\end{tikzpicture}

  \caption{The goal of the arrival problem is to decide whether a partiular walk on a directed graph with a particular
  structure reaches the target. On successive visits to a particular
  vertex the outgoing edge taken alternates. In this example
  the walk begins $s \to v_1 \to v_4 \to s \to v_3 \to \ldots$.} 
\end{figure}
\begin{definition}[Stochastic Game]
  A \emph{stochastic game} $G = (V)$
\end{definition}
